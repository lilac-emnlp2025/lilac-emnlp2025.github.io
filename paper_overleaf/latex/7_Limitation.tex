\section{Limitations}


Our current approach focuses on effectively harmonizing pre-trained multimodal models to achieve enhanced retrieval performance without additional fine-tuning. 
Consequently, the accuracy of our retrieval method significantly depends on the quality of subcomponent extraction. %, especially within image and table modalities.
% As demonstrated in our empirical analysis (e.g., with the \texttt{InfoVQA} dataset), inaccuracies during subcomponent extraction can negatively affect retrieval quality. 
Also, although our retrieval accuracy surpasses existing methods, there remains substantial room for improvement in end-to-end generation tasks. 
% This highlights the necessity of developing more sophisticated integration strategies between retrieval and generation components.



\section*{Acknowledgements}

\updated{
This work was partly supported by the National Research Foundation of Korea(NRF) grant funded by the Korea government(MSIT) (RS-2025-00517736, 50\%), Institute of Information \& communications Technology Planning \& Evaluation (IITP) grant funded by the Korea government(MSIT) (No. RS-2024-00509258, Global AI Frontier Lab, 30\%) (No. RS-2018-II181398, Development of a Conversational, Self-tuning DBMS, 10\%) (No.  RS-2024-00454666, Developing a Vector DB for Long-Term Memory Storage of Hyperscale AI Models, 5\%), and Basic Science Research Program through the National Research Foundation of Korea Ministry of Education(No. RS-2024-00415602, 5\%).
}

% 우리는 현재 존재하는 pre-trained multimodal model들을 effective하게 사용하는 방법에 대해 approach한다.
% Future work에는,  domain-specific하게 late interaction을 학습시키는 방식에 대해 연구하고자 한다.
% 또한, 우리의 방식은 subcomponent extraction method의 성능에 영향을 받는다. 
% Retrieval accuracy는 높지만, end-to-end generation을 위한 room이 존재한다.

% Our current approach focuses on effectively leveraging existing pre-trained multimodal embedding models without fine-tuning. 
% Consequently, the performance of \texttt{LILaC} inherently relies on the quality and generalizability of these embedding models. 
% An important direction for future research involves domain-specific fine-tuning of the late interaction component, which could further enhance retrieval precision by better capturing domain nuances. 
% Our current approach focuses on harmonizing multimodal models in their pre-trained nature to 달성하다 더 좋은 effectivity를.
% Thus, the accuracy of our retrieval method significantly depends on the subcomponent extraction process, particularly for image and table modalities. 
% Imperfections or inaccuracies during subcomponent extraction can directly impact retrieval quality, as seen in our empirical analysis with certain datasets (e.g., InfoVQA).
% Lastly, while our retrieval accuracy outperforms existing approaches, there remains substantial room for improvement in end-to-end generation tasks, emphasizing the need for more sophisticated integration between retrieval and generation components.