%%%%%%%% ICML 2026 EXAMPLE LATEX SUBMISSION FILE %%%%%%%%%%%%%%%%%

\documentclass{article}

% Recommended, but optional, packages for figures and better typesetting:
%\usepackage{microtype}
\usepackage{graphicx}
%\usepackage{subfigure}
\usepackage{booktabs} % for professional tables

% hyperref makes hyperlinks in the resulting PDF.
% If your build breaks (sometimes temporarily if a hyperlink spans a page)
% please comment out the following usepackage line and replace
% \usepackage{icml2026} with \usepackage[nohyperref]{icml2026} above.
\usepackage{hyperref}
\usepackage{booktabs} % For professional quality tables
\usepackage{multirow} % For multi-row cells, though not used in this version, it's good to have
\usepackage{siunitx}  % For aligning numbers by decimal point, optional but good practice
\usepackage{graphicx}   % For including images
\usepackage{subcaption} % For creating subfigures (a) and (b)
\usepackage{float}
\usepackage{tabularx}

% Attempt to make hyperref and algorithmic work together better:
\newcommand{\theHalgorithm}{\arabic{algorithm}}

% Use the following line for the initial blind version submitted for review:
\usepackage{icml2026}
% \usepackage[preprint]{icml2026}

% If accepted, instead use the following line for the camera-ready submission:
%\usepackage[accepted]{icml2026}

% For theorems and such
\usepackage{amsmath}
\usepackage{amssymb}
\usepackage{mathtools}
\usepackage{amsthm}
% \usepackage[breakable]{tcolorbox}
\usepackage{longtable}
\usepackage{booktabs} 


% if you use cleveref..
\usepackage[capitalize,noabbrev]{cleveref}



%%%%%%%%%%%%%%%%%%%%%%%%%%%%%%%%%%%%%%%%%%%%%%%%%%%%%%%%%
%%%%%%%%%%%%%%%%%% User-added packages %%%%%%%%%%%%%%%%%%
%%%%%%%%%%%%%%%%%%%%%%%%%%%%%%%%%%%%%%%%%%%%%%%%%%%%%%%%%

\usepackage{graphicx}
\usepackage{kotex}
\usepackage{booktabs}                                       % nicer horizontal rules
\usepackage{multirow}                                       % \multirow
\usepackage{array}                                          % better column specifiers
\usepackage{fontawesome}                                    % GitHub 아이콘
\usepackage[table]{xcolor}                                  % already in most NeurIPS templates
\usepackage{colortbl}                                       % if not loaded by another package
\usepackage{hyperref}
% \usepackage{tcolorbox}                                      % 뱃지용

\usepackage[most]{tcolorbox}
\tcbuselibrary{breakable,skins,listingsutf8}
\usepackage{listings}

% Make long lines wrap nicely (no overflow)
\lstdefinestyle{promptlisting}{
  basicstyle=\ttfamily\footnotesize,
  columns=fullflexible,
  keepspaces=true,
  showstringspaces=false,
  tabsize=2,
  breaklines=true,
  breakatwhitespace=false
}

% A tcolorbox environment that (1) wraps long titles, (2) wraps long code lines, (3) stays within \linewidth
\newtcblisting{promptbox}[2][]{
  enhanced,
  breakable,
  width=\linewidth,
  colback=gray!10,
  colframe=black,
  boxrule=0.5mm,
  sharp corners,
  left=1.5mm,right=1.5mm,top=1mm,bottom=1mm,
  % Title bar styling
  colbacktitle=black,
  coltitle=white,
  fonttitle=\ttfamily\small,
  title={\parbox{\dimexpr\linewidth-3mm\relax}{\raggedright #2}}, % <-- wraps long titles
  % Listing (verbatim-like) content with line breaking
  listing only,
  listing options={style=promptlisting},
  % allow extra options per box
  #1
}

%%%%%%%%%%%%%%%%%%%%%%%%%%%%%%%%%%%%%%%%%%%%%%%%%%%%%%%%%
%%%%%%%%%%%%%%%%% User-defined commands %%%%%%%%%%%%%%%%%
%%%%%%%%%%%%%%%%%%%%%%%%%%%%%%%%%%%%%%%%%%%%%%%%%%%%%%%%%

\newcommand{\yes}{\checkmark}               % ✓  (requires amssymb or pifont if you prefer)
\newcommand{\no}{\phantom{\checkmark}}      % keep cell width when blank
\newcommand{\customcirc}{\raisebox{-0.6mm}{\scalebox{1.6}{\(\circ\)}}}
\newcommand*\circled[1]{\tikz[baseline=(char.base)]{
            \node[shape=circle,draw,inner sep=0.5pt] (char) {#1};}}
            

\definecolor{linkpink}{HTML}{E91E63} % 예: Material Pink 500

\newtcbox{\codebadge}{
  on line,
  arc=2.2pt,
  colback=linkpink!8,
  colframe=linkpink,
  colupper=linkpink,
  boxrule=0.4pt,
  left=6pt,right=6pt,top=2pt,bottom=2pt
}

\newcommand{\squishlist}{
   \begin{list}{$\bullet$}
    { \setlength{\itemsep}{1pt}
      \setlength{\parsep}{0pt}
      \setlength{\topsep}{2pt}
      \setlength{\partopsep}{0pt}
      \setlength{\listparindent}{-2pt}
      \setlength{\itemindent}{-5pt}
      \setlength{\leftmargin}{1.5em}
      \setlength{\labelwidth}{0em}
      \setlength{\labelsep}{0.5em}
    }
}
\newcommand{\squishlistend}{
    \end{list}  }
\newcommand{\squishend}{
    \end{list}  }

\makeatletter
\setlength{\@dblfptop}{0pt}
\setlength{\@dblfpsep}{6pt}  % space between two table*'s on a float page
\setlength{\@dblfpbot}{0pt}
\makeatother

\newcommand{\Atrav}{\mathcal{A}_{\text{trav}}}
% \Atrav \triangleq \{\textsc{Traverse},\textsc{Backtrack}\}.

%%%%%%%%%%%%%%%%%%%%%%%%%%%%%%%%%%%%%%%%%%%%%%%%%%%%%%%%%
%%%%%%%%%%%%%%%%%%%%%%%%%%%%%%%%%%%%%%%%%%%%%%%%%%%%%%%%%
%%%%%%%%%%%%%%%%%%%%%%%%%%%%%%%%%%%%%%%%%%%%%%%%%%%%%%%%%





















%%%%%%%%%%%%%%%%%%%%%%%%%%%%%%%%
% THEOREMS
%%%%%%%%%%%%%%%%%%%%%%%%%%%%%%%%
\theoremstyle{plain}
\newtheorem{theorem}{Theorem}[section]
\newtheorem{proposition}[theorem]{Proposition}
\newtheorem{lemma}[theorem]{Lemma}
\newtheorem{corollary}[theorem]{Corollary}
\theoremstyle{definition}
\newtheorem{definition}[theorem]{Definition}
\newtheorem{assumption}[theorem]{Assumption}
\theoremstyle{remark}
\newtheorem{remark}[theorem]{Remark}
\usepackage{capt-of} % figure* 대신 caption만 뽑아서 쓸 수 있게

% Todonotes is useful during development; simply uncomment the next line
%    and comment out the line below the next line to turn off comments
%\usepackage[disable,textsize=tiny]{todonotes}
\usepackage[textsize=tiny]{todonotes}
\usepackage{dsfont}
\usepackage{xspace}

\newcommand{\wshan}[1]{{\color{red}#1}}
\newcommand{\dylee}[1]{{\color{blue}#1}}
\newcommand{\jhyun}[1]{{\color{magenta}#1}}

\newcommand{\new}[1]{{\color{blue}#1}}
\newcommand{\revised}[1]{{\color{orange}#1}}
\newcommand{\jhyuntodo}[1]{{\color{red}#1}}







\newcommand{\Ours}{FiF\xspace}
\newcommand{\OurFullName}{Failure is Feedback\xspace}
\newcommand{\Title}{History-Aware Backtracking for Agentic Traversal in Multimodal Graphs\xspace}

\newcommand{\MultimodalQA}{\textsc{MultimodalQA}$^{\texttt{Doc}}$\xspace}
\newcommand{\MMCoQA}{\textsc{MMCoQA}$^{\texttt{Doc}}$\xspace}
\newcommand{\WebQA}{\textsc{WebQA}$^{\texttt{Doc}}$\xspace}






% The \icmltitle you define below is probably too long as a header.
% Therefore, a short form for the running title is supplied here:
%\icmltitlerunning{Submission and Formatting Instructions for ICML 2026}


\icmltitlerunning{\textsc{\OurFullName}: \Title}
\icmlsetsymbol{advising}{$\dagger$}

\newcommand{\icmlEqualAdvising}{\textsuperscript{$\dagger$} Advising}

































\begin{document}


\twocolumn[{
\icmltitle{
\texorpdfstring
{\textsc{\OurFullName}: \Title}
{\textsc{\OurFullName}: \Title}
}


\icmlsetsymbol{equal}{*}
\icmlsetsymbol{advising}{, $\dagger$}



\begin{icmlauthorlist}

\icmlauthor{Joohyung Yun}{postech_cse}
\icmlauthor{Doyup Lee}{directorlabs}
\icmlauthor{Wook-Shin Han}{postech_gsai,advising}

\end{icmlauthorlist}



\icmlaffiliation{postech_gsai}{GSAI, POSTECH}
\icmlaffiliation{postech_cse}{CSE, POSTECH}
\icmlaffiliation{directorlabs}{DirectorLabs}
\icmlcorrespondingauthor{Joohyung Yun}{jhyun@dblab.postech.ac.kr}
\icmlkeywords{Machine Learning, ICML}




% Center image

% \begin{center}
%   \begin{minipage}{0.9\linewidth}
%     \centering
%     \includegraphics[width=\linewidth]{figures/Figure1.pdf}
%     \vspace{-0.4cm}
%     {\captionsetup{hypcap=false}%
%     \captionof{figure}{%
%     \textbf{Manipulating personal objects with VLA.} Existing vision-language-action (VLA) models cannot handle personal objects such as \texttt{<my cup>}, because they can only interpret generic, language-expressible semantics. We address this limitation with a simple framework, \lname (\Ours). It first grounds the user-specific object in the scene by matching it against the memory and then uses visual prompting to guide the VLA. This pipeline enables existing VLA models to manipulate personal objects without any additional training.
%     }
%     \label{fig:teaser}}
%   \end{minipage}
%   \vspace{-0.6cm}
% \end{center}





% =======================
\vskip 0.3in
}]

% this must go after the closing bracket ] following \twocolumn[ ...

% This command actually creates the footnote in the first column
% listing the affiliations and the copyright notice.
% The command takes one argument, which is text to display at the start of the footnote.
% The \icmlEqualContribution command is standard text for equal contribution.
% Remove it (just {}) if you do not need this facility.

%\printAffiliationsAndNotice{}  % leave blank if no need to mention equal contribution
%\printAffiliationsAndNotice{\icmlEqualContribution} % otherwise use the standard text.

\begin{NoHyper}
\printAffiliationsAndNotice{\icmlEqualAdvising}
%\printAffiliationsAndNotice{} % otherwise use the standard text.
\end{NoHyper}


\begin{abstract}

Multimodal document retrieval aims to retrieve query-relevant components from documents composed of textual, tabular, and visual elements. 
An effective multimodal retriever needs to handle two main challenges:
(1) mitigate the effect of irrelevant contents caused by fixed, single-granular retrieval units, and 
(2) support multihop reasoning by effectively capturing semantic relationships among components within and across documents. 
To address these challenges, we propose \texttt{LILaC}, a multimodal retrieval framework featuring two core innovations. 
First, we introduce a \textit{layered component graph}, explicitly representing multimodal information at two layers---each representing coarse and fine granularity---facilitating efficient yet precise reasoning. 
Second, we develop a \textit{late-interaction-based subgraph retrieval} method, an edge-based approach that initially identifies coarse-grained nodes for efficient candidate generation, then performs fine-grained reasoning via late interaction.
\updated{Extensive experiments demonstrate that \texttt{LILaC} achieves state-of-the-art retrieval performance on all five benchmarks, notably without additional fine-tuning.
We make the artifacts publicly available at \href{https://github.com/joohyung00/lilac}{\textcolor{linkpink}{\texttt{github.com/joohyung00/lilac}}}.
}
\end{abstract}



\section{Introduction}
\label{sec:introduction}
\vspace{-0.5em}


\begin{figure}[t]
  \centering
  \includegraphics[width=\linewidth]{figures/introduction.pdf}
  \vspace{-4mm}
  \caption{
    Motivating examples of multihop retrieval failures in existing graph retrieval approaches.
    (a) Vector-similarity-driven traversal follows a spurious cue.
    (b) Fixed retrieval plan produces an underspecified hop and fails to recover from a dead end.
  }
  \label{fig:motivating_example_figure}
  \vspace{-6mm}
\end{figure}



% Paragraph 1) Motivation of Multimodal Document RAG
Searching the web has become a part of everyday life.
This routine increasingly underpins multimodal retrieval-augmented generation (RAG), where a model answers a user query by grounding its output in retrieved evidence~\cite{4_colpali}.
In practice, much of this evidence lives in webpages or PDFs---multimodal documents with three salient characteristics:
(i) each document is composed of multimodal \textit{components} (paragraphs, tables, and images);
(ii) the meaning of a component is often shaped by local document context (e.g., captions and surrounding components); and 
(iii) components are connected through explicit signals (hyperlinks, cross-references) as well as implicit signals (e.g., same-section adjacency).
Moreover, documents themselves are linked via hyperlinks and citations, forming a large graph that users implicitly navigate while browsing.
We refer to the resulting setting as \emph{open-domain multimodal document retrieval} (OMDR): given a query, the system must return a small ranked set of relevant components from this large, noisy, and interlinked graph, often requiring multihop and multimodal exploration~\cite{1_lilac, 9_ircot, 4_colpali}.



% Paragraph 2) Transition to Graph-based Methods 
Given these intricate characteristics of OMDR, representing a document collection as a graph has emerged as a powerful paradigm for capturing the multi-granularity and interconnectedness of multimodal evidence~\cite{1_lilac}.
It shows the pros of preserving the structural dependencies and navigational scaffolds inherent in webpages, which is required when navigating heterogeneous components.
The most recent work introduces the \textit{layered component graph}, which organizes components together with their constituent subcomponents (sentences, table rows, and image objects)~\cite{1_lilac}.
In this formulation, \emph{navigational edges} encode relations among components (e.g., hyperlinks, same-section adjacency), while \emph{hierarchical edges} connect each component to its subcomponents.
By jointly modeling these edge types across layers, a retriever can traverse component-to-component paths for multihop exploration and move up/down the hierarchy to operate at the appropriate granularity.



% Paragraph 3) Their Limitations
While graph-based structures provide a rich representation of multimodal evidence, existing retrieval algorithms often struggle to fully exploit this potential due to operational rigidity.
In particular, (a) traversal is typically driven by a single, hop-agnostic embedding-based scoring rule and (b) executed with a largely pre-specified procedure, limiting dynamic error correction.
Figure~\ref{fig:motivating_example_figure} highlights these failures.
In Figure~\ref{fig:motivating_example_figure}(a), it follows a superficially related textual cue and retrieves an irrelevant snippet, failing to ground on the crucial visual evidence.
In Figure~\ref{fig:motivating_example_figure}(b), it issues an underspecified follow-up (``such battles'') and gets stuck in a dead end, rather than adapting its trajectory after the failure.
As a result, once the retriever follows a spurious edge or reaches a dead end, errors propagate across hops and degrade final retrieval quality~\cite{9_ircot, 10_selfrag}.
Moreover, these methods lack a principled mechanism for deciding \emph{when} expensive reasoning is warranted, leading to under-reasoning on ambiguous hops or over-spending computation across a trajectory.



% Paragraph 4) The Need for a Reasoning-Aware Sequential Decision Process
To overcome these limitations, we argue that a retriever must evolve from a static path-follower into an adaptive decision-maker that navigates the graph through a sequential reasoning process.
Concretely, OMDR is naturally stateful: as evidence accumulates, the information need shifts, and failures reveal which interpretations, routes, or strategies are unproductive.
This suggests casting traversal as a \emph{sequential decision process} over an evolving information state, where each step chooses (i) what to ask next (subquery), (ii) how to retrieve (tool/strategy), and (iii) where to move (edge type and granularity) conditioned on the current evidence.
Achieving this requires addressing three coupled challenges.
First, edge-following is not merely similarity matching; it often requires high-level reasoning to judge whether a candidate node will lead to the final answer under the current context.
Second, the retriever must adapt to evolving context by refining hypotheses and subqueries, and by recovering from dead ends using failure signals rather than adhering to a fixed, pre-defined plan.
Third, it must balance accuracy and efficiency: while LLM-based reasoning can improve retrieval precision, it introduces substantial overhead, so the system must decide economically when to escalate from lightweight matching to intensive reasoning.



% Paragraph 5) Our approach
To address these needs, we propose \textsc{\textbf{\OurFullName}} (\textsc{\Ours}).
We formalize OMDR as a finite-horizon \emph{information-state MDP}, where the state is a structured memory that records accumulated evidence together with the history of attempted subqueries, strategies, and explicit success/failure outcomes.
This formulation turns graph traversal into an \emph{economically-rational} agentic workflow: at each hop, an orchestrator dynamically decides \emph{what to ask}, \emph{how to retrieve}, and \emph{where to move} given the current information state, rather than executing a rigid traversal recipe.
To realize cost-sensitive control, \textsc{\Ours} maintains a portfolio of strategies across an accuracy--efficiency spectrum, starting from low-cost vector matching and escalating to higher-cost LLM reasoning only when a hop is ambiguous or an attempt fails.
Finally, to make multihop navigation resilient in noisy open-domain graphs, \textsc{\Ours} introduces \emph{history-aware backtracking}: unlike standard backtracking that simply reverts the state, our approach piggybacks on failure traces to re-anchor the search to a more promising prior context, revise subsequent subqueries, and avoid repeating previously failed routing patterns.




% Paragraph 6) Contributions
In summary, we make three primary contributions:
\vspace{-2mm}
\squishlist
    \item [1.] We formulate the OMDR problem as a sequential decision process with economic rationality. We redefine OMDR as an information-state MDP, operationalizing it through an LLM-enabled agentic workflow that treats retrieval strategy as a dynamic choice.
    \item [2.] We propose dynamic cost-aware strategy escalation.
    We introduce a novel mechanism that maintains a portfolio of strategies across an accuracy-efficiency spectrum. 
    Our orchestrator avoids over-reasoning by starting with low-cost vector matching and only escalating to high-cost LLM reasoning when a hop is identified as ambiguous or follows a recorded failure.
    \item [3.] We propose history-aware backtracking for resilient navigation, which converts failed traversals into constructive feedback. 
    By piggybacking on failure traces, the orchestrator re-anchors its search to prior contexts while revising its subqueries and escalating its strategy, enhancing both robustness and efficiency.
\squishend
\vspace{-2mm}












%%%%%%%%%%%%%% Version 5

% This capability is becoming a core primitive for multimodal RAG, as multimodal embedders and multimodal LLMs continue to improve in representing diverse modalities and reasoning over retrieved evidence~\cite{4_colpali}.



% While this graph-based structure provides a rich representation of the data, existing retrieval algorithms struggle to fully exploit its potential due to their operational rigidity.
% First, traversal is typically driven by a single, embedding-based scoring rule, which can miss hop-specific semantics that go beyond similarity.
% {\color{red} HI!}
% Second, traversal is often governed by a largely pre-specified procedure (e.g., a fixed sequence of subqueries executed with a fixed traversal routine), which limits dynamic error correction.
% As a result, once the retriever follows a spurious edge or gets trapped in a dead end, mistakes tend to propagate across hops and degrade final retrieval quality~\cite{9_ircot, 10_selfrag}.
% {\color{red} HI!}
% Crucially, these methods also lack a principled way to decide \emph{when} expensive reasoning is warranted: without explicit cost-awareness, they may either under-reason on ambiguous hops or over-spend computation across the trajectory.



% To overcome these limitations, we argue that a retriever must evolve from a static path-follower into an adaptive decision-maker that can navigate the graph through a sequential reasoning process.
% Achieving this requires addressing three key challenges.
% \textit{(1) Incorporating high-level reasoning.}
% The process of following an edge is not merely a similarity matching task; 
% it often requires complex logical deduction to determine whether a specific node will lead to the final answer based on the current evidence.
% \textit{(2) Adapting reasoning to evolving context.}
% As evidence is accumulated, the information need shifts; 
% thus, the retriever must dynamically refine its hypotheses and subqueries, and recover from dead ends by reflecting on failures rather than adhering to a fixed, pre-defined plan.
% \textit{(3) Balancing accuracy and efficiency.}
% While advanced reasoning via large models can improve retrieval precision, it introduces significant computational overhead.
% An effective system must maintain a delicate balance by orchestrating a spectrum of strategies, ranging from lightweight matching to intensive reasoning, depending on the difficulty of the current hop.


% To address these needs, we propose \textsc{\textbf{\OurFullName}} (\textsc{\Ours}), which introduces two key innovations:
% \textit{(1) LLM-enabled agentic traversal as a sequential decision process.}
% We cast graph traversal as a sequential decision process, where each hop decides \emph{what to ask} (subquery), \emph{how to retrieve} (tool/strategy), and \emph{where to move} (edge type/granularity) given the current evidence.
% To operationalize this formulation, we design an agentic framework in which each agent is explicitly equipped to invoke LLM reasoning when needed.
% Specifically, \textsc{\Ours} supports specialized tools (e.g., a planner and a multi-strategy traverser) coordinated by an orchestrator, enabling on-demand LLM reasoning and flexible switching across retrieval modes.
% \textit{(2) Adaptive accuracy--efficiency trade-offs with history-aware backtracking.}
% \textsc{\Ours} maintains a portfolio of strategy variants spanning an accuracy--efficiency spectrum, from low-cost vector matching for easy hops to high-cost LLM-intensive reasoning for ambiguous hops.
% Crucially, we incorporate a history-aware backtracking mechanism that turns failures into actionable feedback, making traversal both efficient and accurate.
% \emph{Efficiency:} the orchestrator begins with cheap strategies and escalates to stronger (but more expensive) reasoning only when a hop is difficult or an attempt fails.
% Upon reaching a dead end, the retriever backtracks while retaining failure traces to avoid redundant exploration and to revise the next hop (subquery/objective and strategy choice).
% \emph{Effectiveness:} these failure signals guide subsequent hops toward more plausible paths, improving multihop reliability without repeatedly paying for expensive reasoning.





%%%%%%%%%%% Version 4






% % Paragraph 1) Motivation of Multimodal Document RAG
% Searching the web has become part of everyday life.
% This routine increasingly underpins multimodal retrieval-augmented generation (RAG), where a model answers a user query by grounding its output in retrieved evidence~\cite{4_colpali}.
% In practice, much of this evidence lives in webpages or PDFs, which are multimodal documents that show the following characteristics
% (i) A document is composed of multimodal \textit{components}, mixing paragraphs, tables, and images.
% (ii) The meaning of each component is often shaped by local context within the document such as captions and surrounding components.
% (iii) Components are connected through explicit signals, including hyperlinks and cross-references, and through implicit signals, including same-section adjacency.
% Furthermore, documents are further linked to other documents through hyperlinks and citations, forming a large graph that users implicitly navigate while browsing.
% In such a situation, the \textit task asks a system to return a small ranked set of relevant components from this large, noisy, and interlinked graph, often requiring multihop and multimodal exploration~\cite{1_lilac, 9_ircot}.
% This capability is becoming a core primitive for multimodal RAG, as multimodal embedders and multimodal LLMs continue to improve in representing diverse modalities and reasoning over retrieved evidence~\cite{4_colpali}.




% % Paragraph 2) Transition to Graph-based Methods 
% Given these intricate characteristics of OMDR, representing a document collection as a graph structure has emerged as a powerful paradigm to capture the multi-granularity and interconnectedness of multimodal evidence~\cite{1_lilac}.
% Because OMDR requires navigating a web of heterogeneous components and their complex relational signals, a graph-based representation is suited to preserving the structural dependencies and navigational scaffolds inherent in webpages.
% The most recent work introduces \textit{layered component graph}, which explicitly organizes the collection of components and their constituent subcomponents (sentences, table rows, and image objects)~\cite{1_lilac}.
% In this formulation, \emph{navigational edges} encode relationships among components (e.g., hyperlinks, cross-references, and same-section adjacency), while \emph{hierarchical edges} connect each component to its subcomponents.
% By jointly modeling these edge types across layers, a retriever can traverse component-to-component paths for multihop exploration and move up/down the hierarchy to operate at the appropriate granularity, enabling multi-granularity retrieval and multihop reasoning within a unified graph.




% % Paragraph 3) Their Limitations
% However, while this graph-based structure provides a rich representation of the data, existing retrieval algorithms struggle to fully exploit its potential due to their operational rigidity.
% First, traversal is typically driven by a single, embedding-based scoring rule, which may not be sufficient to resolve some hop semantics that may go beyond similarity.
% Second, traversal is often governed by a largely pre-specified procedure (e.g., a fixed sequence of subqueries produced upfront and executed with a fixed traversal routine), which limits dynamic error correction.
% As a result, once the retriever follows a spurious edge or gets trapped in a dead end, mistakes tend to propagate across hops and degrade final retrieval quality~\cite{9_ircot, 10_selfrag}.



% % Paragraph 4) The Need for a Reasoning-Aware Sequential Decision Process
% {\color{red} 모두 문장으로 변경}
% To overcome these limitations, we argue that a retriever must evolve from a static path-follower into an adaptive decision-maker that can navigate the graph through a sequential reasoning process.
% Achieving this requires addressing three key challenges.
% \textit{(1) Incorporating high-level reasoning.}
% The process of following an edge is not merely a similarity matching task; 
% it often requires complex logical deduction to determine whether a specific node will lead to the final answer based on the current evidence.
% \textit{(2) Adapting reasoning to evolving context.}
% As evidence is accumulated, the information need shifts; 
% thus, the retriever must dynamically refine its hypotheses and subqueries, and recover from dead ends by reflecting on failures rather than adhering to a fixed, pre-defined plan.
% \textit{(3) Balancing accuracy and efficiency.}
% While advanced reasoning via large models can improve retrieval precision, it introduces significant computational overhead.
% An effective system must maintain a delicate balance by orchestrating a spectrum of strategies, ranging from lightweight matching to intensive reasoning, depending on the difficulty of the current hop.





% % Paragraph 5) Our approach
% {\color{red} 모두 문장으로 변경}
% To address these needs, we propose \textsc{\textbf{\OurFullName}} (\textsc{\Ours}), which introduces two key innovations:
% \textit{(1) LLM-enabled agentic traversal as a sequential decision process.}
% We cast graph traversal as a sequential decision process, where each hop decides \emph{what to ask} (subquery), \emph{how to retrieve} (tool/strategy), and \emph{where to move} (edge type/granularity) given the current evidence.
% To operationalize this formulation, we design an agentic framework in which each agent is explicitly equipped to invoke LLM reasoning when needed.
% Specifically, \textsc{\Ours} supports specialized tools (e.g., a planner and a multi-strategy traverser) coordinated by an orchestrator, enabling on-demand LLM reasoning and flexible switching across retrieval modes.
% \textit{(2) Adaptive accuracy--efficiency trade-offs with history-aware backtracking.}
% \textsc{\Ours} maintains a portfolio of strategy variants spanning an accuracy--efficiency spectrum, from low-cost vector matching for easy hops to high-cost LLM-intensive reasoning for ambiguous hops.
% Crucially, we incorporate a history-aware backtracking mechanism that turns failures into actionable feedback, making traversal both efficient and accurate.
% \emph{Efficiency:} the orchestrator begins with cheap strategies and escalates to stronger (but more expensive) reasoning only when a hop is difficult or an attempt fails.
% Upon reaching a dead end, the retriever backtracks while retaining failure traces to avoid redundant exploration and to revise the next hop (subquery/objective and strategy choice).
% \emph{Effectiveness:} these failure signals guide subsequent hops toward more plausible paths, improving multihop reliability without repeatedly paying for expensive reasoning.


% % Paragraph 6) Contributions
% In summary, we make three primary contributions:
% \squishlist
%     \item [1.] Formulation as a Sequential Decision Process with Economic Rationality: We redefine open-domain multimodal retrieval as an information-state MDP, operationalizing it through an LLM-enabled agentic workflow that treats retrieval strategy as a dynamic choice.
%     % We formulate open-domain multimodal retrieval as a \textbf{sequential decision process} and operationalize it through an \textbf{LLM-enabled agentic workflow}. 
%     %; this allows an orchestrator and specialized tools (e.g., planner, multi-strategy traverser) to adaptively decide what to ask, how to retrieve, and where to move at each hop.
%     \item [2.] Dynamic Cost-Aware Strategy Escalation: We introduce a novel mechanism that maintains a portfolio of strategies across an accuracy-efficiency spectrum. 
%     Our orchestrator avoids over-reasoning by starting with low-cost vector matching and only escalating to high-cost LLM reasoning when a hop is identified as ambiguous or follows a recorded failure.
%     % We introduce an \textbf{adaptive accuracy--efficiency mechanism} that combines cost-aware strategy escalation with \textbf{history-aware backtracking}. %, leveraging traces of failed traversals as constructive feedback to minimize redundant computation while significantly improving multihop reliability.
%     \item [3.] History-Aware Backtracking for Resilient Navigation: We convert failed traversals into constructive feedback. 
%     By piggybacking on failure traces, the agent re-anchors its search to prior contexts while revising its subqueries and escalating its strategy, ensuring both robustness and efficiency in complex graph environments.
%     % Extensive experiments show that \textsc{\Ours} achieves state-of-the-art retrieval performance on the \textsc{MultimodalQA}, \textsc{MMCoQA}, and \textsc{WebQA} benchmarks, significantly outperforming existing graph-based and single-index competitors. % by effectively navigating complex, interlinked multimodal documents.
% \squishend

















%%%%%%% Version 3


% Paragraph 4


% To overcome these limitations, we argue that a retriever must evolve from a static path-follower into an \textbf{adaptive decision-maker} that can navigate the graph through a sequential reasoning process.
% Achieving this requires addressing three key challenges.
% \textbf{(1) Reasoning-intensive retrieval.}
% In layered graphs, selecting what to retrieve (and which edge to follow) can require reasoning beyond direct similarity---e.g., inferring hop semantics from accumulated evidence, or interpreting visual/structural cues such as captions, table headers, and layout.
% \textbf{(2) Adaptive reasoning.}
% The optimal hop strategy changes as evidence accumulates; 
% thus, the retriever must dynamically refine its hypotheses and subqueries, and recover from dead ends by reflecting on failures rather than adhering to a fixed, pre-defined plan.
% \textbf{(3) Balancing accuracy and efficiency.}
% While LLM-based multimodal reasoning can be highly accurate for ambiguous hops, it is expensive; lightweight retrieval is efficient but often brittle.
% A practical OMDR system must orchestrate a spectrum of strategies and allocate computation adaptively per hop to achieve both reliability and low latency.


% To overcome these limitations, we argue that a retriever must evolve from a static path-follower into an \textbf{adaptive decision-maker} that can navigate the graph through a sequential reasoning process.
% Achieving this requires addressing three key challenges.
% \textbf{(1) Adapting reasoning to evolving context.}
% As evidence is accumulated, the information need shifts; thus, the retriever must dynamically adjust its strategy at each step rather than adhering to a fixed, pre-defined plan.
% \textbf{(2) Resilience to dead ends.}
% In open-domain graphs, encountering irrelevant paths or noisy subgraphs is inevitable.
% The challenge is not merely to backtrack to a previous state, but to \emph{recover intelligently} by analyzing why the previous path failed, turning the failure into a signal to prune future search paths.
% \textbf{(3) Orchestrating diverse strategies.}
% Different nodes and edges in a layered graph require different reasoning capabilities---some necessitate simple text matching, while others demand complex multimodal reasoning over visual and structural layouts.
% A robust system must possess a diverse toolkit and an orchestration mechanism to assign the most effective reasoning tool to each specific hop based on the evolving context.



% Paragraph 5

% To address these needs, we propose \textsc{\textbf{\OurFullName}} (\textsc{\Ours}), which introduces two key innovations:
% % Unlike prior methods that treat failure as a stopping condition, \textsc{\Ours} treats failure as a constructive signal through two key innovations.
% \textbf{(1) Agentic Workflow.}
% To execute this adaptive navigation, we introduce an agentic framework comprising a \textit{tool list} that covers diverse reasoning strategies (e.g., local/global hop, LLM reasoning, vector search granularity) and an \textit{orchestrator}.
% At each step, the orchestrator analyzes the current search context---including both successful history and previous failures---to assign the most effective tool and subquery for the next hop.
% This allows the system to seamlessly switch strategies, recovering from dead ends to find the correct evidence path.
% \textbf{(2) History-aware Backtracking.}
% Standard backtracking simply reverts the state, discarding the effort spent on the failed path.
% In contrast, our approach \textit{piggybacks} on the context of failed traversals.
% By leveraging insights from the "rejected" paths, the model optimizes the selection of alternative paths, ensuring it does not repeat the same mistake.


% To address these needs, we propose \textsc{\textbf{\OurFullName}} (\textsc{\Ours}), which introduces two key innovations:
% \textbf{(1) Modeling Traversal as a Sequential Decision Process}
% % Sequential decision process로 modeling한 다음에, 이 process의 각 부분을 agent로 설계한다.
% We introduce an agentic framework comprising a \textit{tool list} that covers diverse reasoning strategies and an \textit{orchestrator}.
% At each step, the orchestrator analyzes the current search context to assign the most effective tool and subquery for the next hop.
% This allows the system to seamlessly switch between different granularities and reasoning modes, ensuring that the most appropriate capability is applied to each specific part of the graph.
% \textbf{(2) Balancing Effectiveness and Efficiency via Adaptive Strategies and Backtracking.}
% To solve the retrieval problem efficiently, \textsc{\Ours} utilizes a range of strategy variants—from high-cost, LLM-intensive paths for complex reasoning to low-cost, vector-based paths for simple matching.
% Furthermore, we incorporate a \textbf{history-aware backtracking mechanism} that treats failures as constructive signals.
% Instead of simply discarding failed paths, our approach analyzes the context of "dead ends" to refine the selection of alternative routes, thereby preventing redundant computations and intelligently recovering from errors to find the correct evidence path.


% To address these needs, we propose \textsc{\textbf{\OurFullName}} (\textsc{\Ours}), which introduces two key innovations:
% \textbf{(1) Agentic Workflow.}
% We introduce an agentic framework comprising a \textit{tool list} that supports diverse hop reasoning operations (e.g., local/global hop, retrieval at multiple granularities, optional multimodal/LLM reasoning for disambiguation and verification) and an \textit{orchestrator}.
% At each step, the orchestrator analyzes the current search context---including the query, gathered evidence, and traversal history---to choose the next tool, subquery, and edge to execute.
% \textbf{(2) Efficient \& effective self-correcting traversal.}
% We explicitly construct multiple traversal variants that trade off cost and accuracy, spanning from efficient but coarse embedding-based strategies to more effective but slower LLM-heavy strategies, and invoke them selectively based on hop difficulty.
% Crucially, we incorporate \textit{history-aware backtracking}: when a path leads to an unproductive region, \textsc{\Ours} backtracks while retaining and exploiting the failure context as negative feedback, pruning future search paths to avoid repeating the same mistakes and improving multihop reliability.

% \textsc{\Ours} maintains a portfolio of strategy variants that span an accuracy--efficiency spectrum, ranging from low-cost vector-based matching for easy hops to high-cost LLM-intensive reasoning for ambiguous hops.
% Rather than committing to a single fixed routine, the orchestrator starts from efficient strategies and \emph{escalates} to stronger (but more expensive) reasoning only when the current hop is difficult or when earlier attempts fail.
% Crucially, we incorporate a \textbf{history-aware backtracking mechanism} that treats failures as constructive feedback.
% When the traversal hits a dead end, \textsc{\Ours} does not merely revert the state; it records the failed evidence, traversed edges, and attempted strategies, and uses them to (i) avoid redundant exploration of similar paths, (ii) revise the subquery or hop objective, and (iii) select a more appropriate strategy variant for the next attempt.
% This backtracking-driven adaptation improves \emph{efficiency} by minimizing unnecessary expensive calls and repeated search, while simultaneously improving \emph{effectiveness} by using failure signals to steer traversal toward more plausible evidence paths.

% \textsc{\Ours} maintains strategy variants spanning cheap vector matching to expensive LLM-intensive reasoning, and escalates cost only when needed.
% When a path fails, our history-aware backtracking records failure traces (evidence/edges/strategies) to avoid redundant exploration, revise the next hop, and choose better strategies, improving both efficiency and multihop reliability.



% To address these needs, we propose \textsc{\textbf{\OurFullName}} (\textsc{\Ours}), which introduces two key innovations:
% \textbf{(1) Modeling Traversal as a Sequential Decision Process with LLM-Enabled Agents.}
% We cast graph traversal as a sequential decision process, where each hop is an action that decides \emph{what to ask} (subquery), \emph{how to retrieve} (tool/strategy), and \emph{where to move} (edge type and granularity) based on the current evidence state.
% To operationalize this formulation, we design an agentic framework in which each agent is explicitly equipped to invoke LLM reasoning when needed.
% Concretely, \textsc{\Ours} provides a tool suite that assigns different roles---e.g., a \emph{planner} that proposes and refines hop-level objectives, and a \emph{multi-strategy traverser} that executes retrieval and edge-following across layers---together with an \emph{orchestrator} that supervises these agents.
% At every step, the orchestrator selects the most suitable agent/tool and generates the next subquery, enabling flexible switching across reasoning modes and retrieval granularities throughout the traversal.
% \textbf{(2) Balancing Effectiveness and Efficiency via Adaptive Strategies and History-Aware Backtracking.}
% \textsc{\Ours} maintains a portfolio of strategy variants that span an accuracy--efficiency spectrum, ranging from low-cost vector-based matching for easy hops to high-cost LLM-intensive reasoning for ambiguous hops.
% Rather than committing to a single fixed routine, the orchestrator starts from efficient strategies and \emph{escalates} to stronger (but more expensive) reasoning only when the current hop is difficult or when earlier attempts fail.
% Crucially, we incorporate a \textbf{history-aware backtracking mechanism} that treats failures as constructive feedback.
% When the traversal hits a dead end, \textsc{\Ours} does not merely revert the state; it records the failed evidence, traversed edges, and attempted strategies, and uses them to (i) avoid redundant exploration of similar paths, (ii) revise the subquery or hop objective, and (iii) select a more appropriate strategy variant for the next attempt.
% This backtracking-driven adaptation improves \emph{efficiency} by minimizing unnecessary expensive calls and repeated search, while simultaneously improving \emph{effectiveness} by using failure signals to steer traversal toward more plausible evidence paths.









%%%%%%%%%%%%%% Version 2


% Paragraph 2) Existing approaches & their problems
% Prior work on OMDR has advanced along two complementary directions.
% The \textit{first direction} reduces multimodal retrieval to a single-space nearest-neighbor search by converting every component into one modality.
% One common pipeline converts non-text components into text through OCR, captioning, or layout-to-text linearization, and then applies dense retrieval in the text embedding space~\cite{unimmqa, solar}.
% Another pipeline rasterizes pages or regions into images and retrieves them in a vision-language embedding space, treating retrieval as image search~\cite{3_visrag, 4_colpali, 5_m3docvqa}.
% These single-index methods are simple and scalable, but the conversion step can be lossy and can blur fine-grained signals into a large retrieval unit~\cite{6_densexretrieval, 7_mixofgran}.
% A flat index also treats each unit independently, so it does not directly exploit explicit connections such as within-page references or hyperlinks that support multihop retrieval.




% Given these intricate characteristics of OMDR, the underlying data for solving the task is inherently \textit{multi-granular} and \textit{highly interconnected}.
% Each document is composed of heterogeneous multimodal \textit{components}---including paragraphs, tables, and images---whose meanings are often shaped by local context such as captions, surrounding text, and layout.
% Moreover, components are linked not only within a document via explicit and implicit signals (e.g., cross-references, hyperlinks, and section-level adjacency), but also across documents through citations and hyperlinks, forming a large, noisy graph that users implicitly traverse while browsing.
% These properties make a graph abstraction particularly natural: by modeling components (and optionally documents/sections) as nodes and their contextual and navigational relations as edges, graph-based retrieval can explicitly leverage the connectivity signals required for multihop evidence discovery, and has thus emerged and evolved as a principled paradigm for OMDR~\cite{1_lilac, 9_ircot}.
% In particular, recent work shows that a \textit{layered component graph} better exploits these characteristics by representing document--component relations at multiple granularities (e.g., coarse document/section-level navigation followed by fine-grained component selection), enabling more effective and scalable multihop retrieval over multimodal corpora~\cite{1_lilac}.



% Gemini
% Paragraph 2) Transition to Graph-based Methods and their Limitations
% Given these intricate characteristics of OMDR, representing a document collection as a \textit{layered component graph} has emerged as a powerful paradigm to capture the multi-granularity and interconnectedness of multimodal evidence~\cite{1_lilac}.
% By modeling each paragraph, table, and image as distinct nodes and encoding their structural relationships---such as intra-document layouts and inter-document hyperlinks---as edges, this approach allows retrievers to navigate the complex information landscape more effectively than flat, single-index methods.

% However, while this graph-based structure provides a rich representation of the data, existing retrieval algorithms fail to fully exploit its potential due to their operational rigidity.
% First, they predominantly rely on a uniform, vector-based similarity metric for traversal, which overlooks the \textit{hop-specific semantics}---failing to distinguish when a hop requires visual pattern matching versus logical text-based deduction.
% Second, these methods typically operate under rigid, pre-defined plans, such as executing a fixed sequence of subqueries generated a priori.
% Consequently, they lack the flexibility for dynamic error correction; once the retriever follows a misleading link or encounters a noise-heavy subgraph, it cannot recover, leading to inevitable error propagation throughout the chain~\cite{9_ircot, 10_selfrag}.



% ChatGPT
% Paragraph 2) Prior Works
% The characteristics of OMDR data naturally motivate a graph-based view.
% Because a corpus consists of heterogeneous \textit{components} (paragraphs, tables, images), whose meanings depend on local context (captions, surrounding text, layout), and because components and documents are densely connected via explicit and implicit links (hyperlinks, cross-references, adjacency), representing the corpus as a component graph allows a retriever to directly exploit the same navigational signals that users follow while browsing.
% Accordingly, recent work has advanced structure-aware retrieval by constructing a \textit{layered component graph} and performing multihop subgraph retrieval on top of it~\cite{1_lilac}.
% In particular, LILaC models multimodal information at dual granularities (coarse vs.\ fine), enabling efficient candidate discovery at a coarse layer while preserving precise evidence selection through fine-grained components~\cite{1_lilac}.
% Overall, this line of work demonstrates that explicitly modeling document--component relations and connectivity is a powerful inductive bias for OMDR.

% However, despite the representational gains of layered graphs, existing graph-based retrievers still share rigidity in their \textit{retrieval algorithms}.
% First, traversal is typically driven by a single, embedding-based scoring rule, which is not sufficient to express \textit{hop-specific needs} that may go beyond similarity---e.g., deciding when to rely on direct multimodal matching versus reasoning over previously accumulated evidence.
% Second, traversal is often governed by a largely pre-specified procedure (e.g., a fixed sequence of subqueries produced upfront and executed with a fixed traversal routine), which limits dynamic error correction.
% As a result, once the retriever follows a spurious edge or gets trapped in a dead end, mistakes tend to propagate across hops and degrade final retrieval quality~\cite{9_ircot, 10_selfrag}.

% ChatGPT
% Paragraph 3) Challenges for reasoning-aware graph traversal
% To overcome this algorithmic rigidity, we view OMDR graph traversal as a \textit{sequential decision process} in which the retriever repeatedly (i) interprets the current state---the user query, retrieved evidence, and traversal history---and (ii) chooses the next retrieval action.
% This shift from static planning to decision-making introduces three key challenges.
% \textbf{(1) Adapting reasoning to evolving context.}
% Since the information need evolves as evidence is accumulated, the retriever must adjust its strategy at each step, rather than committing to a fixed plan decided at the beginning.
% \textbf{(2) Resilience to dead ends.}
% In open-domain graphs, failures are inevitable due to noise, missing links, or misleading associations.
% The goal is not only to backtrack, but to \emph{recover intelligently} by diagnosing why a path failed and using that failure as feedback for selecting a better alternative.
% \textbf{(3) Orchestrating diverse strategies.}
% Different hops require different capabilities---some benefit from coarse-grained navigation, others from fine-grained evidence selection, and still others from multimodal or logical reasoning.
% An effective system therefore needs a diverse toolkit and an orchestration mechanism that can assign the most suitable strategy to each hop based on the current context.










% % Paragraph 2) Prior Works
% The \textit{second direction} makes structure explicit by representing a document collection as a \textit{layered component graph} and performing graph traversal for retrieval~\cite{1_lilac}.
% It models each paragraph, table, and image as a node, utilizing edges to capture co-occurrence within documents and navigational links across them.
% While this structure-aware view supports multihop retrieval, existing approaches exhibit two critical limitations rooted in their rigidity.
% First, they rely on a uniform, vector-based similarity metric for traversal, which overlooks \textit{hop-specific semantics}---failing to distinguish when a hop requires visual matching versus logical deduction.
% Second, these methods typically operate under rigid, pre-defined plans (e.g., executing a fixed sequence of subqueries generated a priori).
% Consequently, they lack the flexibility to perform dynamic error correction; once the retriever follows a misleading link or encounters a dead end, it cannot recover, leading to error propagation throughout the chain~\cite{9_ircot, 10_selfrag}.



% % Paragraph 3) Challenges for reasoning-aware graph traversal
% These limitations suggest that a robust retriever must not only traverse a graph but also operate as an adaptive decision-maker.
% To achieve this, the retrieval process must evolve from static planning to a sequential decision process, addressing three key challenges.
% \textbf{(1) Adapting reasoning to evolving context.}
% Since the information need changes as evidence is accumulated, the retriever must dynamically adjust its strategy at each step, rather than following a static plan.
% \textbf{(2) Resilience to dead ends.}
% In open-domain graphs, failures are inevitable due to noise or missing links.
% The challenge is not merely to backtrack (i.e., revert to a previous state) but to \emph{recover intelligently} by analyzing why the previous path failed, turning the failure into a signal for the next move.
% \textbf{(3) Orchestrating diverse strategies.}
% Different hops require different reasoning capabilities---some need simple text matching, while others require complex multimodal reasoning.
% An effective system must possess a diverse toolkit and an orchestration mechanism to assign the most effective reasoning tool to each specific hop.














%%%%%%%%%%%%%%% Version 1






% % Paragraph 1) Motivation of Multimodal Document RAG
% Searching the web has become part of everyday life.
% This routine increasingly underpins multimodal retrieval-augmented generation (RAG), where a model answers a user query by grounding its output in retrieved evidence~\cite{4_colpali}.
% In practice, much of this evidence lives in webpages or PDFs, which are multimodal documents that show the following characteristics
% (i) A document is composed of multimodal \textit{components}, mixing paragraphs, tables, and images.
% (ii) The meaning of each component is often shaped by local context within the document such as captions, surrounding text, or layout.
% (iii) Components are connected through explicit signals, including hyperlinks and cross-references, and through implicit signals, including same-section adjacency.
% Furthermore, documents are further linked to other documents through hyperlinks and citations, forming a large graph that users implicitly navigate while browsing.
% In such a situation, the \textbf{open-domain multimodal document retrieval (OMDR)} task asks a system to return a small ranked set of relevant components from this large, noisy, and interlinked graph, often requiring multihop exploration~\cite{1_lilac, 9_ircot}.
% This capability is becoming a core primitive for multimodal RAG, as multimodal embedders and multimodal LLMs continue to improve in representing diverse modalities and reasoning over retrieved evidence~\cite{4_colpali}.



% % Paragraph 2) Existing approaches & their problems
% Prior work on OMDR has advanced along two complementary directions.
% The \textit{first direction} reduces multimodal retrieval to a single-space nearest-neighbor search by converting every component into one modality.
% One common pipeline converts non-text components into text through OCR, captioning, or layout-to-text linearization, and then applies dense retrieval in the text embedding space~\cite{unimmqa, solar}.
% Another pipeline rasterizes pages or regions into images and retrieves them in a vision-language embedding space, treating retrieval as image search~\cite{3_visrag, 4_colpali, 5_m3docvqa}.
% These single-index methods are simple and scalable, but the conversion step can be lossy and can blur fine-grained signals into a large retrieval unit~\cite{6_densexretrieval, 7_mixofgran}.
% A flat index also treats each unit independently, so it does not directly exploit explicit connections such as within-page references or hyperlinks that support multihop retrieval.


% The \textit{second direction} makes structure explicit by representing a document collection as a \textit{component graph} and performing graph traversal for retrieval~\cite{1_lilac}.
% It models each paragraph, table, and figure as a node, and it adds edges that capture co-occurrence within the same document and navigational links across documents.
% With each node embedded, it traverses through the graph by expanding through the edges.
% This structure-aware view supports multihop retrieval, but current traversal pipelines still resolve relationships in a shallow manner.
% Edge-based traversal is driven primarily by embedding similarity, which may not interpret what a specific link implies for the query.
% It also traverses using a single fixed plan and fixed scoring rules, driven by decomposed subqueries generated once before the traversal. 
% As a result, the retriever has limited ability to revise subgoals, retry with stronger reasoning, or switch traversal strategies when early steps are ambiguous or wrong~\cite{9_ircot, 10_selfrag}.




% % Paragraph 4) Challenges for reasoning-aware graph traversal
% These limitations point to a missing capability in current structure-aware retrieval.
% A retriever should not only traverse a document graph but also \emph{interpret} what an edge implies for the query, and \emph{revise} its plan as intermediate evidence changes which hop is most promising~\cite{9_ircot}.
% A natural direction is to use an LLM as a controller during traversal~\cite{10_selfrag, 11_react}.
% In this view, retrieval becomes a sequential decision process: at each step, the retriever determines the current subgoal, selects candidates to explore next, and chooses how much computation to spend before committing to a hop.

% A practical solution, however, faces three challenges.
% \textbf{(1) Exploration beyond local neighbors.}
% Query-relevant evidence is not always reachable via short-range neighbor expansion, since hyperlinks and cross-references can be sparse, noisy, or missing in open corpora.
% The retriever therefore must interleave local traversal with occasional \emph{global jumps} (e.g., corpus-wide search), while still exploiting the graph when it provides reliable navigational cues.
% \textbf{(2) Adaptive retrieval strategies.}
% Successful multihop retrieval requires handling diverse cases that cannot be covered by a single fixed procedure.
% First, the appropriate retrieval \emph{granularity} varies across queries and across hops, ranging from coarse components to fine-grained regions~\cite{6_densexretrieval, 7_mixofgran}.
% Second, the need for \emph{explicit reasoning} is also hop-dependent: some steps can be resolved by cheap similarity matching, while others require reasoning over an edge-induced dependency (e.g., interpreting what a link or reference implies under the current subgoal).
% \textbf{(3) Accuracy--efficiency trade-off.}
% Invoking an LLM at every hop for planning, critique, and relation interpretation can be expensive and unstable~\cite{10_selfrag}.
% The retriever should escalate reasoning only when necessary, reuse intermediate results to avoid redundant search, and carefully control the \emph{context budget} passed to the LLM so that reasoning does not rely on ever-growing prompt contexts.

% % Document-level도 지금 생략된 상태
% % Replanning은 지금 생략된 상태






% % Paragraph 5) Our approach
% To address these challenges, we propose \textsc{\Ours}, an accurate component retriever based on LLM-guided graph traversal over a layered component graph.
% We build on a representation that treats a multimodal corpus as connected \textit{components}, following prior work~\cite{1_lilac}.
% \textsc{\Ours} adds adaptive control on top of this structure through two ideas.
% \textbf{(1) Multi-strategy traversal.}
% \textsc{\Ours} maintains a strategy bank that varies traversal scope, retrieval granularity, and reasoning effort.
% Depending on the subgoal and the reliability of local links, the agent can expand only to a node's neighbors, re-seed with a corpus-wide vector search, or combine both when local edges are uninformative.
% Candidate scoring can likewise adapt: it may operate at the coarse component level, descend to fine-grained subcomponents, or use hybrid scoring that focuses on small query-relevant regions inside a component.
% When similarity signals are ambiguous, the agent selectively invokes an LLM to interpret relations induced by links or adjacency and rerank candidates accordingly.
% \textbf{(2) Progressive Traversal Orchestration.}
% To balance accuracy and efficiency, \textsc{\Ours} schedules strategies from cheap to expensive.
% After each retrieval attempt, the orchestrator inspects the retrieved components and judges whether the current subgoal is satisfied; if not, it either retries with a stronger strategy or replans the subgoal using partial evidence already obtained.
% If the subgoal is satisfied, the orchestrator summarizes the newly acquired evidence into an updated state, derives the next subgoal (or subquery), and selects an appropriate strategy to pursue it.
% This failure-aware loop corrects early mistakes without restarting traversal from scratch, while limiting LLM usage by escalating only on demand and keeping prompts compact via evidence selection and reuse.




% % Paragraph 6) Contributions
% In summary, we make four contributions.
% \squishlist
%     \item [1.] We motivate a reasoning-aware and failure-aware view of open-domain multimodal component retrieval, and cast it as sequential, agentic traversal over a linked multimodal component graph.
%     \item [2.] We introduce a multi-strategy traversal agent with a strategy bank that adapts traversal scope (local expansion vs.\ global jumps), retrieval granularity (coarse vs.\ fine), and optional LLM reasoning for relation interpretation and reranking.
%     \item [3.] We design a progressive traversal orchestrator that detects failures, retries, and replans subgoals using intermediate evidence, achieving improved retrieval accuracy and accuracy.
%     \item [4.] Extensive experiments show that our implementation of \textsc{\Ours} achieves the state-of-the art results on the benchmarks of \textsc{MultimodalQA}, \textsc{MMCoQA}, and \textsc{WebQA}, both on retrieval accuracy and end-to-end QA accuracy.
% \squishend





























% \begin{figure*}[t]
%   \centering
%   \includegraphics[width=\linewidth]{figures/Figure3.pdf}
%   \caption{
%     \textbf{Overview of Evaluation Benchmarks.}
%     We evaluate our method across simulation and real-world tasks.
%     \textbf{Top (Simulation):} We construct \textit{Personalized-SIMPLER} (left/middle) and \textit{Personalized-VLABench} (right) by repopulating existing environments with user-specific assets.
%     \textbf{Bottom (Real-world):} We conduct physical experiments on a SO-101 arm using 8 diverse object categories, covering both selection and pick-and-place tasks.
%     In all scenarios, the agent must identify a specific target instance among visually similar distractors, requiring precise instance-level grounding beyond generic category recognition.
%     }
%   \label{fig:benchmarks}
% \end{figure*}




% These limitations point to a missing capability in current structure-aware retrieval. 
% Traversal should be able to interpret what an edge or a relation implies for the query, and it should revise the plan when intermediate evidence changes the next hop~\cite{9_ircot}.
% A natural direction is to use an LLM as a controller during traversal~\cite{10_selfrag, 11_react}.
% The retriever repeatedly decides the current subgoal, the next candidates to explore, and the amount of computation to spend before committing to a hop.
% A practical solution, however, faces three challenges. 
% \textbf{(1) Exploration beyond local neighbors.} 
% Relevant evidence is not always reachable by short neighbor expansion, because hyperlinks can be sparse, noisy, or missing in open corpora.
% The retriever therefore needs to mix local traversal with occasional global jumps, while still exploiting the graph when it is informative.
% \textbf{(2) Adaptive retrieval strategies.}
% 올바른 retrieval을 위해서는 여러 경우의 수를 고려해야 하며, 이런 것들은 다양한 strategy로 표현되어야 한다.
% 첫째는 vector search의 granularity로, the right retrieval unit changes across queries and across hop~\cite{6_densexretrieval, 7_mixofgran}.
% 둘째는 traversal에 LLM의 reasoning이 필요한지의 여부로, Some steps succeed with coarse components and cheap similarity, while others require fine-grained matching or explicit reasoning about a dependency expressed by an edge.
% \textbf{(3) Accuracy--efficiency trade-off.} 
% Agentic control can be expensive and unstable if an LLM is invoked at every hop for planning and critique.
% The retriever should escalate reasoning only when necessary, and it must reuse intermediate results to avoid redundant search.
% 또한, 각 reasoning을 할 때 LLM에게 주어지는 context의 길이를 크게 하지 않도록 조절하는 것 또한 중요하다. 




% To address these challenges, we propose \textsc{\Ours}, an accurate component retriever based on llm-reasoning-powered graph traversal over a layered component graph.
% We build on a graph representation that treats a multimodal corpus as connected \textit{components}, following prior work~\cite{1_lilac}. 
% % Each page contributes coarse nodes for whole paragraphs, tables, and figures, and it also contributes fine nodes for their constituent units such as sentences, table rows, or detected visual objects.
% % Edges preserve navigational signals, including within-page adjacency and cross-document hyperlinks, and they also connect each coarse node to its fine-grained subcomponents.
% \textsc{\Ours} adds adaptive control on top of this structure through two ideas. 
% \textbf{(1) Multi-strategy traversal.} 
% \textsc{\Ours} maintains a strategy bank that varies traversal scope, similarity granularity, and reasoning effort. 
% A strategy may expand only graph neighbors, perform a corpus-wide vector search, or combine both when local edges are uninformative.
% A strategy may score candidates at the coarse level, at the fine level, or with hybrid scoring that focuses on small query-relevant regions inside a component.
% A strategy may also invoke an LLM to interpret relations and rerank candidates when similarity alone is ambiguous. 
% \textbf{(2) Progressive Traversal Orchestration.} 
% \textsc{\Ours} schedules strategies from cheap to expensive to balance accuracy and efficiency.
% After each retrieval attempt, the orchestrator inspects the retrieved components and judges whether the current subgoal is satisfied.
% If it is not, the orchestrator either retries with a stronger strategy or replans the subgoal using the partial evidence already retrieved.
% This failure-aware loop corrects early mistakes without restarting traversal from scratch, and it limits LLM usage by escalating only on demand.
% Finally, we aggregate candidates across attempts and output a compact ranked evidence set for downstream multimodal RAG.



% In summary, we make four contributions. 
% \squishlist
%     \item [1.] We motivate a failure-aware and reasoning-aware view of open-domain multimodal component retrieval, and we cast it as agentic traversal over a multimodal component graph with explicit links.
%     \item [2.] We present \textsc{\Ours}, which augments layered component graphs with adaptive planning while preserving multimodal component boundaries and navigational structure.
%     \item [3.] We introduce a traversal agent with a strategy bank that controls traversal scope, retrieval granularity, and optional LLM reasoning for relation interpretation and reranking.
%     \item [4.] We design a progressive traversal orchestrator that retries, switches strategies, and replans subtasks based on intermediate evidence, and we demonstrate improved retrieval accuracy and accuracy--cost trade-offs on open-domain multimodal multihop benchmarks.
% \squishend


\vspace{-1mm}
\section{Related Work}
\label{sec:relatedwork}
\vspace{-1mm}








\vspace{-1mm}
\subsection{Multimodal Retrieval Methods}
\label{sec:related_multimodal_retrieval}
\vspace{-1mm}

Early multimodal retrievers were largely \emph{TextRAG}-style: they transformed multimodal components into textual surrogates via OCR, captioning, or serialization, enabling mature text retrieval pipelines but inevitably discarding vision-specific cues~\cite{16_mmmulhop, 17_unifiedprese, 18_unifying, 19_helios}.
More recently, \emph{VisRAG}-style pipelines unify modalities by rasterizing documents into page- or region-level screenshots and embedding all content in a single \emph{visual} space~\cite{3_visrag, 4_colpali, 5_m3docvqa}.
However, they suffer from two key limitations: (i) \textit{fixed granularity}, where large screenshots dilute query-relevant signals with irrelevant context, and (ii) \textit{limited multihop reasoning}, treating pages independently without exploiting structural links~\cite{6_densexretrieval, 7_mixofgran}.

Closest to our work, \textsc{LILaC} represents a multimodal document corpus as a layered graph structure and performs structure-aware retrieval through vector-embedding-based graph traversal~\cite{1_lilac}.
It builds a component graph linking coarse nodes (paragraphs, tables, images) and fine-grained subcomponents, using edges to represent both hierarchical containment and navigational relations.
At query time, \textsc{LILaC} performs an edge-wise beam search driven by late-interaction scores between subqueries and nodes.
While effective, this pipeline operates under a rigid, pre-defined plan: it relies on a uniform similarity metric for traversal, overlooking hop-specific semantics, and executes a linear expansion strategy.




\vspace{-2mm}
\subsection{Graph Retrieval Methods}
\vspace{-1mm}

Graph-based retrieval has been extensively studied in knowledge graph QA, where systems traverse graphs curated with \emph{typed} and \emph{semantically meaningful} relations~\cite{13_reasonpath, 24_hierargraph, tog, rog, gog}.
However, multimodal \emph{document} graphs differ fundamentally: edges are primarily \emph{navigational} (e.g., hyperlinks) rather than semantic predicates. 
Consequently, a retriever cannot treat traversal as simple path-finding over valid facts; it must perform online interpretation to resolve what a navigational link implies for the current query context~\cite{1_lilac}.
Thus, KG methods struggle with the adaptive capabilities to navigate large-scale, non-edge-labeled graphs where semantic resolution is needed~\cite{16_mmmulhop, 27_mmapg}.


\vspace{-2mm}
\subsection{Agentic Retrieval Methods}
\vspace{-1mm}

A growing line of work treats retrieval as an iterative decision process interleaved with reasoning, rather than a single-shot nearest-neighbor lookup.
\textsc{IRCoT} shows that multi-step questions benefit from repeatedly generating intermediate sub-questions and retrieving evidence for each step~\cite{9_ircot}.
\textsc{ReAct} formalizes a general reasoning-and-acting loop, motivating RAG controllers that plan retrieval actions based on intermediate observations~\cite{11_react}.

Despite this, most operate on \emph{flat} indices, lacking the structural awareness to navigate inter-document links. 
Furthermore, their error correction is typically limited to query rewriting rather than history-aware backtracking. 
While systems like \textsc{Doc-React} and \textsc{MARA} apply agents to multimodal documents, they target single-document or small-scale contexts~\cite{33_docreact, 35_mara}. 
They do not address the open-domain challenge of traversing vast, interconnected graphs where utilizing failure feedback is critical for routing optimization.





























%%%%%%%%%%%%% Version 2


% \textsc{VisRAG} demonstrates end-to-end vision-based retrieval–augmented generation, and \textsc{ColPali} adapts late-interaction retrieval to document images by producing multi-vector representations for efficient matching~\cite{22_colbert, 23_colbertv2}.
% Despite their strengths, VisRAG pipelines inherit two recurring limitations:
% (i) \textbf{fixed granularity}, where full-page or large-region screenshots mix query-relevant signals with substantial irrelevant context, and
% (ii) \textbf{limited multihop reasoning}, where pages are retrieved independently without exploiting within-document structure or cross-document links~\cite{6_densexretrieval, 7_mixofgran}.






% In this setting, edges directly constrain compositional reasoning, allowing agents to construct valid reasoning chains via beam search or path planning.
% For instance, \textsc{Think-on-Graph} and \textsc{Reason-on-Graph} treat LLMs as agents that explore entities to find supporting subgraphs, relying on the explicit semantics of KG edges to guide the search~\cite{tog, rog}.
% However, multimodal \emph{document} graphs differ fundamentally from labeled KGs.
% Edges in document collections are primarily \emph{navigational} (e.g., hyperlinks, layout adjacency) rather than semantic predicates.

% Current graph-based methods, designed for explicit KGs or small closed-set document collections, lack the adaptive reasoning capabilities required to navigate large-scale, noisy open-domain document graphs where "dead ends" are frequent and failure feedback is essential~\cite{16_mmmulhop, 27_mmapg}.





% Further advancements, such as Self-RAG and Corrective-RAG, introduce critique signals to revise generations or re-retrieve when evidence is insufficient~\cite{10_selfrag, 26_correcrag, 28_unirag, 29_prism}.

% Despite this progress, most agentic retrievers operate on a \emph{flat} index of independent chunks, treating retrieval as repeated similarity search over a fixed granularity~\cite{9_ircot}.
% They typically lack \textbf{structural awareness}—the ability to navigate explicit links between documents—and their error correction is limited to query rewriting rather than \textbf{history-aware backtracking} within a graph.
% While recent systems like \textsc{Doc-React}, \textsc{DocAgent}, and \textsc{MARA} explore tool use and iterative selection for multimodal documents, they are primarily designed for single-document understanding or small candidate sets~\cite{33_docreact, 34_docagent, 35_mara}.
% They do not address the \textbf{open-domain} challenge where an agent must traverse a vast, interconnected graph, and leverage failed attempts as feedback to optimize subsequent routing decisions.


% Despite this progress, most agentic retrievers still inherit two common assumptions from standard RAG pipelines:
% (i) evidence is drawn from a \emph{flat} pool of largely independent units, and
% (ii) retrieval actions are implemented as repeated similarity search and reranking over a \emph{fixed} granularity.
% As a result, directly applying these controllers to multimodal documents often falls back to page-level or OCR-linearized representations, reintroducing the granularity and structure-ignorance issues discussed earlier.
% In the multimodal document domain, agentic systems have begun to explore iterative page/region selection and tool use for multi-page and heterogeneous document QA~\cite{33_docreact, 34_docagent, 35_mara}.
% However, these systems primarily operate \emph{within} a given document or a small candidate set: 
% for example, \textsc{Doc-React} incrementally selects subsets of page images to balance information gain and generation uncertainty, but it does not model cross-document hyperlink navigation and it inherits the limitations of page-level visual retrieval for fine-grained component discovery~\cite{33_docreact}.
% \textsc{DocAgent} focuses on long-context document understanding by extracting a hierarchical outline and interactively retrieving content from the provided document, rather than performing open-domain multihop retrieval over a noisy linked corpus~\cite{34_docagent}.
% Similarly, \textsc{MARA} studies adaptive multimodal retrieval for document QA, but it does not target open-domain component retrieval over hyperlink-connected document graphs~\cite{35_mara}.










%%%%%%%%%%%%%%% Version 1



% \subsection{Multimodal Retrieval Methods}
% \label{sec:related_multimodal_retrieval}

% Early multimodal retrievers were largely \emph{text-first}: they transformed multimodal components (paragraphs, tables, and figures) into textual surrogates via OCR, captioning, or serialization, enabling mature text retrieval pipelines but inevitably discarding vision-specific cues~\cite{16_mmmulhop, 17_unifiedprese, 18_unifying, 19_helios}.
% A complementary line keeps separate encoders for text and images and fuses their retrieval scores at ranking time, which preserves modality signals but often relies on heuristic aggregation and becomes brittle when evidence must be composed across modalities~\cite{20_surveymrag, 21_beyondtext}.


% More recently, \emph{VisRAG}-style pipelines unify modalities by rasterizing documents into page- or region-level screenshots and embedding all content in a single \emph{visual} space~\cite{3_visrag, 4_colpali, 5_m3docvqa}.
% \textsc{VisRAG} demonstrates end-to-end vision-based retrieval–augmented generation, and \textsc{ColPali} adapts late-interaction retrieval to document images by producing multi-vector representations for efficient matching~\cite{22_colbert, 23_colbertv2}.
% Despite their strengths, VisRAG pipelines inherit two recurring limitations: 
% (i) \textbf{fixed granularity}, where full-page or large-region screenshots mix query-relevant signals with substantial irrelevant context, and 
% (ii) \textbf{limited multihop reasoning}, where pages are retrieved independently without exploiting within-document structure or cross-document links~\cite{6_densexretrieval, 7_mixofgran}.


% Closest to our work, \textsc{LILaC} represents a multimodal document corpus as a layered graph structure and performs structure-aware retrieval through vector-embedding-based graph traversal~\cite{1_lilac}.
% In the offline stage, it builds a two-layer component graph where coarse nodes correspond to paragraphs, tables, and images, and fine nodes further decompose them into modality-specific subcomponents such as sentences, table-row segments, and detected visual objects; 
% edges encode (i) hierarchical containment from coarse components to their subcomponents and (ii) navigational relations that preserve intra- and cross-document connectivity without pre-assigning a specific semantic meaning to each link.
% At query time, \textsc{LILaC} first uses an LLM to decompose the query into modality-targeting subqueries and embeds them accordingly, and then performs an edge-wise beam-search traversal that repeatedly expands one-hop neighbors while scoring edges via late interaction between the subqueries and the incident fine-grained subcomponents.
% While effective, this pipeline largely follows a \emph{single fixed traversal policy} once the initial decomposition is produced: traversal proceeds with local neighbor expansion and a similarity-driven edge scoring rule, and it does not explicitly support revising subgoals, switching traversal scope (e.g., re-seeding with a new global search), or escalating relation interpretation only when intermediate evidence indicates ambiguity or failure.





% \subsection{Graph Retrieval Methods}

% Graph-based retrieval has been extensively studied in KGQA and graph reasoning, where the input graph is typically curated with \emph{typed} and \emph{semantically meaningful} relations.
% In this setting, systems traverse the graph to recover query-relevant reasoning paths or supporting subgraphs, and edge semantics directly constrain both expansion and compositional reasoning~\cite{13_reasonpath, 24_hierargraph}.
% Representative examples of graph-conditioned retrieval-and-generation include:
% \textsc{Think-on-Graph}, which treats an LLM as an agent that explores entities/relations via beam search on a KG to construct promising reasoning chains~\cite{tog};
% \textsc{Reason-on-Graph}, which first plans relation paths and then retrieves valid KG reasoning paths to guide faithful step-wise reasoning~\cite{rog};
% and \textsc{Generate-on-Graph}, which further considers incompleteKGs and allows the LLM to generate missing triples while searching, integrating internal and external knowledge~\cite{gog}.

% However, multimodal \emph{document} graphs differ fundamentally from labeled knowledge graphs.
% Edges in document collections are often \emph{navigational} (hyperlinks, cross-references, and layout adjacency) rather than explicit semantic predicates, and their query-dependent meaning must be resolved online from local context such as anchor text, surrounding paragraphs, or section structure~\cite{1_lilac}.
% Recent multimodal multi-hop QA systems also have approaching that build structured knowledge or planning graphs, but they typically assume a \emph{distractor} setting where a small candidate set of documents is given, and is not easily extendable to open-domain settings~\cite{16_mmmulhop, 27_mmapg}.



% \subsection{Agentic Retrieval Methods}

% A growing line of work treats retrieval as an iterative decision process interleaved with reasoning, rather than a single-shot nearest-neighbor lookup.
% \textsc{IRCoT} shows that multi-step questions benefit from repeatedly generating intermediate sub-questions and retrieving evidence for each step~\cite{9_ircot}.
% \textsc{ReAct} formalizes a general reasoning-and-acting loop, motivating RAG controllers that plan retrieval actions based on intermediate observations~\cite{11_react}.
% Self-reflective and corrective RAG variants further introduce critique signals to decide when to retrieve and how to revise generations when evidence is missing or inconsistent~\cite{10_selfrag, 26_correcrag}.
% Recent agentic retrieval frameworks extend these ideas to scalable search and graph-centric settings, including decomposition-and-rewriting controllers and parallel reasoning chains over retrieved evidence~\cite{28_unirag, 29_prism, 30_mirage, 31_agentg}.

% Despite this progress, most agentic retrievers still inherit two common assumptions from standard RAG pipelines:
% (i) evidence is drawn from a \emph{flat} pool of largely independent units, and
% (ii) retrieval actions are implemented as repeated similarity search and reranking over a \emph{fixed} granularity.
% As a result, directly applying these controllers to multimodal documents often falls back to page-level or OCR-linearized representations, reintroducing the granularity and structure-ignorance issues discussed earlier.
% In the multimodal document domain, agentic systems have begun to explore iterative page/region selection and tool use for multi-page and heterogeneous document QA~\cite{33_docreact, 34_docagent, 35_mara}.
% However, these systems primarily operate \emph{within} a given document or a small candidate set: 
% for example, \textsc{Doc-React} incrementally selects subsets of page images to balance information gain and generation uncertainty, but it does not model cross-document hyperlink navigation and it inherits the limitations of page-level visual retrieval for fine-grained component discovery~\cite{33_docreact}.
% \textsc{DocAgent} focuses on long-context document understanding by extracting a hierarchical outline and interactively retrieving content from the provided document, rather than performing open-domain multihop retrieval over a noisy linked corpus~\cite{34_docagent}.
% Similarly, \textsc{MARA} studies adaptive multimodal retrieval for document QA, but it does not target open-domain component retrieval over hyperlink-connected document graphs~\cite{35_mara}.






\vspace{-1mm}
\section{Multimodal Document Retrieval}
\vspace{-1mm}
\label{sec:probdef}

We study \emph{open-domain multimodal document retrieval} to find relevant components from a large multimodal corpus on a natural language query.
In this study, we follow the setup of graph-based retrieval approaches~\cite{1_lilac}, which have shown promising performances over existing naive approaches.

\begin{figure}[t]
  \centering
  \includegraphics[width=\linewidth]{figures/preliminaries.pdf}
  \caption{
    Visualization of an example corpus $\mathcal{D}$ and its corresponding layered component graph $\mathcal{G}$.
    }
  \label{fig:preliminaries_example}
  \vspace{-6mm}
\end{figure}

\vspace{-1mm}
\subsection{Problem Setting}
\vspace{-1mm}
\noindent\textbf{Corpus and components.}
For the source of retrieval, a corpus $\mathcal{D}=$$\{D_1,$$\ldots,$$D_{k_{doc}}\}$ is a set of documents.
Each document $D_j$ includes an ordered list of multi-modal components $D_j=[C_{j,1},$ $\ldots,$ $C_{j,k_j}]$, where the global component pool is defined as $\mathcal{C}$ $=$ $\bigcup_{j=1}^{k_{doc}}\{$ $C_{j,1},$ $\ldots,$ $C_{j,k_j}\}$.
The modality of each component $C\in\mathcal{C}$ can be a paragraph $P$, a table $T$, or an image $I$ as shown in Figure~\ref{fig:preliminaries_example}.

\vspace{-1mm}
\noindent\textbf{Navigational links.}
We assume a link signal $\mathcal{L}$ capturing navigational associations (e.g., hyperlinks and cross-document pointers), modeled as $\mathcal{L}:\mathcal{C}\rightarrow\mathcal{D}$.

\vspace{-1mm}
\noindent\textbf{Multimodal retrieval task.}
Given $Q$, $\mathcal{D}$, and $\mathcal{L}$, the retriever ranks components in $\mathcal{C}$ and returns $\mathcal{C}_{R}=[C_{R_1},\ldots,C_{R_n}]$.
Let $\mathcal{C}_{gt}(Q)=\{C_{gt_1},\ldots,C_{gt_r}\}$ be the ground-truth relevant set; the goal is to rank its elements in $\mathcal{R}$, ideally near the top.


% Here, you should add some notes how previous study formulate and solve this retrieval task. 
% For example, there computes the similarity between the embedding of each component and query, but they cannot consider the structure, semantic, and context of each document. 
% Also, they are very inefficient. 
% Here, we can also add a brief note of that we formulate and solve this program as sequential action process as shown in Section 4.
% {\color{blue} 
% \noindent\textbf{Existing Approaches.}
% 기존의 방법론들은 webpage의 screenshot 별로 embedding vector 하나씩을 만들고, 한 번의 knn을 진행하는 형태로 retrieval을 진행한다~\cite{4_colpali, 3_visrag}.
% 다만 이 경우 서로 다른 component 간의 관계를 고려하기 어렵기 때문에, multihop retrieval에서 struggle한다.
% 더 최근의 연구는 navigational link를 이용하여 component를 연결해 놓고, 두 개의 pair 단위로 score를 계산하여 multihop을 해결하고자 했다.
% 이 과정에서 component를 subcomponent로 쪼개고, query 또한 subquery로 쪼갠 후에 이들 간의 fine-grained vector similarity를 비교한다.
% 하지만 이는 vector similarity에 기반하여 query-relevancy를 재기 때문에 accuracy에 한계가 있으며, 직접적으로 연결되어 있지 않은 두 component의 연관성을 파악하기에 어려움이 있다.
% }

% \jhyun{
% \noindent\textbf{Existing Approaches.}
% Early lines of work typically vision-vector-based.
% They construct a single embedding vector per webpage screenshot and perform retrieval via a one-shot $k$NN search~\cite{4_colpali, 3_visrag}.
% While efficient, this design largely ignores the relationships between components across pages and across documents, which makes multihop retrieval brittle.
% A more recent work explicitly expresses such relationships by connecting components using navigational links as edges, and attempts to solve multihop retrieval by scoring edges~\cite{1_lilac}.
% In doing so, it further splits each component into subcomponents and decomposes the query into subqueries, then compares vector similarities for the fine-grained units.
% However, it still may struggle on accurately resolving the component relationship as it relies on vector-based scores.
% Also, the static query decomposition may lead to an error propagation when done erroneously. 
% }











\vspace{-1mm}
\subsection{Layered Component Graph}
\label{sec:lcg}
\vspace{-1mm}

We incorporate a layered component graph~\cite{1_lilac} to support an effective \emph{coarse-to-fine} retrieval across documents and their components.
To efficiently retrieve relevant documents and evidence for a given query, open-domain retrieval requires to plan both which documents to visit and which components to read.
Thus, we adopt the three-layered component graph to effectively represent documents, components, subcomponents, and their complex relations.
Figure~\ref{fig:preliminaries_example} shows an example graph $\mathcal{G}$.

\vspace{-1mm}
\noindent\textbf{Nodes and Layers.}
Let $\mathcal{G}=(\mathcal{V},\mathcal{E})$ denote the graph. 
Using the definitions from Section~\ref{sec:probdef}, we construct a three-layered hierarchy $\mathcal{V} = V_0 \cup V_1 \cup V_2$:

\vspace{-2mm}
\squishlist
    \item \textit{Layer 0 (Documents) $V_0$}
    \item \textit{Layer 1 (Components) $V_1$}: paragraphs/tables/images
    \item \textit{Layer 2 (Subcomponents) $V_2$}: sentences, table rows, or visual objects
\squishend
\vspace{-2mm}
% \dylee{
For nodes in $V_1\cup V_2$, we save the raw multimodal content.
For nodes in $V_0$, we save a short textual summary of each document to avoid long inputs while preserving high-level semantics for global routing.% [DOYUP] do we need this?}

Different from LILAC~\cite{1_lilac}, which uses two layers only with components and subcomponents, we add an explicit \emph{Document Layer} to store a concise \emph{textual summary} per document node.
Adding document layer enables early pruning before descending to fine-grained evidence and can significantly improve the retrieval performance.

\vspace{-1mm}
\noindent\textbf{Hierarchical Edges.}
These edges represent the ``contains'' relationship, allowing the agent to drill down from coarse to fine granularity.
\vspace{-2mm}
\squishlist
    \item Edges $(D_j, C_{j,i})$ for all components $C_{j,i} \in D_j$.
    \item Edges linking a component to its extracted subcomponents. %(e.g., $v(T) \to v(T_{row})$).
\squishend
\vspace{-1mm}

\vspace{-1mm}
\noindent\textbf{Navigational Edges.}
These edges capture explicit navigational paths across the corpus, allowing the retriever to transition between different document contexts based on the link signal $\mathcal{L}$.
Using the link signal $\mathcal{L}$, we generate an edge $(C,D_k)$ if $\mathcal{L}(C)=D_k$.
% \vspace{-2mm}
% \squishlist
% \textit{Component$\to$Document ($V_1\!\to\!V_0$)}: 
%     \item \textit{Document$\to$Document ($V_0\!\to\!V_0$)}: if $\exists\,C\in D_j$ with $\mathcal{L}(C)=D_k$, add $(v(D_j),v(D_k))$.
% \squishend
% \vspace{-2mm}

Figure~\ref{fig:preliminaries_example} illustrates the two edge types: dotted lines express hierarchical edges, and blue lines express navigational edges.



















%%%%%%%%%%%% Version 3


% We consider the task of \emph{open-domain multimodal document retrieval}: for a natural-language query, the system returns a ranked shortlist of components drawn from a large multimodal document corpus.
% We adopt the overall setup of \textsc{LILaC}~\cite{1_lilac}.


% \noindent\textbf{Corpus, documents and components.}
% A corpus $\mathcal{D}$ $=$ $\{D_1,$ $\ldots,$ $D_{k_{doc}}\}$ consists of multimodal documents.
% Each document $D_j$ is parsed into an ordered collection of components $D_j$ $=$ $[C_{j,1},$ $\ldots,$ $C_{j,k_j}]$, and we denote the global component pool as $\mathcal{C}$ $=$ $\bigcup_{j=1}^{k_{doc}} \{C_{j,1},$ $\ldots,$ $C_{j,k_j}\}$.
% Every component $C\in\mathcal{C}$ has a modality type in $\{P,T,I\}$:
% \squishlist
%     \item \textit{Paragraph} $P$: a token sequence representing an unstructured text span.
%     \item \textit{Table} $T$: a matrix of cells (or records) with rows denoted by $T_i$.
%     \item \textit{Image} $I$: a pixel tensor $I\in\mathbb{R}^{w\times h\times a}$ with width $w$, height $h$, and channels $a$.
% \squishend
% An example corpus is given in Figure~\ref{fig:preliminaries_example}.


% \noindent\textbf{Navigational links.}
% Beyond containment within a document, we assume access to a link signal $\mathcal{L}$ that captures navigational associations commonly observed in webpages.
% Concretely, $\mathcal{L}$ connects components to documents through references such as hyperlinks and cross-document pointers, i.e.,
% $\mathcal{L}:\mathcal{C}\rightarrow \mathcal{D}$. % (and can be viewed as inducing a document/component-level navigation graph). 




% \noindent\textbf{Retrieval task.}
% Given a query $Q$, the corpus $\mathcal{D}$, and the link signal $\mathcal{L}$, the retriever produces a ranking over components. %, e.g.,
% via a scoring function $s(Q,C)$ that induces an order on $\mathcal{C}$.
% The output is the top-$n_{ret}$ list $\mathcal{R} = [\hat{C}_1,\ldots,\hat{C}_{n_{ret}}]$. % sorted by decreasing query-relevancy.
% Let $\mathcal{C}_{gt}(Q)=\{C_{gt_1},\ldots,C_{gt_r}\}$ be the set of ground-truth relevant components for $Q$.
% The goal is to place the elements of $\mathcal{C}_{gt}(Q)$ into $\mathcal{R}$, preferably at high ranks.


% Open-domain multimodal retrieval necessitates a balance between global planning (selecting documents) and local evidence gathering (reading components).
% While \textsc{LILaC}~\cite{1_lilac} facilitates multihop reasoning via a layered graph of components and subcomponents, it treats document-level context merely as metadata rather than explicit structural nodes.
% Consequently, traversal remains confined to the component layer, where the high branching factor in large corpora renders exploration and backtracking computationally expensive.
% To bridge this gap, we extend the \textsc{LILaC} formulation with an explicit \emph{Document Layer}.
% We assign each document node a concise textual \emph{summary} instead of a raw concatenation of contents, thereby preventing semantic dilution and avoiding excessive context length.
% This structure enables a \emph{coarse-to-fine} search strategy: the retriever can navigate global hops using high-level semantics and prune irrelevant branches early, descending into fine-grained component evidence only when necessary.
% An example layered component graph is shown in Figure~\ref{fig:preliminaries_example} as $\mathcal{G}$.



% \squishlist
%     \item \textit{Layer 0 (Documents) $V_0$}: The set $\{v(D_j) \mid D_j \in \mathcal{D}\}$, where each node represents a full document.
%     \item \textit{Layer 1 (Components) $V_1$}: The set $\{v(C) \mid C \in \mathcal{C}\}$, corresponding to the coarse-grained paragraphs, tables, and images.
%     \item \textit{Layer 2 (Subcomponents) $V_2$}: Fine-grained nodes representing sentences, table rows, or visual objects.
% \squishend
% Each node $v$ stores a content field $x(v)$. 
% For component and subcomponent nodes ($V_1 \cup V_2$), $x(v)$ remains the raw multimodal content. 
% However, for document nodes ($V_0$), raw concatenation of all components is often too lengthy to process effectively and efficiently. 
% Thus, we set $x(v(D_j))$ to a concise \emph{textual summary} of $D_j$. 
% This summary provides a high-level semantic overview, allowing the agent to perform global hops and decisions without processing the entire document content.



% We construct $E_{nav}$ by mapping these links to both the component and document layers:
% \squishlist
%     \item \textit{Component$\to$Document ($V_1 \to V_0$)}: 
%     If a component $C \in V_1$ references a target document $D_{k}$ (i.e., $\mathcal{L}(C) = D_{k}$), we add an edge $(v(C), v(D_{k}))$. 
%     % This allows the agent to follow a specific hyperlink but forces it to land on the target's summary node first, enabling a high-level relevance check before descending into details.
    
%     \item \textit{Document$\to$Document ($V_0 \to V_0$)}: 
%     We also project navigational links to the document layer to support coarse-level planning.
%     For any two documents $D_{j}$ and $D_{k}$, if there exists any component $C$ inside $D_{j}$ that links to $D_{k}$ (i.e., $C \in D_{j} \land \mathcal{L}(C) = D_{k}$), we add a directed edge $(v(D_{j}), v(D_{k}))$.
%     % This creates a ``macro'' navigation graph, allowing the agent to jump directly between related documents without traversing intermediate component nodes.
% \squishend
% These edges are expressed in Figure~\ref{fig:preliminaries_example} as dotted lines.






%%%%%%%%%%%%%%%%%%%%%% Version 2





% Open-domain multimodal retrieval in large corpora often requires multihop navigation: the system must decide \emph{which documents to visit} (global planning) and then \emph{which components to read} (local evidence gathering).
% \textsc{LILaC}~\cite{1_lilac} addresses multihop reasoning by building a layered component graph over components and their fine-grained subcomponents, and emphasizes component-level hops via intra-/inter-document component edges.
% However, its traversal still effectively \emph{starts from} (and is dominated by) the component layer: when the corpus is vast, the branching factor at the component level can make exploration and backtracking expensive.
% Moreover, while document-level signals (e.g., document title) can be attached as metadata to components, \textsc{LILaC} does not explicitly model documents as graph nodes nor introduce document-level navigation edges.
% To enable a coarse-to-fine search strategy, we extend the \textsc{LILaC} formulation with an explicit \emph{document layer}.
% Each document node stores a short textual \emph{summary} rather than concatenating all component contents, since a document may contain many long components and a raw aggregation is often prohibitively lengthy and semantically diluted.
% This document-level abstraction allows the retriever to make global hops guided by high-level semantics, prune irrelevant branches early, and then descend to component/subcomponent evidence only when needed.

% Real-world multimodal corpora are not a flat set of independent chunks. 
% Within a document, the meaning of a component (paragraph, table, or image) is shaped by local context of surrounding components. 
% Across documents, components are connected by explicit navigational signals, including hyperlinks. 
% % A flat index discards these structural cues, making multihop evidence discovery brittle. 
% To address this, we represent the corpus as a \emph{layered component graph} that exposes a coarse-to-fine hierarchy.

% While \textsc{LILaC}~\cite{1_lilac} previously introduced a layered graph to model components and subcomponents, it focused primarily on component-level traversal; document-level information (e.g., titles) was treated merely as metadata labels attached to component nodes. 
% We argue that this is insufficient for open-domain traversal, where an agent must efficiently prune vast search spaces. 
% Therefore, we extend the \textsc{LILaC} formulation by introducing an explicit \textbf{Document Layer} ($V_0$). 
% This allows the system to assess document relevance via summaries before expending resources on specific components.



% \noindent\textbf{Navigation Edges ($E_{nav}$).}
% We unify all lateral and cross-reference connections into a single set of navigation edges $E_{nav}$, enabling both local context flow and global jumps.
% \squishlist
%     \item \textit{Local Context (Intra-document)}: 
%     To capture the flow within a document $D_j$, we add edges between adjacent or co-occurring components in $V_1$. 
%     For example, $(v(C_{j,i}), v(C_{j,i+1})) \in E_{nav}$ allows the agent to ``read around'' to neighboring paragraphs or images without leaving the current document context.
    
%     \item \textit{Global Links (Cross-document)}: 
%     We incorporate the link signal $\mathcal{L}$. 
%     If a component $C \in V_1$ references a target document $D_{k}$ (i.e., $\mathcal{L}(C) = D_{k}$), we add an edge $(v(C), v(D_{k})) \in E_{nav}$. 
%     Unlike component-to-component links, this routes navigation through the target's Layer 0 summary node. 
%     This forces the agent to evaluate the target document's high-level relevance before descending into its details, facilitating efficient pruning.
% \squishend



% % ChatGPT
% \subsection{Layered Component Graph}
% \label{sec:lcg}

% Real-world multimodal corpora are not a flat set of independent chunks. Within a document, the meaning of a
% component (paragraph/table/image) is shaped by local context (captions, surrounding text, section adjacency),
% and across documents, components are connected by explicit navigation signals (hyperlinks, citations, cross-references).
% A flat index discards these structural cues, making multihop evidence discovery brittle. We therefore represent the corpus
% as a \emph{layered component graph} that (i) preserves navigational relations for multihop traversal and (ii) exposes a
% coarse-to-fine containment hierarchy for efficient hopping and precise evidence extraction.

% We follow the layered construction of \textsc{LILaC}~\cite{1_lilac} and introduce a lightweight extension that adds an explicit
% \emph{document layer} to enable coarse, semantics-guided jumps via document summaries.

% \noindent\textbf{Graph, layers, and node contents.}
% Let $D=\{D_j\}_{j=1}^{k_{\text{doc}}}$ be the corpus, where each document $D_j=[C_{j,1},\ldots,C_{j,k_j}]$ is an ordered list of
% components, and let $\mathcal{C}=\bigcup_{j=1}^{k_{\text{doc}}}\{C_{j,1},\ldots,C_{j,k_j}\}$ denote the global component pool.
% We assume a link signal $L:\mathcal{C}\rightarrow D$ capturing cross-document pointers (e.g., hyperlinks).

% We define an \emph{extended layered component graph} as
% $G=(V,E,\lambda,\tau)$, where $\lambda$ is a layer map and $\tau$ is a type map.
% We use three layers: a document layer, a coarse component layer, and a fine-grained subcomponent layer:
% \squishlist
%     \item \textit{Document nodes} $V_D=\{v^D_j\}_{j=1}^{k_{\text{doc}}}$, where each $v^D_j$ corresponds to $D_j$.
%     \item \textit{Coarse component nodes} $V_0 = V_{\text{para}}\cup V_{\text{tbl}}\cup V_{\text{img}}$, where each node $v(C)\in V_0$
%           corresponds to a top-level component $C\in\mathcal{C}$ (paragraph/table/image).
%     \item \textit{Fine subcomponent nodes} $V_1 = V_{\text{sent}}\cup V_{\text{row}}\cup V_{\text{obj}}$, where each node $v(c)\in V_1$
%           corresponds to a subcomponent $c\in S(C)$ (sentence/row/object), as available.
% \squishend
% The full node set is $V = V_D \cup V_0 \cup V_1$.
% We set the layer map $\lambda:V\rightarrow\{-1,0,1\}$ by $\lambda(v)=-1$ for $v\in V_D$, $\lambda(v)=0$ for $v\in V_0$, and
% $\lambda(v)=1$ for $v\in V_1$. The type map is $\tau:V\rightarrow\{\text{doc},\text{para},\text{tbl},\text{img},\text{sent},\text{row},\text{obj}\}$.

% Each node $v$ stores a content field $x(v)$ used for retrieval and reasoning.
% For $v^D_j$, we set $x(v^D_j)$ to a concise textual \emph{summary} of $D_j$ (e.g., generated offline), so that document-level hops
% are guided by high-level semantics. For coarse component nodes $v(C)\in V_0$, $x(v(C))$ is the component content.
% For fine nodes $v(c)\in V_1$, $x(v(c))$ is the corresponding subcomponent content.

% \noindent\textbf{Containment edges.}
% We add directed containment edges that encode the document$\rightarrow$component$\rightarrow$subcomponent hierarchy:
% \squishlist
%     \item \textit{Document$\rightarrow$component}: $E_{D\downarrow}=\{(v^D_j, v(C_{j,i})) \mid C_{j,i}\in D_j\}$.
%     \item \textit{Component$\rightarrow$subcomponent}: $E_{\downarrow}=\{(v(C), v(c)) \mid c\in S(C)\}$.
% \squishend

% \noindent\textbf{Inter-component edges (coarse layer).}
% Following \textsc{LILaC}~\cite{1_lilac}, we encode navigational relations among coarse components with
% $E_0 = E_{\text{intra}} \cup E_{\text{inter}} \subseteq V_0\times V_0$:
% \begin{align}
% E_{\text{intra}} &= \{(v(C_i), v(C_j)) \mid C_i \neq C_j,\; C_i,C_j \in D_j \}, \label{eq:eintra}\\
% E_{\text{inter}} &= \{(v(C), v(C')) \mid L(C)=D_{j'},\; C'\in D_{j'} \}. \label{eq:einter}
% \end{align}
% $E_{\text{intra}}$ captures within-document co-occurrence/proximity, while $E_{\text{inter}}$ preserves cross-document navigation by
% connecting a linking component to the components of the linked document.

% \noindent\textbf{Document-level navigation edges (our extension).}
% We lift the same link signal $L$ to the document layer to enable coarse global jumps before descending to components:
% \squishlist
%     \item \textit{Component$\rightarrow$document}: $E_{C\rightarrow D}=\{(v(C), v^D_{j'}) \mid L(C)=D_{j'}\}$.
%     \item \textit{Document$\rightarrow$document}: $E_{D\rightarrow D}=\{(v^D_j, v^D_{j'}) \mid \exists\, C\in D_j \text{ s.t. } L(C)=D_{j'}\}$.
% \squishend

% \noindent\textbf{Edge set and neighborhood.}
% The full edge set is
% $E = E_0 \cup E_{\downarrow} \cup E_{D\downarrow} \cup E_{C\rightarrow D} \cup E_{D\rightarrow D}$.
% We write $N(v)=\{u \mid (v,u)\in E\}$ for outgoing neighbors.
% This unified view allows retrieval to alternate between (i) document-level jumps guided by summaries,
% (ii) coarse-layer local hops via $E_0$, and (iii) fine-grained evidence extraction via $E_{\downarrow}$.



















%%%%%%%%%%%%%%%%% Version  1

% \new{

% \subsection{Layered Component Graph}
% \label{sec:lcg}

% We represent the corpus as a \emph{layered component graph} that makes both intra-document structure
% and cross-document navigation explicit. We follow the layered construction of \textsc{LILaC}~\cite{1_lilac},
% and introduce a lightweight extension that adds an explicit \emph{document layer}.

% \noindent\textbf{Nodes and layers.}
% Let $\mathcal{G}=(\mathcal{V},\mathcal{E})$ denote the graph.
% We define two primary node types:
% \squishlist
%     \item \textit{Document nodes} $\mathcal{V}_D=\{v^D_j\}_{j=1}^{k_{doc}}$, where each $v^D_j$ corresponds to a document $D_j$.
%     \item \textit{Component nodes} $\mathcal{V}_C=\{v(C): C\in\mathcal{C}\}$, where each $v(C)$ corresponds to a component $C$.
% \squishend
% Each node $v$ stores a content field $x(v)$ used for retrieval and reasoning.
% For $v^D_j$, we set $x(v^D_j)$ to a concise textual \emph{summary} of $D_j$ (e.g., generated offline),
% so that document-level hops can be guided by high-level semantics rather than raw component text.
% For component nodes $v(C)$, $x(v(C))$ is the component content (paragraph/table/image).

% Optionally, a component can be further decomposed into finer-grained subcomponents (e.g., sentences,
% table regions/rows, or image regions), yielding additional layers.
% We denote the full node set as $\mathcal{V}=\mathcal{V}_D \cup \mathcal{V}_C \cup \mathcal{V}_{\text{sub}}$,
% where $\mathcal{V}_{\text{sub}}$ contains subcomponent nodes when available.
% (The document layer itself can also be hierarchical, e.g., page$\rightarrow$section$\rightarrow$subsection;
% our formulation naturally extends by adding more document-layer levels.)

% \noindent\textbf{Hierarchical (containment) edges.}
% We add directed containment edges from documents to their top-level components:
% $(v^D_j, v(C_{j,i}))\in\mathcal{E}$ for all $C_{j,i}\in D_j$.
% If subcomponents exist, we also add edges from a component to its subcomponents.

% \noindent\textbf{Intra-document edges.}
% To capture local context within a document, we connect top-level component nodes that co-occur within
% the same document (e.g., adjacency or section-level proximity). Concretely, for each $D_j$ we add
% edges among $\{v(C_{j,1}),\ldots,v(C_{j,k_j})\}$ to enable local hops that exploit within-document context.

% \noindent\textbf{Navigation edges (cross-document).}
% We incorporate the link signal $\mathcal{L}:\mathcal{C}\rightarrow\mathcal{D}$ (hyperlinks, cross-document pointers).
% For a link from a component $C\in D_j$ to a target document $D_{j'}$ (i.e., $\mathcal{L}(C)=D_{j'}$), we add:
% \squishlist
%     \item \textit{Component$\rightarrow$document}: $(v(C), v^D_{j'})\in\mathcal{E}$.
%     \item \textit{Document$\rightarrow$document}: $(v^D_{j}, v^D_{j'})\in\mathcal{E}$.
% \squishend
% Thus, any cross-document pointer emitted by a component induces a document-level navigation edge,
% allowing the retriever to move between documents at a coarse level before descending to components.

% \noindent\textbf{Neighborhood.}
% We write $\mathcal{N}(v)=\{u\mid (v,u)\in\mathcal{E}\}$ for outgoing neighbors.
% This unified graph view allows retrieval to alternate between (i) global document-level jumps guided by summaries,
% (ii) local hops via $\mathcal{N}(\cdot)$, and (iii) component-level evidence extraction.

% }




% LILaC에서의 layered component graph를 바탕으로 간단히 서술하고, 
% 우리가 이를 extend하여서 document layer를 추가하고, 해당 node 내에는 document에 대한 summary를 기재하도록 하였다고 할 것.
% Document layer 간에서도 layer가 있을 수 있고, A라는 document에서 a라는 child component에서 B라는 document로의 edge (i.e., hyperlink)가 존재하면 A에서 B node로의 edge를 생성하였다고 할 것 (A, a, B로 설명하지 말고 식으로 설명)





\section{Proposed Method}


We propose \texttt{LILaC}, a novel retrieval algorithm utilizing a layered component graph and traversal method to retrieve a query-relevant subgraph. 
As shown in Figure~\ref{fig:idea_overview}, it consists of two stages: 
(i) \textbf{Layered Graph Construction} organizes multimodal documents into a layered component graph with explicit intra- and inter-document edges. 
(ii) \textbf{Late-interaction-based Subgraph Retrieval} iteratively traverses the layered graph in an edge-wise manner.
To score an edge using node-level embeddings, it uses late interaction between the decomposed subqueries and low-layer subcomponents of an edge.
















\subsection{Layered Component Graph Construction}
% 0.75페이지


In the offline phase, \texttt{LILaC} constructs a layered graph structure $\mathcal{G}$, called the \textit{layered component graph}, from the multimodal document set $\mathcal{D}$ and the associated link mapping $\mathcal{L}$.
\updated{
The graph is designed to represent \emph{relationships among components} while also allowing each component to be expressed via \emph{fine-grained constituent elements}.
It comprises two distinct layers explicitly designed to represent semantic relationships among multimodal components, offering two primary advantages.
First, the top layer supports multihop retrieval by explicitly modeling relationships between components and documents, enabling identification of relevant contexts.
Second, the lower layer facilitates precise, fine-grained reasoning by further decomposing components into finer \textit{subcomponents} (defined in Definition~\ref{def:subcomponent}), thus providing detailed context for accurate retrieval.
In addition, the edges explicitly encode two relations among these nodes: 
(i) \emph{hierarchical containment}, which links coarse components to fine-grained subcomponents; 
and (ii) \emph{navigational relations}, which preserve potential cross-component affinity (both intra- and cross-document) without prematurely committing to a specific semantic.}

% This graph comprises two distinct layers explicitly designed to represent semantic relationships among multimodal components, offering two primary advantages.
% First, the top layer supports multihop retrieval by explicitly modeling relationships between components and documents, enabling identification of relevant contexts.
% Second, the lower layer facilitates precise, fine-grained reasoning by further decomposing components into finer \textit{subcomponents}, thus providing detailed context for accurate retrieval.

\begin{definition}[\textbf{Layered Component Graph}]
\label{def:layered_component_graph}
We define a \textit{layered component graph} as $\mathcal{G} = (V, E, \lambda, \tau)$, where $V$ is a set of vertices.
A vertex $v$ belongs to one of the two layers, determined by the layer map \(\lambda : V \to \{0,1\}\), where $0$ and $1$ corresponds to the coarse-grained and fine-grained nodes, respectively.
\vspace{-3mm}
\begin{align*}
    V_0 &= V_{\text{para}}\;\cup\;V_{\text{tbl}}\;\cup\;V_{\text{img}} \\ 
    V_1 &= V_{\text{sent}}\;\cup\;V_{\text{row}}\;\cup\;V_{\text{obj}} 
    \vspace{-3mm}
\end{align*}
We denote each vertex set - \(V_{\text{para}}\): paragraphs, \(V_{\text{tbl}}\): tables, \(V_{\text{img}}\): images, \(V_{\text{sent}}\): sentences, \(V_{\text{row}}\): table rows, \(V_{\text{obj}}\): visual objects detected in images.
The type map
\(\tau : V \to
  \{\texttt{para},\texttt{tbl},\texttt{img},\texttt{sent},\texttt{row},\texttt{obj}\}
\)
refines the vertex set $V$ into the six disjoint categories.
The edge set \(E \subseteq V \times V\) is the union \(E = E_{0} \cup E_\downarrow\) where
\vspace{-3mm}
\begin{align*}
  E_0 &\;=\; \bigl\{(u,v)\in V_0^2\}              \\
  E_\downarrow &\;=\; \bigl\{(u, v)\mid u \in V_0, v \in V_1\}
  \vspace{-3mm}  
\end{align*}
\(E_0\) captures \emph{relationships} between the macro components, while \(E_\downarrow\) captures the containment of a macro component of its subcomponent.
\end{definition}

\begin{definition}[\textbf{Subcomponent}]
\label{def:subcomponent}
Let \(C\) be a multimodal component.  
A \emph{subcomponent} \(c \in \mathcal{S}(C)\) is defined in a modality-specific manner:
\squishlist
    \item \textbf{Paragraph.}  
          For a paragraph \(P = [p_1,\dots,p_{k_{sent}}]\) consisting of sentences, each sentence $p_j$ is a subcomponent.
    \item \textbf{Table.}  
          Let \(T = [T_0;T_1;\dots;T_{k_{row}}]\) where \(T_0\) is the header row.  
          For every data row \(T_i \;(1 \le i \le k_{row})\), the two-row segment $t_i = [\,T_0;\,T_i\,]$ is a subcomponent.
    \item \textbf{Image.}  
          Given an image tensor \(I \in \mathbb{R}^{w \times h \times a}\) and an object detector that
          returns a bounding box \((x_1,y_1,x_2,y_2)\), the corresponding patch
          \vspace{-3mm}
          \[
            i = I[x_1:x_2,\;y_1:y_2,\,:]
            \vspace{-3mm}
          \]
          is a subcomponent.
\squishend
\end{definition}

% Paragraph 2) Tree를 먼저 만들고, 이들끼리 link를 만들어 최종 graph를 생성한다.
Layered component graph $\mathcal{G}$ is constructed in two steps.
First, \texttt{LILaC} builds a \textit{component tree} for each component $C$ within $\mathcal{D}$.
A component tree is a two-level tree structure with the root representing the component itself and its children representing the subcomponents, which are extracted differently depending on the modality of the component.
\updated{The roots and leaves of these trees form the nodes of $V_0$ and the nodes of $V_1$, respectively, while the parent–child links correspond to the edges in $E_{\downarrow}$.}
For a paragraph $P$, \texttt{LILaC} utilizes a Sentence-aware Transformer (\texttt{SaT}) model to split it into a set of sentences.
A table $T$ is parsed to generate a set of table segments.
Lastly, a multimodal LLM is used to detect objects within $I$.
\texttt{LILaC} then generates an edge $(C, c) \in E_\downarrow$ for $c \in \mathcal{S}(C)$.

    % Paragraph 3) Component끼리의 edge를 만드는 방법
In the next step, \texttt{LILaC} generates the inter-component edges $E_0$ using both inherent structural relationships and hyperlink-based connections. 
For every document $D \in \mathcal{D}$, a clique is formed among its components:
\vspace{-3mm}
\begin{equation}
    E_{intra} = \{(C_i, C_j) | C_i \neq C_j, C_i, C_j \in D\}
    \vspace{-3mm}
\end{equation}
To enable cross-document multihop reasoning, \texttt{LILaC} then follows the link mapping $\mathcal{L}$.
For each pair $(C, D) \in \mathcal{L}$, it connects $C$ to every component in the linked document $\mathcal{D}$.
\vspace{-3mm}
\begin{equation}
    E_{inter} = \{(C, C') | (C, D) \in \mathcal{L}, C' \in D\}
    \vspace{-3mm}
\end{equation}
The inter-component edge set for the top layer is therefore $E_0 = E_{intra} \cup E_{inter}$.
Finally, every node $v \in V$ receives an embedding $\textbf{v} = f(v)$ from a pre-trained multimodal encoder $f$.











\subsection{Late-interaction-based Subgraph Retrieval}


During the online phase, \texttt{LILaC} retrieves a query-relevant subgraph $\mathcal{G}'$ from the layered component graph $\mathcal{G}$ given a query $Q$.  
This retrieval faces two key challenges: 
(1) Direct identification of an optimal subgraph from all possible candidates is computationally infeasible due to a combinatorial explosion~\cite{grag}. 
In particular, the layered component graph contains numerous edges, making explicit embedding of all edges prohibitively expensive in terms of space and computation.
(2) Queries often lack explicit modality instructions, causing ambiguity for multimodal embedders, particularly in complex multihop scenarios~\cite{uniir}.
To address these, we introduce a two-step retrieval strategy: 
(i) \textit{LLM-driven query decomposition}, which explicitly generates modality-specific subqueries, and 
(ii) \textit{Late-interaction-guided graph traversal}, a beam-search traversal method dynamically scoring edges based on fine-grained interactions within the low-level nodes.






\subsubsection{LLM-driven Query Decomposition}

Given a potentially complex query $Q$, \texttt{LILaC} first leverages an LLM to explicitly decompose $Q$ into simpler modality-specific subqueries.
Specifically, we utilize a zero-shot prompting strategy to generate a small set of subqueries:
\vspace{-2mm}
\begin{equation}
\{q_1, \dots, q_{k_{sub}}\} = \text{LLM}(Q;\, prompt_{\mathrm{dec}})
    \vspace{-2mm}
\end{equation}
Each subquery is then classified into a modality label  
$m_j\!\in\!\{\texttt{text},\texttt{table},\texttt{image}\}$ with a second prompt:
\vspace{-2mm}
\begin{equation}
m_j \;=\; \text{LLM}(q_j;\, \textit{prompt}_{\mathrm{mod}})
    \vspace{-2mm}
\end{equation}
Using these labels, we obtain modality-specific embeddings $\mathbf{q}_j = f(q_j;\, m_j)$ for every subquery, while the original query is embedded coarsely as $\mathbf{Q} = f(Q;\, \varepsilon)$ to seed the initial candidate search.  
We denote the set of embedded subqueries as $\textbf{Q}_{\text{sub}} = \{\mathbf q_1,\dots,\mathbf q_{k_{\text{sub}}}\}$.
Full prompt templates appear in \S\ref{sec:prompt_templates}.





















\subsubsection{Late-interaction-guided Graph Traversal}
\label{sec:late_interaction}

At inference time, \texttt{LILaC} searches for a subgraph $\mathcal{G}'\!\subseteq\!\mathcal{G}$ that best matches the query.  
\texttt{LILaC} maintains a beam of size $b$ and iteratively identify a candidate subgraph $\mathcal{G}_t=({V}_t,{E}_t, \lambda, \tau)$ consisting of $b$ edges.
Initially, to efficiently narrow the search space from numerous candidate nodes, \texttt{LILaC} identifies a set of top-$b$ top-level nodes $V_0$ most relevant to the query.
\begin{equation}
{V}_{0}
  = \operatorname*{arg\,max}_{C\in V_0}^{b}
    \operatorname{sim}\!\bigl(\mathbf{Q},\mathbf{C}\bigr),
\quad
{E}_{0}= \{\} 
\end{equation}


\texttt{LILaC} then initiates iterative traversal of the graph starting from these candidate nodes. 
In each iteration, \texttt{LILaC} first expands the candidate nodes via one-hop traversal to consider adjacent nodes, dynamically computing query-relevance scores for all edges formed by these expansions. 
Subsequently, only the top-$b$ scored edges are retained for the next iteration forming subgraph, and their constituent nodes become the new set of candidate nodes, forming $\mathcal{G}_i = (V_i, E_i, \lambda, \tau)$. 
After the final iteration $n_i$, \texttt{LILaC} returns the top-$n_{ret}$ nodes from the final subgraph $\mathcal{G}_{n_i}$.

\begin{figure}[ht]
  \centering
  \includegraphics[width=\linewidth]{figures/late_interaction.pdf}
  \caption{An example case of edge-level late interaction.}
  \label{fig:late_interaction}
\end{figure}

\textbf{Late Interaction Edge Scoring.}
As previously discussed, naively calculating edge scores negatively impacts both effectiveness and efficiency. 
Specifically, this is because 
(1) subqueries, each potentially targeting distinct modalities, must accurately align with the relevant nodes, and 
(2) embedding all edges within the layered graph is inefficient due to their vast number.

To efficiently address these issues, \texttt{LILaC} employs a \emph{late interaction} strategy, scoring each edge on-the-fly with \emph{fine-grained} evidence.
\updated{\texttt{LILaC} extends the standard token-level late interaction to operate at the node-subquery level, by matching decomposed subqueries against the subcomponents contained within an edge.}
Let an edge be $e = (C_\alpha, C_\beta)$ and $\mathcal{S}_e = \mathcal{S}(C_\alpha) \cup \mathcal{S}(C_\beta)$.
\texttt{LILaC} gathers every subcomponent that could provide evidence on either side of the edge in the set $\mathcal{S}_e$.
\vspace{-3mm}
\begin{equation}
s(e;\textbf{Q}_{sub})
    \;=\;
    \sum_{\mathbf q\in\mathbf{Q}_{sub}}
        \max_{c\in\mathcal S_e}
        \operatorname{sim}\bigl(f(c),\mathbf q\bigr).
    \vspace{-3mm}
\label{eq:edge_score}
\end{equation}
The inner \(\max\) selects, for each sub-query \(\mathbf q\), the single most relevant sub-component \(\mathbf c\) incident to the edge, while the outer sum ensures every sub-query contributes exactly once. 
Figure~\ref{fig:late_interaction} shows two example cases of late interaction scoring.
This scoring approach is designed to reflect practical scenarios where each subquery specifically targets fine-grained details located within particular subcomponents.
By aggregating the maximum similarity scores across these detailed elements, rather than relying solely on coarse component embeddings, \texttt{LILaC} effectively prioritizes precise, subcomponent-level matches. 
This strategy enhances retrieval accuracy by focusing directly on relevant information, reducing the noise introduced by broader, less relevant contexts.



We introduce two special cases of edge scoring: 
\textit{(i) Isolated nodes.}  
If a component $C$ has no explicit neighbor, we introduce a dummy edge $(C,\varepsilon)$ so that $C$ can still be considered.
\textit{(ii) One-sided matches.}  
If an edge score $s(e;Q)$ equals the best single-node score of one endpoint, we return only that node to avoid including irrelevant neighbors.
Refer to Figure~\ref{fig:late_interaction} (b) for a specific example.




\vspace{-2mm}
\section{Experiments}
\label{sec:experiments}
\vspace{-1mm}







\begin{table*}[!t]
    \caption{Efficiency (Time, \# LLM Calls and API Usage) comparison of \textsc{\Ours} and its competitors for the three benchmarks.}
    \vspace{-3.5mm}
    \label{tab:retrieval_efficiency}
    \begin{center}
    \begin{small}
    \scalebox{0.82}{
    \renewcommand{\arraystretch}{0.95}
    \resizebox{1.2\textwidth}{!}{
    \begin{tabular}{llrrrrrrrr}
    
        \toprule
        \multirow{2}{*}{Dataset} & \multirow{2}{*}{Algorithm} & 
        \multicolumn{4}{c}{Time (ms)} & \multirow{2}{*}{\# LLM Calls} &
        \multicolumn{3}{c}{API Usage} \\
        \cmidrule(lr){3-6}\cmidrule(lr){8-10}
        & & Total & LLM & Vector Search & Embedding & & \# Input Toks & \# Output Toks & \$ \\
        \midrule
        
        \multirow{5}{*}{\MultimodalQA}
        & \textsc{VisRAG}  & 371    & 0      & 218   & 153 & 0.00 & 0      & 0     & 0.0000 \\
        & \textsc{ColPali} & 9,849  & 0      & 9,210 & 639 & 0.00 & 0      & 0     & 0.0000 \\
        & \textsc{LILaC}   & 19,528 & 15,943 & 3,153 & 432 & 1.00 & 1,165  & 1,119 & 0.0115 \\
        & \textsc{IRCoT}   & 95,744 & 95,140 & 119   & 484 & 5.44 & 41,943 & 3,296 & 0.0752 \\
        & \textsc{\Ours}   & 114,591& 113,642& 538   & 411 & 7.92 & 48,863 & 4,785 & 0.1109 \\
        \midrule
        
        \multirow{5}{*}{\MMCoQA}
        & \textsc{VisRAG}  & 362    & 0      & 215   & 147 & 0.00 & 0      & 0     & 0.0000 \\
        & \textsc{ColPali} & 1,836  & 0      & 1,173 & 663 & 0.00 & 0      & 0     & 0.0000 \\
        & \textsc{LiLaC}   & 20,244 & 16,879 & 2,977 & 388 & 1.00 & 1,160  & 1,041 & 0.0107 \\
        & \textsc{IRCoT}   & 97,816 & 97,220 & 116   & 479 & 5.61 & 52,286 & 3,360 & 0.0753 \\
        & \textsc{\Ours}   & 105,355& 104,630& 366   & 359 & 7.31 & 53,149 & 4,151 & 0.1037 \\ 
        \midrule
        
        \multirow{5}{*}{\WebQA}
        & \textsc{VisRAG}  & 386    & 0      & 225   & 161 & 0.00 & 0      & 0     & 0.0000 \\
        & \textsc{ColPali} & 7,919  & 0      & 7,298 & 621 & 0.00 & 0      & 0     & 0.0000 \\
        & \textsc{LILaC}   & 19,187 & 14,822 & 3,782 & 583 & 1.00 & 1,162  & 737   & 0.0077 \\
        & \textsc{IRCoT}   & 173,815& 173,011& 152   & 651 & 6.94 & 65,167 & 3,913 & 0.1082 \\
        & \textsc{\Ours}   & 128,748& 127,664& 581   & 505 & 8.61 & 64,683 & 5,159 & 0.1278 \\
        \bottomrule
        
    \end{tabular}
    }
    \renewcommand{\arraystretch}{1.0}
    }
    \end{small}
    \end{center}
    \vspace{-4mm}
\end{table*}








\begin{table*}[htbp]
    \caption{Ablation study analyzing retrieval accuracy and efficiency of different \textsc{\Ours} variants.}
    \vspace{-3.5mm}
    \label{tab:ablation_study}
    \begin{center}
    \begin{small}
    \scalebox{0.82}{
    \renewcommand{\arraystretch}{0.95}
    \resizebox{1.2\textwidth}{!}{
    \begin{tabular}{llccrcccc}
    
        \toprule
        \multirow{2}{*}{Dataset} & \multirow{2}{*}{Variation} &
        \multicolumn{2}{c}{Retrieval Accuracy} &
        \multicolumn{5}{c}{Efficiency} \\
        \cmidrule(lr){3-4}\cmidrule(lr){5-9}
        & & R@10 & MRR@10 & Time (ms) & \# LLM Calls & \# Input Toks & \# Output Toks & \$ \\
        \midrule
        
        \multirow{6}{*}{\footnotesize \MultimodalQA}
        & \textsc{\OurFullName}                        & 85.82 & 86.82 & 
                                                    117,651 & 7.97 & 49,153 & 4,812 & 0.1111 \\
                                                    
        & w/o Backtracking Orchestration        & 79.26 & 84.31 &
                                                    211,951 & 12.53 & 78,864 & 6,113 & 0.1502 \\
                                                    
        & w/o LLM Reasoning in traversal agent  & 78.04 & 83.97 & 
                                                    168,780 & 7.64 & 45,079 & 4,203 & 0.0887 \\
                                                    
        & w/o Global Hop                        & 75.79 & 82.56 & 
                                                    85,417 & 7.02 & 51,839 & 4,137 & 0.1039 \\
                                                    
        & w/o Vector Granularity                & 83.16 & 85.43 &
                                                    116,918 & 8.28 & 55,281 & 5,179 & 0.1158 \\
                                                    
        & w/o Subquery Planner                  & 76.2 & 83.77 & 
                                                    66,974 & 5.31 & 35,984 & 3,335 & 0.0765 \\
        \midrule
        
        \multirow{6}{*}{\WebQA}
        & \textsc{\OurFullName}                        & 81.91 & 89.24 & 
                                                    126,567 & 8.55 & 63,971 & 5,103 & 0.1270 \\
                                                    
        & w/o Backtracking Orchestration        & 77.45 & 87.62 & 
                                                    228,934 & 13.77 & 89,244 & 7,214 & 0.1715 \\
                                                    
        & w/o LLM Reasoning in traversal agent  & 78.92 & 88.82 & 
                                                    135,274 & 7.57 & 50,501 & 4,275 & 0.0960 \\
                                                    
        & w/o Global Hop                        & 73.30 & 86.83 & 
                                                    93,115 & 9.05 & 68,483 & 4,082 & 0.1247 \\
                                                    
        & w/o Vector Granularity                & 79.21 & 87.77 &
                                                    132,972 & 8.72 & 66,201 & 5,319 & 0.1290 \\
                                                    
        & w/o Subquery Planner                  & 76.38 & 89.95 & 
                                                    85,583 & 5.40 & 40,721 & 3,358 & 0.0820 \\
        \bottomrule
        
    \end{tabular}
    }
    \renewcommand{\arraystretch}{1.0}
    }
    \end{small}
    \end{center}
    \vspace{-6mm}
\end{table*}
















\vspace{-1mm}
\subsection{Experimental Setups}
\vspace{-1mm}


\noindent\textbf{Datasets and evaluation metrics.}
We evaluate open-domain multimodal \emph{component} retrieval and downstream QA on three benchmarks: \textsc{MultimodalQA}~\cite{multimodalqa}, \textsc{MMCoQA}~\cite{mmcoqa}, and \textsc{WebQA}~\cite{webqa}.
Following \textsc{LILaC}~\cite{1_lilac}, we use the URL-annotated setting to reconstruct realistic webpage-style corpora, parsing each page into multimodal components (paragraphs, tables, images).
This yields \MultimodalQA (3,235 pages, avg. $\sim$37 components), \MMCoQA (453 pages, avg. $\sim$32 components), and \WebQA (7,662 pages, avg. $\sim$13 components).
Consistent with prior work~\cite{1_lilac}, we report retrieval Recall@$k$ (R@$k$, $k\in\{1,2,5,10\}$) and MRR@10: R@$k$ checks whether at least one ground-truth component appears in the top-$k$ list, and MRR@10 captures the rank of the first relevant component.
For end-to-end QA, we feed the top-$10$ retrieved components into the same multimodal LLM and report Exact Match (EM) and token-level F1.

\vspace{-1mm}
\noindent\textbf{Compared methods.}
We compare \textsc{\Ours} with strong baselines spanning graph traversal, agentic retrieval, and single-shot indexing.
We include \textsc{LILaC}~\cite{1_lilac}, a layered-graph retriever designed for multi-hop scenarios, and \textsc{IRCoT}~\cite{9_ircot}, an agentic retriever that interleaves retrieval with chain-of-thought reasoning.
Since \textsc{IRCoT} was originally proposed for text-only corpora, we adapt it to our multimodal setting by (i) replacing its retriever with the same multimodal embedder used throughout our experiments and (ii) using the same multimodal LLM for reasoning and generation over multimodal components.
For \emph{VisRAG} approaches, we employ \textsc{VisRAG-Ret}~\cite{3_visrag}, which directly encodes document images via VLMs, and \textsc{ColPali}~\cite{4_colpali}, which uses late-interaction multi-vector embeddings from document images.
We also compare with \textsc{NV-Embed-v2}~\cite{nvembedv2}, a \emph{TextRAG} baseline that embeds textualized components.

\vspace{-1mm}
\noindent\textbf{Model configurations.}
To ensure fair comparison, we standardize backbone models across all methods whenever applicable.
We use \textsc{MM-Embed}~\cite{mmembed} as the unified multimodal embedder, and the \textsc{OpenAI API} (\textsc{gpt-5})~\cite{gpt5} with \texttt{reasoning\_effort = low} as the multimodal LLM for all planning, reasoning, reranking, and generation steps.


% \noindent\textbf{Datasets \& evaluation metrics.}
% We evaluate open-domain multimodal component retrieval and downstream QA on three benchmarks: \textsc{MultimodalQA}~\cite{multimodalqa}, \textsc{MMCoQA}~\cite{mmcoqa}, and \textsc{WebQA}~\cite{webqa}. 
% Following \textsc{LILaC}~\cite{1_lilac}, we use the URL-annotated setting to reconstruct realistic webpage-style corpora, parsing them into multimodal components (paragraphs, tables, images). 
% This yields \MultimodalQA (3,235 pages, avg. $\sim$37 components), \MMCoQA (453 pages, avg. $\sim$32 components), and \WebQA (7,662 pages, avg. $\sim$13 components).

% Consistent with prior work~\cite{1_lilac}, we evaluate retrieval using Recall@$k$ (R@$k$) for $k\in\{1,2,5,10\}$ and Mean Reciprocal Rank (MRR@10).
% R@$k$ measures the presence of at least one ground-truth component in the top-$k$ results, while MRR@10 captures the rank of the first relevant component. 
% For end-to-end QA, we feed the top-$10$ retrieved components into a multimodal LLM and report Exact Match (EM) and token-level F1.


% \noindent\textbf{Compared methods.}
% We compare \textsc{\Ours} against several state-of-the-art baselines across different retrieval paradigms.
% We mainly compare \textsc{LILaC}~\cite{1_lilac}, a layered graph retriever for multi-hop scenarios, and \textsc{IRCoT}~\cite{9_ircot}, an agentic retriever that interleaves retrieval with chain-of-thought reasoning. 
% IRCoT는 text에 대해서만 제안되었지만, multimodal embedder과 multimodal LLM을 사용하여 multimodal content에 대해 대응 가능하게끔 변형을 가하였음.
% For \emph{VisRAG} approaches, we employ \textsc{VisRAG-Ret}, which directly encodes document images via VLMs~\cite{3_visrag}, and \textsc{ColPali}, which utilizes late-interaction multi-vector embeddings from document images~\cite{4_colpali}.
% We also compare with \textsc{NV-Embed-v2}, a baseline \emph{TextRAG} method that uses a 7.85B model to embed textualized components~\cite{nvembedv2}.


% \noindent\textbf{Model Configurations.} 
% To ensure a fair and controlled comparison, we standardize the backbone models across all evaluated methods. 
% Specifically, we employ \textsc{MM-Embed}~\cite{mmembed} as the unified multimodal embedder and utilize the \textsc{OpenAI API} (\textsc{gpt-5})~\cite{gpt5} with \texttt{reasoning\_effort = low} as the multimodal LLM for all reasoning and generation tasks.











\vspace{-1mm}
\subsection{Retrieval Accuracy Comparison}
\label{sec:retrieval_accuracy}
\vspace{-1mm}


We evaluate open-domain multimodal \emph{component} retrieval on \MultimodalQA, \MMCoQA, and \WebQA\ using Recall@$k$ ($k\in\{1,2,5,10\}$) and MRR@10, with results reported in Table~\ref{tab:retrieval_performance}.
\textsc{\Ours} achieves the best performance across all three benchmarks and all reported cutoffs, indicating its overall effectiveness.
Averaged across datasets, \textsc{\Ours} reaches average Recall@10 of 80.26 and average MRR@10 of 83.16, improving over \textsc{LILaC} +22.03\% and +18.60\%, respectively.
Compared with the strongest agentic baseline \textsc{IRCoT}, \textsc{\Ours} gains 12.88\% Recall@10 and 9.31\% MRR@10.
The gap is even larger against single-shot embedding-based retrievers.

We analyze two interesting points.
One is that the largest dataset-specific margin over \textsc{LILaC} appears on \WebQA at higher cutoffs, suggesting that \WebQA more frequently requires \emph{escaping} local neighborhoods to reach dispersed evidence.
This aligns with \WebQA's construction where ground-truth components are not necessarily adjacent or tightly coupled, making global navigation critical.
In contrast, the performance gap over \textsc{IRCoT} is most pronounced on the more explicitly multi-hop benchmarks \MultimodalQA and \MMCoQA, where effectively leveraging the underlying link/structure signal (rather than pure global searching) is essential.
Second, improvements are particularly strong at low-$k$, indicating that \textsc{\Ours} not only increases \emph{coverage} of relevant components but also ranks them substantially earlier.


\vspace{-1mm}
\subsection{End-to-end QA Accuracy Comparison}
\vspace{-1mm}


We measure end-to-end QA performance by feeding the top-$10$ retrieved components into the same multimodal LLM generator for every method, and report EM and token-level F1 in Table~\ref{tab:qa_em_f1}.
\textsc{\Ours} is consistently the best-performing method on all three datasets, achieving average EM/F1 of 59.63/68.45.
Relative to \textsc{LILaC}, \textsc{\Ours} improves EM by +16.37\% and F1 by +16.79\% on average, confirming that more reliable evidence discovery yields better grounded generation.
Compared to \texttt{IRCoT}, \textsc{\Ours} still provides a clear advantage of +8.71\% EM and +7.14\% F1, despite both methods being agentic.
Dataset-wise, the gains are especially visible on \MMCoQA and \WebQA, consistent with Table~\ref{tab:retrieval_performance} where \textsc{\Ours} yields substantially higher top-$k$ retrieval accuracy.


\vspace{-1mm}
\subsection{Algorithm Efficiency}
\vspace{-1mm}


% We report efficiency metrics in Table~\ref{tab:retrieval_efficiency}, including wall-clock retrieval time (with breakdown into LLM, vector search, and embedding), the number of LLM calls, token usage, and estimated API cost.
% As expected, single-shot retrievers (\textsc{VisRAG}, \textsc{ColPali}, \textsc{NV-Embed-v2}) are the most efficient in wall-clock latency and incur no LLM API cost during retrieval.
% Traversal-based methods are more expensive due to iterative decision making.
% Among them, \textsc{LILaC} is relatively efficient because it uses a small, fixed number of LLM calls (1.00 per query), while \texttt{IRCoT} and \textsc{\Ours} perform multiple reasoning steps.
% Across datasets, \textsc{\Ours} uses 7.31--8.61 LLM calls per query and costs about \$0.10--\$0.13 per query, while delivering the best retrieval and QA accuracy.
% Notably, \textsc{\Ours} is comparable in runtime to \texttt{IRCoT}: it is slower on \MultimodalQA and \MMCoQA by +19.68\% and +7.71\%, but substantially faster on \WebQA by 25.93\%, yielding a slightly lower average runtime overall (116.23s vs.\ 122.46s).


% %jhyuntodo: 다른 식으로 분석
% The runtime breakdown highlights a clear bottleneck: for agentic methods, the LLM dominates the end-to-end retrieval time.
% For \textsc{\Ours}, LLM execution accounts for $\sim$99\% of total latency across datasets (e.g., 113,642ms out of 114,591ms on \MultimodalQA), while vector search and embedding remain below 1\%.
% This suggests that future efficiency gains will primarily come from reducing LLM calls, shortening prompts (e.g., more compact memory representations), or using cheaper models for easier hops while retaining strong models for ambiguous hops.
% In contrast, \textsc{ColPali} exhibits a different bottleneck: although it uses no LLM calls, its multi-vector late interaction makes vector search the dominant cost (e.g., 9,210ms vector search out of 9,849ms total on \MultimodalQA).
% Overall, Table~\ref{tab:retrieval_efficiency} reflects a practical trade-off: \textsc{\Ours} deliberately spends additional computation on adaptive exploration and failure-aware replanning/backtracking, which substantially improves evidence discovery quality and, consequently, downstream QA accuracy.



Table~\ref{tab:retrieval_efficiency} reports wall-clock retrieval time (with breakdown into LLM, vector search, and embedding), the number of LLM calls, token usage, and estimated API cost.
As expected, single-shot retrievers (\textsc{VisRAG}, \textsc{ColPali}, \textsc{NV-Embed-v2}) are the fastest and incur no LLM API cost during retrieval.
Among agentic methods, \textsc{\Ours} shows a slightly lower average runtime than \textsc{IRCoT} (116,231 vs.\ 122,458 ms), which is the strongest agentic baseline: 
while \textsc{\Ours} executes more reasoning steps (7.31--8.61 vs.\ 5.44--6.94 LLM calls), it uses fewer input tokens and achieves comparable or lower latency overall.
Notably, \textsc{\Ours} is substantially faster on \WebQA (128,748 vs.\ 173,815 ms; $-25.93\%$), suggesting that structure-aware navigation together with failure-aware re-anchoring reduces unproductive reasoning on large and sparsely connected corpora; 
The efficiency comes with a moderate increase in API usage compared to \textsc{IRCoT} (\$0.10--\$0.13 vs.\ \$0.075--\$0.108 per query), consistent with our higher retrieval/QA accuracy.
% , \textsc{\Ours} intentionally spends more computation.
Relative to \textsc{LILaC}, \textsc{\Ours} is 5.20--6.71$\times$ slower in wall-clock time and incurs 9.64--16.60$\times$ higher API cost, quantifying the additional budget required by adaptive agentic control.
% Finally, the breakdown shows that for agentic retrievers the LLM dominates end-to-end time (e.g., for \textsc{\Ours} $\sim$99\% of latency), whereas vector search and embedding contribute $<1\%$, indicating that future speedups should primarily target reducing LLM calls/prompt length or routing easy hops to cheaper models.






\vspace{-1mm}
\subsection{Ablation Study}
\vspace{-1mm}


% Table~\ref{tab:ablation_study} presents an ablation study that isolates the contribution of major design components in \textsc{\Ours}.
% We evaluate retrieval accuracy together with efficiency on \MultimodalQA and \WebQA.
% The tested variants remove: (i) \emph{LLM reasoning inside the traversal agent}, (ii) \emph{global hops}, (iii) \emph{adaptive vector-search granularity}, and (iv) the \emph{subquery planner}.
% First, LLM reasoning inside the traversal agent is crucial not only for accuracy but also for efficient navigation: removing it reduces R@10 by 7.78 points on \MultimodalQA and increases runtime by 43\%, indicating that reasoning helps avoid redundant exploration and reach informative regions more directly.
% Second, global hops are essential for escaping local neighborhoods: disabling global hops causes the largest accuracy drops on both \MultimodalQA (R@10: 75.79) and \WebQA (R@10: 73.30).
% The effect is particularly strong on \WebQA (drop of 8.61 points), consistent with the dataset-level behavior discussed in \S\ref{sec:retrieval_accuracy}.
% Third, the subquery planner exhibits a clear accuracy--efficiency trade-off.
% Removing replanning substantially reduces LLM calls and cost, but significantly degrades recall, suggesting that adaptive replanning is expensive but necessary for recovering from early planning mistakes.

% Table~\ref{tab:ablation_study} isolates the contribution of each major design component in \textsc{\Ours} on \MultimodalQA and \WebQA, reporting retrieval accuracy (R@10, MRR@10) together with efficiency.
% \MMCoQA는 제외하였고, conversational dataset이기에 conversation의 기록이 실험 결과를 왜곡시킬 수 있겠다 생각했다.
% Due to OpenAI API costs, we run these ablations on a representative 10\% subset of each dataset.

Table~\ref{tab:ablation_study} isolates the contribution of each major design component in \textsc{\Ours} on \MultimodalQA and \WebQA, reporting retrieval accuracy (R@10, MRR@10) alongside efficiency.
We omit \MMCoQA\ from this analysis: because it is conversational, accumulated dialogue history can introduce confounding factors (e.g., varying context length and carryover information) that may blur the impact of individual retrieval modules.
We run all ablations on a representative 10\% subset of each dataset due to OpenAI API costs.
(i) \textit{History-aware backtracking orchestration.} 
Removing backtracking consistently hurts both \emph{effectiveness} and \emph{efficiency}: R@10 drops by 4.46--6.56, while runtime increases by $\sim$80\% and LLM calls rise by 57--61\%.
This highlights that backtracking prevents wasted hops by adapting effort only when needed.
It also improves accuracy by returning to more promising anchors and narrowing candidates to the right neighborhood, rather than repeatedly exploring uninformative branches.
(ii) \textit{LLM reasoning inside the traversal agent.} 
Disabling LLM reasoning during traversal substantially degrades retrieval and can even \emph{slow down} the search: on \MultimodalQA, R@10 drops by 7.78 and runtime increases by 43\%, despite lower per-step cost.
We reason that additional replanning and extra iterations were triggered, as traversal is more likely to take ambiguous or unproductive hops without LLM reasoning.
(iii) \textit{Global hop.}
Global hops are crucial for escaping local neighborhoods.
When disabled, R@10 suffers the largest drop (10.03 on \MultimodalQA; 8.61 on \WebQA), even though the variant becomes faster.
This indicates that neighbor-based traversal alone is insufficient: some questions require jumping across distant regions of the corpus, including multi-hop paths and cases where relevant evidence is not directly linked.
(iv) \textit{Adaptive vector-search granularity.}
Removing granularity adaptation yields consistent but smaller drops (about 2.66--2.70 R@10) with minimal efficiency change.
(v) \textit{Subquery planner.}
emoving replanning reduces LLM calls and cost substantially (e.g., $-33\%$ to $-37\%$ calls), but causes large recall drops (9.62 on \MultimodalQA; 5.53 on \WebQA).
Without the option to revise the plan, the Orchestrator is more likely to terminate early once progress stalls, which simultaneously lowers cost and recall.
This underscores the importance of correcting early planning mistakes through evidence-conditioned replanning.


% 해당 실험은 데이터셋 별로 대표성을 잘 보이는 10\%의 샘플로 돌아갔다, OpenAI API의 값 때문에.
% Removing backtracking consistently degrades both \emph{effectiveness} and \emph{efficiency}: R@10 drops by 4.46$~$6.56, while runtime increases by $\sim$80\% and LLM calls increase by 57--61\%.
% Backtracking, 구체적으로 cost-aware strategy escalation이 efficiency에 지대한 영향을 끼치는 것을 확인할 수 있고, 또한 promising document로 돌아가 neighbor document로 후보를 올바르게 좁히는 것이 accuracy에도 큰 영향을 끼치는 것을 알 수 있다.
% Disabling LLM reasoning during traversal significantly hurts retrieval and can even \emph{slow down} the search: on \MultimodalQA, R@10 drops by 7.78 and runtime increases by 43\%, despite a lower cost.
% 오히려 search가 잘 되지 않음으로써 여러 번 subquery planning이 일어나고, 이로서 오히려 inaccurate하면서도 늦을 search가 일어나게 된다.
% Global hops are essential for escaping local neighborhoods.
% Without them, R@10 suffers the largest drop (10.03 on \MultimodalQA; 8.61 on \WebQA), even though the variant becomes faster.
% Retrieval을 하는 데 Neighbor도 물론 중요하지만, multi-hop이나 아예 연결되지 않은 도큐먼트를 고려하는 것이 매우 중요하다는 것이 보인다.
% Removing replanning reduces LLM calls and cost substantially (e.g., $-33\%$ to $-37\%$ calls), but causes large recall drops (9.62 on \MultimodalQA; 5.53 on \WebQA).
% Plan을 바꾼다는 선택지가 없기 때문에 orchestrator가 빠르게 retrieval을 포기하고, 이가 곧 recall과 cost의 동시 하락으로 이어진 것.
% 이로서 initial error를 수정하는 것에 대한 중요성이 보인다.








%%%%%%%%%%%%%%% Version 1




% Our proposed \textsc{\Ours} extends these by incorporating multi-strategy traversal with evaluation-driven feedback and history-aware backtracking.

% \noindent\textbf{Compared methods.}
% We compare \textsc{FiF} (reported as \textsc{\Ours} in tables) against representative multimodal retrieval baselines used in prior work:
% \squishlist
%     \item \textbf{NV-Embed-v2}: a strong \emph{TextRAG} baseline that embeds textualized components (e.g., linearized tables and text surrogates for visuals) and retrieves by dense similarity.
%     \item \textbf{VisRAG-Ret}: a \emph{VisRAG} retriever that directly encodes document images (or page screenshots) via a vision-language model and retrieves in a visual embedding space.
%     \item \textbf{ColPali}: a late-interaction, multi-vector retriever over document images.
%     \item \textbf{\textsc{LILaC}}: a layered component graph retriever that combines coarse candidate generation with fine-grained late-interaction scoring for multihop multimodal retrieval.
%     \item \textbf{IRCoT}: an agentic multi-step retriever that \emph{interleaves retrieval with chain-of-thought reasoning}, iteratively generating intermediate reasoning steps and using them to drive subsequent retrieval.
%     \item \textbf{\textsc{\Ours} (Ours)}: our \textsc{FiF} traversal framework with multi-strategy traversal, evaluation-driven feedback, and history-aware backtracking.
% \squishend
%     위 데이터셋 소개 부분 조금 더 간단히 표현
%     (예시: We employ two SOTA methods of VisRAG approaches - \texttt{VisRAG}, which directly encodes document images via VLMs~\cite{visrag}, and \texttt{ColPali}, which employs late-interaction multi-vector embeddings from document images~\cite{colpali}.
%     We additionally compare with \texttt{NV-Embed-v2}, a SOTA TextRAG method reported by \texttt{VisRAG}. It utilizes a 7.85B model for embedding textualized components.)



% \noindent\textbf{Applied multimodal embedding models \& multimodal LLMs.}
% 우리는 fair한 비교를 위해 모든 알고리즘에서 동일한 multimodal LLM ( \textsc{ChatGPT-5 OpenAI API} with \texttt{reasoning\_effort = low}.)과 multimodal embedder (\textsc{MM-Embed})를 사용한다.
% We use three multimodal embedders: \texttt{MM-Embed}~\cite{mmembed}, \texttt{UniME}~\cite{unime} and \texttt{mmE5}~\cite{mme5}.
% Details about the embedding models can be further found in \textsection~\ref{sec:appendix_model_details}.
% \noindent\textbf{Applied multimodal embedding models \& multimodal LLMs.}
% To ensure a controlled comparison, we use a single multimodal embedder and a single multimodal LLM backbone across our pipeline.
% We build a unified component index using \textsc{MM-Embed} as the multimodal embedding model for similarity search.
% For all LLM-driven modules (e.g., decomposition/planning, traversal evaluation, and answer generation), we use the \textsc{ChatGPT-5 OpenAI API} with \texttt{reasoning\_effort = low}.
% 위 부분도 담백하게 표현
% (예시: We use three multimodal embedders: \texttt{MM-Embed}~\cite{mmembed}, \texttt{UniME}~\cite{unime} and \texttt{mmE5}~\cite{mme5}.
% Details about the embedding models can be further found in \textsection~\ref{sec:appendix_model_details}.)




% \subsection{Retrieval Accuracy Comparison}

% We evaluated retrieval accuracies using Recall@$k$ ($k \in \{1, 2, 5, 10\}$ and MRR@10 across three benchmarks, and its results are reported in Table~\ref{tab:retrieval_performance}.
% \textsc{\Ours} consistently yields the best performance across all datasets and cutoffs, demonstrating the benefit of casting multimodal subgraph discovery as a feedback-driven sequential traversal process.
% 구체적으로는, \textsc{LILaC} 대비 Recall@10, MRR@10 기준 22.03\%, 12.88\% 증가하였고, \textsc{IRCoT} 대비 Recall@10, MRR@10 기준 18.60\%, 9.31\% 증가하였다.
% \textsc{NV-Embed-v2}, \textsc{VisRAG}, \textsc{ColPali} 등의 embedding-based single-knn 기법들과는 더 극심하고 확연한 차이가 나는 것을 확인할 수 있다.

% Analysis를 통해 다음을 알 수 있다.
% \textsc{LILaC} 기준 \WebQA 데이터셋에서 Recall@10이 24.23\%까지 차이 남으로써 가장 차이가 많이 나는데, 이는 \WebQA 데이터셋은 생성 방식 상 ground truth component들이 붙어 있을 필요가 없어, global hop의 중요성이 대두되는 데이터셋이기 때문이다.
% 동시에 \textsc{IRCoT}는 graph 구조를 사용하지 않고, global hop만을 진행하기 때문에 \WebQA에서는 성능 차가 recall@10 8.23\%로 차이가 상대적으로 적은 것을 확인할 수 있다.
% 다만 component 끼리의 연결 구조를 잘 사용해야 하는 \MultimodalQA, \MMCoQA에서는 14.47\%, 16.33\%까지 차이 나는 것을 통해 이러한 연결 구조의 중요성을 파악할 수 있다.
% 또한, 
% 또한, Recall@$k$를 볼 때 $k$의 값이 커질수록 

\vspace{-1mm}
\section{Conclusion}
\vspace{-1mm}


We presented \texttt{LILaC}, a multimodal retrieval framework designed to address the limitations of existing methods by incorporating layered component graph and late-interaction-based subgraph retrieval.
Our layered graph construction explicitly captures semantic relationships among multimodal components, facilitating effective multihop reasoning. 
The late-interaction retrieval method dynamically evaluates fine-grained component relevance, significantly enhancing retrieval accuracy, yet efficient.
\updated{\texttt{LILaC}'s usage of pretrained multimodal encoders allows it to inherit the improvements from newer off-the-shelf embeddings.
Extensive experiments confirm that \texttt{LILaC} consistently outperforms state-of-the-art approaches across all five benchmarks, also demonstrating its broad applicability and effectiveness in open-domain multimodal retrieval.}



\newpage
\bibliography{main}
\bibliographystyle{icml2026}

%\clearpage
\appendix
\onecolumn


\section{Limitations}

% LLM을 사용하는 알고리즘의 특성 상, LILaC과 비교하였을 때 retrieval time이 오래 걸리는 문제가 있음.
% 또한, 우리는 document가 이미 component 형태로 parsing되어 있다는 것을 가정한다.
% 즉 이러한 parsing 및 preprocessing quality에 따라 우리의 accuracy가 결정된다.
% 향후로는 어떠한 webpage에도 적용 가능한 retrieval 방법론을 개발하고자 한다.
% \textbf{Latency and cost.}
\textsc{\Ours} incurs higher wall-clock latency and API cost than lightweight or fixed-call traversal methods such as \textsc{LILaC}.
Although our cost-aware strategy escalation mitigates unnecessary computation, multi-step orchestration and occasional LLM-based reranking remain as bottlenecks. %, limiting throughput in high-volume or real-time settings.
% \textbf{Dependence on preprocessing.}
Our method assumes that each webpage/document is already parsed into a clean set of multimodal components and that navigational signals are reliably extracted.
% In practice, parsing and preprocessing quality (e.g., component segmentation, table extraction, OCR for images, and link resolution) can vary widely across sources, and errors in this pipeline may propagate to retrieval decisions and reduce accuracy.
% \textbf{Scope of applicability.}
While we focus on URL-annotated corpora with structured navigation and component graphs, truly open-web deployment may involve noisier pages, weaker link signals, dynamic content, and heterogeneous layouts.
An important direction for future work is to develop retrieval pipelines that are robust to diverse webpage structures and can jointly learn or adapt the parsing, graph construction, and retrieval policy so that the approach generalizes more seamlessly to arbitrary web content.




\section{Model Details}
\label{sec:appendix_model_details}


\textbf{(Multimodal) Large language models:}
\squishlist
    \item \texttt{Open-AI ChatGPT5}
\squishend

\noindent \textbf{Text embedders}
\squishlist
    \item \texttt{NV-Embed-v2}: 7.85B parameters
\squishend

\noindent \textbf{Cross-modal embedders:}
\squishlist
    \item \texttt{ColPali}: 3B parameters
    \item \texttt{VisRAG}: 3.43B parameters
\squishend


\noindent \textbf{Multimodal embedders:}
\squishlist
    \item \texttt{MM-Embed}: 8.18B parameters
\squishend






\section{Experimental Details}
\label{appendix:exp_details}




\subsection{Hardware and Software Settings}
\label{appendix:hardware_software_settings}
All experiments were conducted on a Linux server equipped with an Intel Xeon Gold 6230 CPU @ 2.10~GHz, 1~TB of RAM, and four NVIDIA RTX~A6000 GPUs, running Ubuntu~22.04.3~LTS. 


\subsection{Implementation Details}
\label{sec:implementation_details}

Our main hyperparameter is the vector-search shortlist size $k$ used by the Multi-strategy Traverser at both the document-level and component-level stages.
Unless otherwise noted, we set $k{=}30$ for all experiments.


\subsection{Benchmark Details}


\MultimodalQA: We use the extended version of \texttt{MultimodalQA}, following the augmentation procedure introduced in M3DocRAG~\cite{m3docrag}. 
It spans diverse document modalities (text, images, and tables) and is designed to stress multi-hop reasoning over multiple documents. 
The evaluation split contains 2,441 questions grounded in 3235 webpages.

\MMCoQA: We use an extended variant of \texttt{MMCoQA} that moves beyond the original distractor-only setting to evaluate conversational, multi-turn multimodal QA. 
The benchmark consists of coherent dialogue sessions in which later questions depend on earlier context and require aggregating evidence across text, images, and tables. 
It includes 5,753 questions grouped into 1,179 conversations, with a corpus of 218,285 text passages, 10,042 tables, and 57,058 images.

\WebQA:
\WebQA\ is a Wikipedia-based multimodal QA benchmark with 4,966 questions over 7,662 documents.
Because the original answers are often verbose, we rewrite them into concise references using the ChatGPT-5 OpenAI API with the prompt below.



\begin{promptbox}{Answer Concisification Prompt for \WebQA}
You are an assistant that extracts concise answers from an Original Answer.

Task:
Given a Question, its Question Category (Qcate), and an Original Answer, extract a concise version of the answer.

Category hints:
- YesNo: respond only "Yes" or "No" matching the polarity of the Original Answer.
- text: return the minimal noun phrase/name that answers the question.
- choose: return only the chosen option or label.
- number: return the numeric value (and unit if present) without extra words.
- color: return the color term(s) only.
- shape: return the shape descriptor only.
- Others: follow the general concise rules below.

Rules:
- Output ONLY the concise answer text (no extra words, no labels, no punctuation-only output).
- Keep the minimum span that directly answers the Question.
- Prefer a single word when possible.
- If the question asks what object/thing, output the object noun phrase only (e.g., fountain).
- If the question asks for a choice/comparison attribute (e.g., taller or shorter, happy or upset, or similar), output only the chosen option word from the answer (e.g., "taller", "upset").
- If the Original Answer is verbose by repeating or paraphrasing words/phrases already present in the Question, do NOT copy those repeated Question words into the concise answer; extract only the new, directly-question-answering information (If those repeated words are necessary for answering the question, then you may include them).
- Preserve the original casing/pluralization as used in the Original Answer (e.g., "Circles").
- Do not include locations, explanations, or surrounding context unless they are required to uniquely answer the question.

Examples:
# Example 1
Question Category: YesNo
Question: Does a Minnetonka Rhododendron flower have petals in a cup shape?
Original Answer: No, a Minnetonka Rhododendron flower does not have petals in a cup shape.
Concise Answer: No

# Example 2
Question Category: Others
Question: What water-related object is sitting in front of the Torre del Reloj?
Original Answer: A fountain is sitting in front of the Torre del Reloj.
Concise Answer: fountain

# Example 3
Question Category: choose
Question: Is the fence in front of The Glass House in Fulham taller or shorter than a bicycle?
Original Answer: The fence in front of the building is taller than a typical bicycle.
Concise Answer: taller

# Example 4
Question Category: shape
Question: What shape is found 3 times on the front of the Archway in King Charles Street?
Original Answer: Circles may be spotted three times on the face of the Archway on King Charles Street.
Concise Answer: Circles

# Example 5
Question Category: choose
Question: Does the character in the work \"Beslotentuinfeest\" look happy or upset?
Original Answer: The character in the work \"Beslotentuinfeest\" looks upset.
Concise Answer: upset

# Example 6
Question Category: number
Question: How many more skis were used by Anders S\u00f6dergren during the 2010 Olympics than were used by Martin Rulsch during the 2020 Winter Youth Olympics?
Original Answer: Anders S\u00f6dergren used two more skis during the 2010 Olympics.
Concise Answer: two

Inputs:
Question Category (Qcate): {qcate}
Question: {question}
Original Answer: {answer}
Concise Answer:
\end{promptbox}



\section{Prompts Templates of \textsc{\OurFullName}}

% \begin{tcolorbox}
% [breakable, title = Prompt Used to Generate Long Textual Description for Hard Prompt Methods, colback = gray!10, colframe = black, sharp corners, boxrule=0.5mm]
% Describe only the {object} in the image. Focus on its most distinctive features such as shape, color, and material. Write the result in English as a short noun phrase, not a full sentence. Do not mention orientation, position, or surroundings.
% \end{tcolorbox}

% \begin{tcolorbox}
% [breakable, title = Prompt Used to Generate Short Textual Description for Hard Prompt Methods, colback = gray!10, colframe = black, sharp corners, boxrule=0.5mm]
% Describe only the {object} in the image. Write the result in English as a single, highly detailed noun phrase, not a full sentence. Include as many intrinsic and identifying features as possible, such as overall shape, dimensions, proportions, material, surface finish, textures, patterns, colors, edges, rims, openings, and any decorative or structural details. Do not mention orientation, position, or surroundings.
% \end{tcolorbox}






\begin{promptbox}{Orchestrator}
You are a retrieval action-decider assistant.

Task:
Given a user query, a retrieval plan, serialized retrieval memory from prior steps, and the titles of neighbor documents from the most recent step, decide the next retrieval action.

You must output EXACTLY ONE action describing:
- what action to perform
   - stop retrieving because there is plenty of information
   - search for a new piece of information
   - replan the subtasks because the plan is stale, misaligned, or incomplete


# Decision requirements:
1) Use ONLY provided context
   - You MUST use ONLY information directly inferable from:
     (a) the user query,
     (b) the initial plan,
     (c) the serialized retrieval memory,
     (d) the neighbor docs list.
   - Do NOT add facts, assumptions, background knowledge, or outside context.

2) No clarifying questions
   - You are not allowed to ask the user for clarification.
   - Make the best decision using only the given inputs.

3) Internal reasoning only (if needed)
   - Perform your analysis internally (do NOT output reasoning), and perform it only in cases where analysis is necessary.
   - Output JSON only no explanations or extra text.


# Rule Regarding Next Retrieval Subtask: 
  - If the latest retrieval was marked "answerable" OR the previous action used "llm reasoning" for BOTH `document_search_mode` and `component_search_mode`, you should advance to the next unresolved subtask when selecting `next_retrieval_subtask`.
  - Otherwise, keep targeting the current subtask (or the earliest unresolved subtask if none is explicitly active).
  - When advancing, prefer the earliest unresolved subtask in the list; if none remain unresolved, continue with the last subtask that still needs information or reuse the most recent unresolved one.


# Rule Regarding Dynamic Cost-Aware Strategy Escalation:
  - Treat retrieval configuration as a cost ladder:
    neighbors < vector search < llm reasoning
    and granularity: document < component < subcomponent.
  - Default (cheapest) for a new/clean subtask:
    document_search_mode="neighbors" (if Neighbor Docs likely contain missing info),
    else document_search_mode="vector search";
    component_search_mode="vector search";
    vector_granularity="document";
    anchor=null unless a clearly relevant prior candidate set exists.
  - Escalate ONLY when justified by evidence in Serialized Retrieval Memory:
    * current subtask 's latest attempt is marked Failure/unanswerable, OR
    * repeated near-misses (retrieved content is close but misses a constraint), OR
    * the hop is ambiguous/underspecified per memory (multiple plausible targets / unclear referent), OR
    * the prior attempt already used low-cost modes and did not progress.
  - Escalation policy (monotonic within the same anchor unless you backtrack or replan):
    1) neighbors + vector search + document granularity
    2) vector search (global over docs) + vector search components + document granularity
    3) vector search with vector_granularity="component"
    4) component_search_mode="llm reasoning" (keep doc mode as-is)
    5) document_search_mode="llm reasoning" AND component_search_mode="llm reasoning" (highest cost)
  - Scope widening rule (local -> global):
    If Neighbor Docs are irrelevant OR the same neighborhood fails twice, switch document_search_mode
    from "neighbors" to "vector search" (or "llm reasoning" if ambiguity persists).

# Rule Regarding History-Aware Backtracking (Failure-is-Feedback):
  - Use failure traces in Serialized Retrieval Memory as first-class signals (do not ignore them).
  - Define a "failed routing pattern" as repeating essentially the same route:
    same subtask intent + same (or null) anchor + same document_search_mode/component_search_mode/granularity,
    where the memory marks Failure/unanswerable or shows no new evidence gained.
  - If a failed routing pattern exists for the current subtask, you MUST change at least one of:
    (i) anchor, (ii) document_search_mode (scope), (iii) component_search_mode, (iv) vector_granularity,
    or (v) rewrite next_retrieval_subtask to add missing constraints / choose a different target entity.
  - Backtracking triggers:
    * >=2 consecutive failures on the current subtask, OR
    * the previous attempt already used ("llm reasoning","llm reasoning") and still failed, OR
    * Neighbor Docs list is exhausted/irrelevant for the missing information.
  - Backtracking procedure (re-anchoring):
    1) Prefer re-anchoring to the most recent Success (or best partial/near-success) step in memory:
      set anchor to that step index so downstream retrieval starts from its candidate documents.
    2) If multiple candidate anchors exist, prefer the one whose retrieved evidence best matches
      the unresolved constraint(s) of the current subtask (as described in memory).
    3) After re-anchoring, prefer document_search_mode="neighbors" if the missing info is likely
      adjacent to that anchor context; otherwise use "vector search"/"llm reasoning" to escape the neighborhood.
  - Backtracking MUST revise the next_retrieval_subtask to incorporate lessons from failures:
    explicitly negate dead ends, add missing constraints, or select the next-most-likely candidate entity.
  - Replan is reserved for plan-level problems:
    choose "replan" only if subtasks are stale/misaligned, or if backtracking across >=2 distinct anchors
    still fails to make progress.



# Action Fields

(1) "stop"
   - Meaning: Retrieval memory already contains sufficient components to answer the original query.

Return schema: 
{{
  "action": {{
    "next_action": "stop",
    "next_retrieval_subtask": null,
    "document_search_mode": null,
    "component_search_mode": null
  }}
}}

(2) "search"
   - Meaning: More retrieval is needed to answer the original query.
   - Fields to specify:
   
   a) next_retrieval_subtask (string)
      - Meaning: The next concrete retrieval task to run.
      - Must be a short, actionable retrieval prompt (imperative verb + object).
      - Must be consistent with the user query and the initial plan.
      - Must be chosen to address what is still missing or failed, as indicated by the serialized retrieval memory.
      - De-contextualize: avoid pronouns; restate the entity/target explicitly.
      - Generate a retrieval task that tries to solve either the first `Target Subtask to Solve`, or combinations of them. Note that we are retrieving one of paragraph/table/image components - generate a retrieval task that most suits this granularity of retrieval.
      - [IMPORTANT] If the `User Query` is about finding a single entity that meets a certain condition, and `Target Subtask to Solve` indicates retrieving multiple entities then choosing one that meets a certain condition, 
         try using common sense to pick the most likely entity that meets the condition and generate a retrieval task for that entity only.
         * If the memory suggests that finding one entity failed, then indicate it as that such entity does not exist, and then try to pick the second most likely entity that meets the condition.

   b) document_search_mode (one of: "neighbors" | "vector search" | "llm reasoning")
      - Meaning: How to select which documents to look at next (before selecting components inside them).
      - "neighbors":
         Use when the neighbor documents list (given as `# Neighbor Docs`) is likely sufficient for the next_retrieval_subtask.
         Choose this if the subtask is a direct continuation of the last step and the missing info is likely in adjacent/related documents.
      - "vector search":
         Use when neighbor docs (given as `# Neighbor Docs`) are likely insufficient and you need to search across the entire corpus.
         This performs a vector search over all documents (fast but potentially less accurate).
         Choose this when the subtask is relatively specific/unambiguous and broad recall is needed.
      - "llm reasoning":
         Use when neighbor docs are likely insufficient AND the next_retrieval_subtask is ambiguous, underspecified, or requires careful disambiguation.
         This performs an initial vector search to shortlist documents, then uses LLM reasoning to choose the best document(s) (slower but more accurate).

   c) component_search_mode (one of: "vector search" | "llm reasoning")
      - Meaning: How to select components (paragraphs, tables, images) within the chosen documents.
      - "vector search":
          Use when the subtask target is fairly specific and likely to match component embeddings directly (fast but less accurate).
      - "llm reasoning":
         Use when the subtask is ambiguous, requires multi-constraint matching, or prior attempts show that pure vector search returns near-misses.
         This filters components using vector search first, then uses LLM reasoning to pick the most relevant components (slower but more accurate).

   d) anchor (integer or null)
      - Meaning: If provided, reuse candidate documents from the memory step at this 0-based index.
      - How to use:
        * anchor refers to the memory's retrieval steps list index.
        * If anchor is provided, downstream retrieval will start from the candidate documents of that step's last attempt.
      - When to set:
        * Use when a previous step already surfaced a good candidate set that should be reused/refined.
        * Otherwise set to null.

   e) vector_granularity (one of: "document" | "component" | "subcomponent")
      - Meaning: The granularity at which vector search should be applied for the next retrieval step.
      - How to choose:
        * "document": do vector search over documents, then pick components within them (default).
        * "component": do vector search directly over components.
        * "subcomponent": do vector search at a finer granularity (e.g., sentences/snippets) when component-level recall is too coarse.

Return schema:
{{
  "action": {{
    "next_action": "search",
    "next_retrieval_subtask": str,
    "document_search_mode": "neighbors" | "vector search" | "llm reasoning",
    "component_search_mode": "vector search" | "llm reasoning",
    "anchor": int | null,
    "vector_granularity": "document" | "component" | "subcomponent"
  }}
}}

(3) "replan"
   - Meaning: The current retrieval plan is inadequate or misaligned; produce a refreshed set of subtasks before continuing.
   - Choose this when: target subtasks are exhausted, clearly off-target, missing necessary steps, or memory shows repeated failures that imply the plan is wrong.
   - No retrieval is executed in this step; the system will run the replan tool using the existing context.

Return schema:
{{
  "action": {{
    "next_action": "replan",
    "next_retrieval_subtask": null,
    "document_search_mode": null,
    "component_search_mode": null
  }}
}}


Output format (strict):
Return ONLY valid JSON matching exactly one of the following schemas (no markdown, no extra text):
{{
  "action": {{
    "next_action": "stop",
    "next_retrieval_subtask": null,
    "document_search_mode": null,
    "component_search_mode": null
  }}
}}
{{
  "action": {{
    "next_action": "search",
    "next_retrieval_subtask": str,
    "document_search_mode": "neighbors" | "vector search" | "llm reasoning",
    "component_search_mode": "vector search" | "llm reasoning",
    "anchor": int | null
  }}
}}
{{
  "action": {{
    "next_action": "replan",
    "next_retrieval_subtask": null,
    "document_search_mode": null,
    "component_search_mode": null
  }}
}}



Inputs:
# User Query
{query}

# Split Retrieval Subtasks (answerability status and answers also)
{serialized_subtasks}

# Serialized Retrieval Memory (what has been tried + what succeeded/failed + what is still missing)
{serialized_memory}

# Neighbor Docs (titles of documents neighboring the last retrieved documents)
{neighbor_docs}

Output:
\end{promptbox}




\begin{promptbox}{Document-level Traverser}
You are a retrieval selection assistant.

Task:
Given (1) an original user query, (2) a subtask query, and (3) a list of candidate documents, select which candidate documents should be retrieved next.
Vector granularity (document | component | subcomponent) is provided to signal the intended search granularity; prioritize candidates that best match the subtask at that granularity.

Objective:
Select up to {max_results} candidate documents that are most directly useful for fulfilling BOTH:
- the Original Query, and
- the Subtask Query (the immediate retrieval goal).

Hard constraints:
1) Use ONLY provided context
   - You MUST use ONLY information directly inferable from:
     (a) the Original Query,
     (b) the Subtask Query,
     (c) the Candidate documents list.
   - Do NOT add facts, assumptions, background knowledge, or external context.
   - Do NOT introduce any documents that are not in the candidate list.

2) No clarifying questions
   - You are not allowed to ask the user for clarification.
   - Make the best selection using only the given inputs.

3) Validity + exact matching
   - Indices must be valid 0-based indices into the candidate list.
   - Filenames must exactly match a filename from the candidate list.

4) Internal reasoning only
   - Perform analysis internally (do NOT output reasoning).
   - Output JSON only no explanations or extra text.

Scoring rules (for "score"):
- Assign a relevance score in [0.0, 1.0] for each selected document.
- Scores should reflect *direct usefulness* for retrieval to answer BOTH queries:
  - 1.0: highly likely to contain the needed information/evidence for the subtask while staying aligned with the original query
  - 0.5: partially relevant (may help, but incomplete or tangential)
  - 0.0: clearly irrelevant
- Prefer documents that strongly match the Subtask Query, but do not pick documents that obviously diverge from the Original Query 's scope/constraints.

Selection & ordering rules:
- Output ONLY up to {max_results} selections (or fewer if fewer are relevant).
- Do NOT include candidates outside the top selections.
- Sort selections by descending relevance score (highest first).
- If unsure, pick the most likely candidates based on the queries.
- Never return an empty selection unless nothing is relevant at all (i.e., all candidates are clearly irrelevant).

Output format (strict):
Return ONLY valid JSON matching exactly this schema (no markdown, no extra text):
{{
  "selection": [
    {{
      "index": int,           // 0-based index matching the candidate list
      "filename": string,     // exact filename from the candidate list
      "score": float          // optional relevance score (0.0-1.0)
    }}
  ]
}}

Inputs:
# Original Query
{original_query}

# Subtask Query
{subtask_query}

# Vector granularity
{vector_granularity}

# Candidate documents (0-based indices):
{candidates}

# Max results
{max_results}

Output:
\end{promptbox}


\begin{promptbox}{Component-level Traverser}
You are a component selection assistant for retrieval.

Task:
Given (1) an original user query, (2) a subtask query, and (3) a list of candidate components, select which components should be kept for the next step.
Vector granularity (document | component | subcomponent) is provided to signal the intended search granularity; prioritize components that best match the subtask at that granularity.

Objective:
Select up to {max_results} candidate components that are most directly useful for fulfilling BOTH:
- the Original Query, and
- the Subtask Query (the immediate retrieval goal).

Hard constraints:
1) Use ONLY provided context
   - You MUST use ONLY information directly inferable from:
     (a) the Original Query,
     (b) the Subtask Query,
     (c) the Candidate components list.
   - Do NOT add facts, assumptions, background knowledge, or external context.
   - Do NOT invent or introduce any components that are not in the candidate list.

2) No clarifying questions
   - You are not allowed to ask the user for clarification.
   - Make the best selection using only the given inputs.

3) Validity + exact matching
   - Indices must be valid 0-based indices into the candidate list.
   - Filenames and component_ids must exactly match a candidate entry.

4) Internal reasoning only
   - Perform analysis internally (do NOT output reasoning).
   - Output JSON only no explanations or extra text.

Scoring rules (for "score"):
- Assign a relevance score in [0.0, 1.0] for each selected component.
- Scores should reflect *direct usefulness* for retrieval to answer BOTH queries:
  - 1.0: highly likely to contain the needed information/evidence for the subtask while staying aligned with the original query
  - 0.5: partially relevant (may help, but incomplete or tangential)
  - 0.0: clearly irrelevant
- Prefer components that strongly match the Subtask Query, but do not pick components that obviously diverge from the Original Query 's scope/constraints.
- If useful information may be distributed across multiple components, include multiple complementary components when within {max_results}.

Selection & ordering rules:
- Output ONLY up to {max_results} selections (or fewer if fewer are relevant).
- Do NOT include candidates outside the selected set.
- Sort selections by descending relevance score (highest first).
- If unsure, pick the most likely candidates based on the queries.
- Avoid empty outputs unless nothing is relevant at all (i.e., all candidates are clearly irrelevant).

Output format (strict):
Return ONLY valid JSON matching exactly this schema (no markdown, no extra text):
{{
  "selection": [
    {{
      "index": int,           // 0-based index matching the candidate list
      "filename": string,     // exact filename from the candidate list
      "component_id": string, // exact component_id from the candidate list
      "score": float          // optional relevance score (0.0-1.0)
    }}
  ]
}}

Inputs:
# Original Query
{original_query}

# Subtask Query
{subtask_query}

# Vector granularity
{vector_granularity}

# Candidate components (0-based indices):
{candidates}

# Max results
{max_results}

Output:
\end{promptbox}

\begin{promptbox}{Subquery Planner}

You are a retrieval-plan revision assistant.

Task:
Given (1) an original user query and (2) serialized retrieval memory from a failed attempt, produce 1-2 concrete retrieval tasks (strings) that should be attempted next to gather the missing information needed to answer the original query.

Rules:
1) Task count
   - If the remaining gap is small and single-scope, output exactly 1 task.
   - Otherwise output 2 tasks.

2) Task wording
   - Each task must be a short, **actionable retrieval prompt** (imperative verb + object).
   - Do not put reasoning (logical derivation) as a task. Such reasoning should be done along with corresponding retrieval.
   - Keep tasks focused: ideally one clear information need per task.

3) Use ONLY provided inputs
   - You MUST use ONLY information directly inferable from:
     (a) the Original Query, and
     (b) the Retrieval Memory.
   - Do NOT add facts, assumptions, background, or likely details not present or directly implied by the inputs.
   - Do NOT rely on outside knowledge.

4) Entity & noun-phrase coverage
   - Every named entity and key noun phrase from the Original Query that is still needed must appear at least once across the tasks (you may distribute them).
   - Use the Retrieval Memory to identify which entities/noun phrases are missing, underspecified, or unresolved.

5) De-contextualize
   - Replace pronouns and implicit references so each task is understandable standalone.
   - Avoid it/they/this; restate the referenced entity.
   - If the Retrieval Memory introduces placeholders or intermediate identifiers, restate them explicitly.

6) Constraint distribution
   - Spread constraints logically across tasks instead of cramming everything into one task.
   - Prefer separating: identification/disambiguation vs. evidence gathering vs. attribute extraction.

7) Dependency ordering
   - If one task's output is needed for another, put the prerequisite first.
   - Express the dependency explicitly in the task description using (i), (ii), (iii), etc.
   - When a later task depends on earlier results, explicitly reference them by replacing (i) with the specific entity name once known (i.e., keep the "(i)" placeholder until resolved).

Output format (strict):
Return ONLY valid JSON with this schema and nothing else:
{{
  "tasks": ["string", ...]
}}

Examples:



Inputs:
# Original Query
{original_query}

# Retrieval Memory
{memory}

Output:
\end{promptbox}


\begin{promptbox}{Traversal Evaluator}

You are a retrieval evaluation assistant.

Task:
Given an Original Query, a target Subtask Query, a list of all Subtasks (with their current statuses, if any), and the Retrieved components, decide whether retrieval succeeded for the target Subtask Query and which subtasks are answerable using the current components.

Objective:
Evaluate whether the retrieved components contain enough clearly relevant evidence to answer the target Subtask Query (or to progress directly to answering it), while remaining consistent with the Original Query. Also identify any other subtasks that are answerable from the same retrieved components.

Hard constraints:
1) Use ONLY provided context
   - You MUST use ONLY information directly inferable from:
     (a) the Original Query,
     (b) the target Subtask Query,
     (c) the Subtasks list, and
     (d) the Retrieved components list.
   - Do NOT add facts, assumptions, background knowledge, or external context.
   - Do NOT assume missing details (including what unseen documents might contain).

2) No clarifying questions
   - You are not allowed to ask the user for clarification.
   - Make the best determination using only the given inputs.

3) Internal reasoning only
   - Perform analysis internally (do NOT output reasoning).
   - Output JSON only no explanations or extra text.

# Decision rules for `status`:
- Output "answerable" if at least one retrieved component appears:
  (i) clearly relevant to the given `# Subtask Query`, OR
  (ii) directly answers the `# Original query`, if combined with the `# Subtask status`, OR
  (iii) directly answers any of the `# Subtask query` that are marked as "not answerable" or "unknown" in the `# Subtask status` list, , if combined with the `# Subtask status`.
- Otherwise output "not answerable".

# Updated subtasks (independent of target status):
Using the `#Retrieved components` and `# Subtask status`, scan the provided Subtask status list and identify which subtasks are answerable now.
- "updated_subtasks" MUST include ONLY the unanswerable subtasks from the `# Subtask status` that have become answerable using the current Retrieved components + subtask status.
- For each included subtask that are not answerable:
  - If it is answerable using `Retrieved components + subtask status`, then
    - Provide its 1-based index from the provided Subtask status list.
    - Set status = "answerable".
    - Provide a concise answer drawn ONLY from the Retrieved components + subtask status.
  - Else skip it.
- If no subtasks are answerable, output an empty list [].


Output format (strict):
Return ONLY valid JSON matching exactly this schema (no markdown, no extra text):
{{
  "status": "answerable" | "not answerable",
  "updated_subtasks": [
    {{
      "index": int,                  // 1-based index of the subtask from the provided list
      "status": "answerable",
      "answer": string               // required; answer derived only from Retrieved components
    }}
  ]
}}

Inputs:
# Original Query
{original_query}

# Subtask Query
{subtask_query}

# Retrieved components (0-based indices):
{candidates}

# Subtask status (1-based index; include current status if any)
{subtasks}

Output:     
\end{promptbox}


\begin{promptbox}{Reranker}
You are a reranking assistant for retrieval.

Task:
Given a user query and a list of candidate components, select and rank the TOP-{top_k} candidates by how directly useful they are for answering the user query.

Hard constraints:
1) Use only provided context
   - You MUST use ONLY information directly inferable from:
     (a) the user query, and
     (b) the candidate components list.
   - Do NOT add facts, assumptions, background knowledge, or external context.
   - Do NOT complete missing details with parametric knowledge.

2) No clarifying questions
   - You are not allowed to ask the user for clarification.
   - Make the best ranking using only the given inputs.

3) Internal reasoning only
   - Perform an internal step-by-step analysis before finalizing scores (do NOT output the reasoning).
   - During reasoning, consider semantic overlap, specificity to the query, and usefulness for answering.

Scoring rules:
- Assign a relevance score in [0.0, 1.0] to selected candidates.
- Scores should reflect *direct* usefulness for answering the query:
  - 1.0: highly likely to contain the needed answer or the most relevant evidence
  - 0.5: partially relevant or tangentially useful
  - 0.0: clearly irrelevant
- Relevant information may be distributed across multiple components, such as components that are needed to resolve a query
  - Including identifying, disambiguating, or mapping implicit entities (e.g., aliases, acronyms, related identifiers) that are not explicitly stated in the query but are required to answer it.
  - Give appropriate credit to partially query-relevant candidates that could contribute necessary pieces of the answer, even if they are not sufficient on their own.
- If multiple candidates seem similarly relevant, prefer the ones that more directly match the query's key entities/constraints.

Selection & ordering rules:
- Output ONLY the TOP-{top_k} candidates by relevance (or fewer if fewer than {top_k} candidates are provided).
- Do NOT include candidates outside the top-{top_k}, even if they are mildly relevant.
- Sort the output by descending score (highest first).
- Indices are 0-based and must match the candidate list.

Output format (strict):
Return ONLY valid JSON matching exactly this schema (no markdown, no extra text):
{{
  "ranking": [
    {{
      "index": int,           // 0-based index matching the candidate list
      "filename": string,     // exact filename from the candidate object
      "component_id": string, // exact component_id from the candidate object
      "score": float          // relevance score (0.0-1.0)
    }}
  ]
}}


Example:

Input:
User Query: "How does the payment processing component handle errors?"
Top-K: 2
Candidates:
  0: {{"filename": "billing.json", "component_id": "po_0"}}
  1: {{"filename": "auth.json", "component_id": "t_1"}}
  2: {{"filename": "billing.json", "component_id": "i_10"}}

Output:
{{
  "ranking": [
    {{"index": 0, "filename": "billing.json", "component_id": "po_0", "score": 0.93}},
    {{"index": 2, "filename": "billing.json", "component_id": "i_10", "score": 0.71}}
  ]
}}

REMINDER:
Your objective is to read the user query, reason internally about each candidate's relevance using ONLY the provided inputs, select the TOP-{top_k} candidates, assign scores, sort them by descending score, and output only the defined JSON.

Inputs:
# User Query
{query}

# Candidate Components (indices are 0-based; refer to them as "index"):
{candidates}

# Top-K
{top_k}

Output:
\end{promptbox}


\end{document}





% This document was modified from the file originally made available by
% Pat Langley and Andrea Danyluk for ICML-2K. This version was created
% by Iain Murray in 2018, and modified by Alexandre Bouchard in
% 2019 and 2021 and by Csaba Szepesvari, Gang Niu and Sivan Sabato in 2022.
% Modified again in 2023 and 2024 by Sivan Sabato and Jonathan Scarlett.
% Previous contributors include Dan Roy, Lise Getoor and Tobias
% Scheffer, which was slightly modified from the 2010 version by
% Thorsten Joachims & Johannes Fuernkranz, slightly modified from the
% 2009 version by Kiri Wagstaff and Sam Roweis's 2008 version, which is
% slightly modified from Prasad Tadepalli's 2007 version which is a
% lightly changed version of the previous year's version by Andrew
% Moore, which was in turn edited from those of Kristian Kersting and
% Codrina Lauth. Alex Smola contributed to the algorithmic style files.
