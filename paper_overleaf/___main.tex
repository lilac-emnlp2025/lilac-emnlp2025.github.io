% This must be in the first 5 lines to tell arXiv to use pdfLaTeX, which is strongly recommended.
\pdfoutput=1
% In particular, the hyperref package requires pdfLaTeX in order to break URLs across lines.

\documentclass[11pt]{article}

% Change "review" to "final" to generate the final (sometimes called camera-ready) version.
% Change to "preprint" to generate a non-anonymous version with page numbers.
\usepackage[final]{acl}

% Standard package includes
\usepackage{times}
\usepackage{latexsym}

% For proper rendering and hyphenation of words containing Latin characters (including in bib files)
\usepackage[T1]{fontenc}
% For Vietnamese characters
% \usepackage[T5]{fontenc}
% See https://www.latex-project.org/help/documentation/encguide.pdf for other character sets

% This assumes your files are encoded as UTF8
\usepackage[utf8]{inputenc}

% This is not strictly necessary, and may be commented out,
% but it will improve the layout of the manuscript,
% and will typically save some space.
\usepackage{microtype}

% This is also not strictly necessary, and may be commented out.
% However, it will improve the aesthetics of text in
% the typewriter font.
\usepackage{inconsolata}

%Including images in your LaTeX document requires adding
%additional package(s)
\usepackage{graphicx}


\usepackage{enumitem}
\usepackage{listings}
\usepackage{amssymb}
\usepackage{wrapfig}
\usepackage{booktabs}
\usepackage{amsmath}
\usepackage{tabularx}
\usepackage{adjustbox}
\usepackage{amsthm}
\newtheorem{definition}{Definition}
\usepackage{caption}
\usepackage[ruled,vlined,noend,linesnumbered]{algorithm2e}


%%%%%%%%%%%%%%%%%%%%%%%%%%%%%%%%%%%%%%%%%%%%%%%%%%%%%%%%%
%%%%%%%%%%%%%%%%%% User-added packages %%%%%%%%%%%%%%%%%%
%%%%%%%%%%%%%%%%%%%%%%%%%%%%%%%%%%%%%%%%%%%%%%%%%%%%%%%%%

\usepackage{kotex}

% put these in your preamble once
\usepackage{multirow}   % \multirow
\newcommand{\yes}{\checkmark}     % ✓  (requires amssymb or pifont if you prefer)
\newcommand{\no}{\phantom{\checkmark}} % keep cell width when blank
\usepackage[table]{xcolor}   % already in most NeurIPS templates
\usepackage{colortbl}        % if not loaded by another package

%%%%%%%%%%%%%%%%%%%%%%%%%%%%%%%%%%%%%%%%%%%%%%%%%%%%%%%%%
%%%%%%%%%%%%%%%%% User-defined commands %%%%%%%%%%%%%%%%%
%%%%%%%%%%%%%%%%%%%%%%%%%%%%%%%%%%%%%%%%%%%%%%%%%%%%%%%%%

\newcommand{\customcirc}{\raisebox{-0.6mm}{\scalebox{1.6}{\(\circ\)}}}
\newcommand*\circled[1]{\tikz[baseline=(char.base)]{
            \node[shape=circle,draw,inner sep=0.5pt] (char) {#1};}}
            
\newcommand{\squishlist}{
   \begin{list}{$\bullet$}
    { \setlength{\itemsep}{1pt}
      \setlength{\parsep}{0pt}
      \setlength{\topsep}{2pt}
      \setlength{\partopsep}{0pt}
      \setlength{\listparindent}{-2pt}
      \setlength{\itemindent}{-5pt}
      \setlength{\leftmargin}{1.5em}
      \setlength{\labelwidth}{0em}
      \setlength{\labelsep}{0.5em} 
    } 
}

\newcommand{\squishend}{
    \end{list}  }

% --- Preamble additions ---
\usepackage{tabularx, array}
\newcolumntype{L}[1]{>{\raggedright\arraybackslash}p{#1}}
\newcolumntype{R}[1]{>{\raggedleft\arraybackslash}p{#1}}

% Distinct labels for Q/A (sans-serif + bold)
\newcommand{\Qlabel}{\textsf{\textbf{Question:}}}
\newcommand{\Alabel}{\textsf{\textbf{Answer:}}}
\usepackage{calc} % for width arithmetic

% ---- Text-table column setup (no tabular used) ----
\newlength{\Sep}\setlength{\Sep}{0.5em} % spacing around pipes
\newlength{\ColA}\newlength{\ColB}\newlength{\ColC}\newlength{\ColD}
% Choose A,B,D; C is computed to exactly fit \linewidth minus separators
\setlength{\ColA}{0.50\linewidth}
\setlength{\ColB}{0.30\linewidth}
\setlength{\ColC}{1.6\linewidth-\ColA-\ColB-\ColD-6\Sep} % 3 separators, each has left+right \Sep
\setlength{\ColD}{0.25\linewidth}


% Row macro: four \parbox columns with pipes between; last column right-aligned
% Cell helpers (top-aligned, no extra vertical glue)
\newcommand{\CellBox}[2]{%
  \begin{minipage}[t]{#1}\raggedright\setlength{\parskip}{0pt}\strut #2\strut\end{minipage}%
}
\newcommand{\CellBoxR}[2]{%
  \begin{minipage}[t]{#1}\raggedleft\setlength{\parskip}{0pt}\strut #2\strut\end{minipage}%
}
\newcommand{\TextRow}[4]{%
  \noindent
  \CellBox{\ColA}{#1}%
  \hspace{\Sep}\texttt{|}\hspace{\Sep}%
  \CellBox{\ColB}{#2}%
  \hspace{\Sep}\texttt{|}\hspace{\Sep}%
  \CellBox{\ColC}{#3}%
  \hspace{\Sep}\texttt{|}\hspace{\Sep}%
  \CellBox{\ColD}{#4}\par
}
% Header row macro (bold labels)
\newcommand{\TextHeader}{%
  \noindent\textbf{
    \CellBox{\ColA}{Deployment}%
    \hspace{\Sep}\texttt{|}\hspace{\Sep}%
    \CellBox{\ColB}{Organization}%
    \hspace{\Sep}\texttt{|}\hspace{\Sep}%
    \CellBox{\ColC}{Operation}%
    \hspace{\Sep}\texttt{|}\hspace{\Sep}%
    \CellBox{\ColD}{Personnel}}%
  \par
}
\definecolor{linkpink}{HTML}{E91E63}
% \usepackage[colorlinks = true, urlcolor=linkpink]{hyperref}
\hypersetup{
  colorlinks=true,
  urlcolor=linkpink,
  linkcolor=linkpink,
  citecolor=linkpink
}
\usepackage[most]{tcolorbox} % 뱃지용
\usepackage{fontawesome}     % GitHub 아이콘

% 인라인 배지 스타일
\newtcbox{\codebadge}{
  on line,
  arc=2.2pt,
  colback=linkpink!8,
  colframe=linkpink,
  colupper=linkpink,
  boxrule=0.4pt,
  left=6pt,right=6pt,top=2pt,bottom=2pt
}


%%%%%%%%%%%%%%%%%%%%%%%%%%%%%%%%%%%%%%%%%%%%%%%%%%%%%%%%%
%%%%%%%%%%%%%%%%%%%%%%%%%%%%%%%%%%%%%%%%%%%%%%%%%%%%%%%%%
%%%%%%%%%%%%%%%%%%%%%%%%%%%%%%%%%%%%%%%%%%%%%%%%%%%%%%%%%

\newcommand{\updated}[1]{{\textcolor{black}{#1}}}
\newcommand{\fillin}[1]{{{{\textcolor{black}{#1}}}}}
\newcommand{\todo}[1]{{\textcolor{black}{#1}}}
\newcommand{\wshan}[1]{{{\textcolor{black}{#1}}}}
\newcommand{\dylee}[1]{{{{\textcolor{black}{#1}}}}}
\newcommand{\jhyun}[1]{{\textcolor{black}{#1}}}




% If the title and author information does not fit in the area allocated, uncomment the following
%
%\setlength\titlebox{<dim>}
%
% and set <dim> to something 5cm or larger.

\title{LILaC: Late Interacting in Layered Component Graph for \\ Open-domain Multimodal Multihop Retrieval}

% Author information can be set in various styles:
% For several authors from the same institution:
% \author{Author 1 \and ... \and Author n \\
%         Address line \\ ... \\ Address line}
% if the names do not fit well on one line use
%         Author 1 \\ {\bf Author 2} \\ ... \\ {\bf Author n} \\
% For authors from different institutions:
% \author{Author 1 \\ Address line \\  ... \\ Address line
%         \And  ... \And
%         Author n \\ Address line \\ ... \\ Address line}
% To start a separate ``row'' of authors use \AND, as in
% \author{Author 1 \\ Address line \\  ... \\ Address line
%         \AND
%         Author 2 \\ Address line \\ ... \\ Address line \And
%         Author 3 \\ Address line \\ ... \\ Address line}

\author{
  Joohyung Yun \\
  POSTECH \\
  Republic of Korea \\
  \texttt{jhyun@dblab.postech.ac.kr} \\\And
  Doyup Lee \\
  DirectorLabs \\
  United States \\
  \texttt{doyup@directorlabs.ai} \\\And
  Wook-Shin Han \thanks{\ \ Corresponding author.} \\
  POSTECH \\
  Republic of Korea \\
  \texttt{wshan@dblab.postech.ac.kr} \\
}

%\author{
%  \textbf{First Author\textsuperscript{1}},
%  \textbf{Second Author\textsuperscript{1,2}},
%  \textbf{Third T. Author\textsuperscript{1}},
%  \textbf{Fourth Author\textsuperscript{1}},
%\\
%  \textbf{Fifth Author\textsuperscript{1,2}},
%  \textbf{Sixth Author\textsuperscript{1}},
%  \textbf{Seventh Author\textsuperscript{1}},
%  \textbf{Eighth Author \textsuperscript{1,2,3,4}},
%\\
%  \textbf{Ninth Author\textsuperscript{1}},
%  \textbf{Tenth Author\textsuperscript{1}},
%  \textbf{Eleventh E. Author\textsuperscript{1,2,3,4,5}},
%  \textbf{Twelfth Author\textsuperscript{1}},
%\\
%  \textbf{Thirteenth Author\textsuperscript{3}},
%  \textbf{Fourteenth F. Author\textsuperscript{2,4}},
%  \textbf{Fifteenth Author\textsuperscript{1}},
%  \textbf{Sixteenth Author\textsuperscript{1}},
%\\
%  \textbf{Seventeenth S. Author\textsuperscript{4,5}},
%  \textbf{Eighteenth Author\textsuperscript{3,4}},
%  \textbf{Nineteenth N. Author\textsuperscript{2,5}},
%  \textbf{Twentieth Author\textsuperscript{1}}
%\\
%\\
%  \textsuperscript{1}Affiliation 1,
%  \textsuperscript{2}Affiliation 2,
%  \textsuperscript{3}Affiliation 3,
%  \textsuperscript{4}Affiliation 4,
%  \textsuperscript{5}Affiliation 5
%\\
%  \small{
%    \textbf{Correspondence:} \href{mailto:email@domain}{email@domain}
%  }
%}

\begin{document}
\maketitle
\begin{abstract}

Multimodal document retrieval aims to retrieve query-relevant components from documents composed of textual, tabular, and visual elements. 
An effective multimodal retriever needs to handle two main challenges:
(1) mitigate the effect of irrelevant contents caused by fixed, single-granular retrieval units, and 
(2) support multihop reasoning by effectively capturing semantic relationships among components within and across documents. 
To address these challenges, we propose \texttt{LILaC}, a multimodal retrieval framework featuring two core innovations. 
First, we introduce a \textit{layered component graph}, explicitly representing multimodal information at two layers---each representing coarse and fine granularity---facilitating efficient yet precise reasoning. 
Second, we develop a \textit{late-interaction-based subgraph retrieval} method, an edge-based approach that initially identifies coarse-grained nodes for efficient candidate generation, then performs fine-grained reasoning via late interaction.
\updated{Extensive experiments demonstrate that \texttt{LILaC} achieves state-of-the-art retrieval performance on all five benchmarks, notably without additional fine-tuning.
We make the artifacts publicly available at \href{https://github.com/joohyung00/lilac}{\textcolor{linkpink}{\texttt{github.com/joohyung00/lilac}}}.
}
\end{abstract}



\section{Introduction}
\label{sec:introduction}
\vspace{-0.5em}


\begin{figure}[t]
  \centering
  \includegraphics[width=\linewidth]{figures/introduction.pdf}
  \vspace{-4mm}
  \caption{
    Motivating examples of multihop retrieval failures in existing graph retrieval approaches.
    (a) Vector-similarity-driven traversal follows a spurious cue.
    (b) Fixed retrieval plan produces an underspecified hop and fails to recover from a dead end.
  }
  \label{fig:motivating_example_figure}
  \vspace{-6mm}
\end{figure}



% Paragraph 1) Motivation of Multimodal Document RAG
Searching the web has become a part of everyday life.
This routine increasingly underpins multimodal retrieval-augmented generation (RAG), where a model answers a user query by grounding its output in retrieved evidence~\cite{4_colpali}.
In practice, much of this evidence lives in webpages or PDFs---multimodal documents with three salient characteristics:
(i) each document is composed of multimodal \textit{components} (paragraphs, tables, and images);
(ii) the meaning of a component is often shaped by local document context (e.g., captions and surrounding components); and 
(iii) components are connected through explicit signals (hyperlinks, cross-references) as well as implicit signals (e.g., same-section adjacency).
Moreover, documents themselves are linked via hyperlinks and citations, forming a large graph that users implicitly navigate while browsing.
We refer to the resulting setting as \emph{open-domain multimodal document retrieval} (OMDR): given a query, the system must return a small ranked set of relevant components from this large, noisy, and interlinked graph, often requiring multihop and multimodal exploration~\cite{1_lilac, 9_ircot, 4_colpali}.



% Paragraph 2) Transition to Graph-based Methods 
Given these intricate characteristics of OMDR, representing a document collection as a graph has emerged as a powerful paradigm for capturing the multi-granularity and interconnectedness of multimodal evidence~\cite{1_lilac}.
It shows the pros of preserving the structural dependencies and navigational scaffolds inherent in webpages, which is required when navigating heterogeneous components.
The most recent work introduces the \textit{layered component graph}, which organizes components together with their constituent subcomponents (sentences, table rows, and image objects)~\cite{1_lilac}.
In this formulation, \emph{navigational edges} encode relations among components (e.g., hyperlinks, same-section adjacency), while \emph{hierarchical edges} connect each component to its subcomponents.
By jointly modeling these edge types across layers, a retriever can traverse component-to-component paths for multihop exploration and move up/down the hierarchy to operate at the appropriate granularity.



% Paragraph 3) Their Limitations
While graph-based structures provide a rich representation of multimodal evidence, existing retrieval algorithms often struggle to fully exploit this potential due to operational rigidity.
In particular, (a) traversal is typically driven by a single, hop-agnostic embedding-based scoring rule and (b) executed with a largely pre-specified procedure, limiting dynamic error correction.
Figure~\ref{fig:motivating_example_figure} highlights these failures.
In Figure~\ref{fig:motivating_example_figure}(a), it follows a superficially related textual cue and retrieves an irrelevant snippet, failing to ground on the crucial visual evidence.
In Figure~\ref{fig:motivating_example_figure}(b), it issues an underspecified follow-up (``such battles'') and gets stuck in a dead end, rather than adapting its trajectory after the failure.
As a result, once the retriever follows a spurious edge or reaches a dead end, errors propagate across hops and degrade final retrieval quality~\cite{9_ircot, 10_selfrag}.
Moreover, these methods lack a principled mechanism for deciding \emph{when} expensive reasoning is warranted, leading to under-reasoning on ambiguous hops or over-spending computation across a trajectory.



% Paragraph 4) The Need for a Reasoning-Aware Sequential Decision Process
To overcome these limitations, we argue that a retriever must evolve from a static path-follower into an adaptive decision-maker that navigates the graph through a sequential reasoning process.
Concretely, OMDR is naturally stateful: as evidence accumulates, the information need shifts, and failures reveal which interpretations, routes, or strategies are unproductive.
This suggests casting traversal as a \emph{sequential decision process} over an evolving information state, where each step chooses (i) what to ask next (subquery), (ii) how to retrieve (tool/strategy), and (iii) where to move (edge type and granularity) conditioned on the current evidence.
Achieving this requires addressing three coupled challenges.
First, edge-following is not merely similarity matching; it often requires high-level reasoning to judge whether a candidate node will lead to the final answer under the current context.
Second, the retriever must adapt to evolving context by refining hypotheses and subqueries, and by recovering from dead ends using failure signals rather than adhering to a fixed, pre-defined plan.
Third, it must balance accuracy and efficiency: while LLM-based reasoning can improve retrieval precision, it introduces substantial overhead, so the system must decide economically when to escalate from lightweight matching to intensive reasoning.



% Paragraph 5) Our approach
To address these needs, we propose \textsc{\textbf{\OurFullName}} (\textsc{\Ours}).
We formalize OMDR as a finite-horizon \emph{information-state MDP}, where the state is a structured memory that records accumulated evidence together with the history of attempted subqueries, strategies, and explicit success/failure outcomes.
This formulation turns graph traversal into an \emph{economically-rational} agentic workflow: at each hop, an orchestrator dynamically decides \emph{what to ask}, \emph{how to retrieve}, and \emph{where to move} given the current information state, rather than executing a rigid traversal recipe.
To realize cost-sensitive control, \textsc{\Ours} maintains a portfolio of strategies across an accuracy--efficiency spectrum, starting from low-cost vector matching and escalating to higher-cost LLM reasoning only when a hop is ambiguous or an attempt fails.
Finally, to make multihop navigation resilient in noisy open-domain graphs, \textsc{\Ours} introduces \emph{history-aware backtracking}: unlike standard backtracking that simply reverts the state, our approach piggybacks on failure traces to re-anchor the search to a more promising prior context, revise subsequent subqueries, and avoid repeating previously failed routing patterns.




% Paragraph 6) Contributions
In summary, we make three primary contributions:
\vspace{-2mm}
\squishlist
    \item [1.] We formulate the OMDR problem as a sequential decision process with economic rationality. We redefine OMDR as an information-state MDP, operationalizing it through an LLM-enabled agentic workflow that treats retrieval strategy as a dynamic choice.
    \item [2.] We propose dynamic cost-aware strategy escalation.
    We introduce a novel mechanism that maintains a portfolio of strategies across an accuracy-efficiency spectrum. 
    Our orchestrator avoids over-reasoning by starting with low-cost vector matching and only escalating to high-cost LLM reasoning when a hop is identified as ambiguous or follows a recorded failure.
    \item [3.] We propose history-aware backtracking for resilient navigation, which converts failed traversals into constructive feedback. 
    By piggybacking on failure traces, the orchestrator re-anchors its search to prior contexts while revising its subqueries and escalating its strategy, enhancing both robustness and efficiency.
\squishend
\vspace{-2mm}












%%%%%%%%%%%%%% Version 5

% This capability is becoming a core primitive for multimodal RAG, as multimodal embedders and multimodal LLMs continue to improve in representing diverse modalities and reasoning over retrieved evidence~\cite{4_colpali}.



% While this graph-based structure provides a rich representation of the data, existing retrieval algorithms struggle to fully exploit its potential due to their operational rigidity.
% First, traversal is typically driven by a single, embedding-based scoring rule, which can miss hop-specific semantics that go beyond similarity.
% {\color{red} HI!}
% Second, traversal is often governed by a largely pre-specified procedure (e.g., a fixed sequence of subqueries executed with a fixed traversal routine), which limits dynamic error correction.
% As a result, once the retriever follows a spurious edge or gets trapped in a dead end, mistakes tend to propagate across hops and degrade final retrieval quality~\cite{9_ircot, 10_selfrag}.
% {\color{red} HI!}
% Crucially, these methods also lack a principled way to decide \emph{when} expensive reasoning is warranted: without explicit cost-awareness, they may either under-reason on ambiguous hops or over-spend computation across the trajectory.



% To overcome these limitations, we argue that a retriever must evolve from a static path-follower into an adaptive decision-maker that can navigate the graph through a sequential reasoning process.
% Achieving this requires addressing three key challenges.
% \textit{(1) Incorporating high-level reasoning.}
% The process of following an edge is not merely a similarity matching task; 
% it often requires complex logical deduction to determine whether a specific node will lead to the final answer based on the current evidence.
% \textit{(2) Adapting reasoning to evolving context.}
% As evidence is accumulated, the information need shifts; 
% thus, the retriever must dynamically refine its hypotheses and subqueries, and recover from dead ends by reflecting on failures rather than adhering to a fixed, pre-defined plan.
% \textit{(3) Balancing accuracy and efficiency.}
% While advanced reasoning via large models can improve retrieval precision, it introduces significant computational overhead.
% An effective system must maintain a delicate balance by orchestrating a spectrum of strategies, ranging from lightweight matching to intensive reasoning, depending on the difficulty of the current hop.


% To address these needs, we propose \textsc{\textbf{\OurFullName}} (\textsc{\Ours}), which introduces two key innovations:
% \textit{(1) LLM-enabled agentic traversal as a sequential decision process.}
% We cast graph traversal as a sequential decision process, where each hop decides \emph{what to ask} (subquery), \emph{how to retrieve} (tool/strategy), and \emph{where to move} (edge type/granularity) given the current evidence.
% To operationalize this formulation, we design an agentic framework in which each agent is explicitly equipped to invoke LLM reasoning when needed.
% Specifically, \textsc{\Ours} supports specialized tools (e.g., a planner and a multi-strategy traverser) coordinated by an orchestrator, enabling on-demand LLM reasoning and flexible switching across retrieval modes.
% \textit{(2) Adaptive accuracy--efficiency trade-offs with history-aware backtracking.}
% \textsc{\Ours} maintains a portfolio of strategy variants spanning an accuracy--efficiency spectrum, from low-cost vector matching for easy hops to high-cost LLM-intensive reasoning for ambiguous hops.
% Crucially, we incorporate a history-aware backtracking mechanism that turns failures into actionable feedback, making traversal both efficient and accurate.
% \emph{Efficiency:} the orchestrator begins with cheap strategies and escalates to stronger (but more expensive) reasoning only when a hop is difficult or an attempt fails.
% Upon reaching a dead end, the retriever backtracks while retaining failure traces to avoid redundant exploration and to revise the next hop (subquery/objective and strategy choice).
% \emph{Effectiveness:} these failure signals guide subsequent hops toward more plausible paths, improving multihop reliability without repeatedly paying for expensive reasoning.





%%%%%%%%%%% Version 4






% % Paragraph 1) Motivation of Multimodal Document RAG
% Searching the web has become part of everyday life.
% This routine increasingly underpins multimodal retrieval-augmented generation (RAG), where a model answers a user query by grounding its output in retrieved evidence~\cite{4_colpali}.
% In practice, much of this evidence lives in webpages or PDFs, which are multimodal documents that show the following characteristics
% (i) A document is composed of multimodal \textit{components}, mixing paragraphs, tables, and images.
% (ii) The meaning of each component is often shaped by local context within the document such as captions and surrounding components.
% (iii) Components are connected through explicit signals, including hyperlinks and cross-references, and through implicit signals, including same-section adjacency.
% Furthermore, documents are further linked to other documents through hyperlinks and citations, forming a large graph that users implicitly navigate while browsing.
% In such a situation, the \textit task asks a system to return a small ranked set of relevant components from this large, noisy, and interlinked graph, often requiring multihop and multimodal exploration~\cite{1_lilac, 9_ircot}.
% This capability is becoming a core primitive for multimodal RAG, as multimodal embedders and multimodal LLMs continue to improve in representing diverse modalities and reasoning over retrieved evidence~\cite{4_colpali}.




% % Paragraph 2) Transition to Graph-based Methods 
% Given these intricate characteristics of OMDR, representing a document collection as a graph structure has emerged as a powerful paradigm to capture the multi-granularity and interconnectedness of multimodal evidence~\cite{1_lilac}.
% Because OMDR requires navigating a web of heterogeneous components and their complex relational signals, a graph-based representation is suited to preserving the structural dependencies and navigational scaffolds inherent in webpages.
% The most recent work introduces \textit{layered component graph}, which explicitly organizes the collection of components and their constituent subcomponents (sentences, table rows, and image objects)~\cite{1_lilac}.
% In this formulation, \emph{navigational edges} encode relationships among components (e.g., hyperlinks, cross-references, and same-section adjacency), while \emph{hierarchical edges} connect each component to its subcomponents.
% By jointly modeling these edge types across layers, a retriever can traverse component-to-component paths for multihop exploration and move up/down the hierarchy to operate at the appropriate granularity, enabling multi-granularity retrieval and multihop reasoning within a unified graph.




% % Paragraph 3) Their Limitations
% However, while this graph-based structure provides a rich representation of the data, existing retrieval algorithms struggle to fully exploit its potential due to their operational rigidity.
% First, traversal is typically driven by a single, embedding-based scoring rule, which may not be sufficient to resolve some hop semantics that may go beyond similarity.
% Second, traversal is often governed by a largely pre-specified procedure (e.g., a fixed sequence of subqueries produced upfront and executed with a fixed traversal routine), which limits dynamic error correction.
% As a result, once the retriever follows a spurious edge or gets trapped in a dead end, mistakes tend to propagate across hops and degrade final retrieval quality~\cite{9_ircot, 10_selfrag}.



% % Paragraph 4) The Need for a Reasoning-Aware Sequential Decision Process
% {\color{red} 모두 문장으로 변경}
% To overcome these limitations, we argue that a retriever must evolve from a static path-follower into an adaptive decision-maker that can navigate the graph through a sequential reasoning process.
% Achieving this requires addressing three key challenges.
% \textit{(1) Incorporating high-level reasoning.}
% The process of following an edge is not merely a similarity matching task; 
% it often requires complex logical deduction to determine whether a specific node will lead to the final answer based on the current evidence.
% \textit{(2) Adapting reasoning to evolving context.}
% As evidence is accumulated, the information need shifts; 
% thus, the retriever must dynamically refine its hypotheses and subqueries, and recover from dead ends by reflecting on failures rather than adhering to a fixed, pre-defined plan.
% \textit{(3) Balancing accuracy and efficiency.}
% While advanced reasoning via large models can improve retrieval precision, it introduces significant computational overhead.
% An effective system must maintain a delicate balance by orchestrating a spectrum of strategies, ranging from lightweight matching to intensive reasoning, depending on the difficulty of the current hop.





% % Paragraph 5) Our approach
% {\color{red} 모두 문장으로 변경}
% To address these needs, we propose \textsc{\textbf{\OurFullName}} (\textsc{\Ours}), which introduces two key innovations:
% \textit{(1) LLM-enabled agentic traversal as a sequential decision process.}
% We cast graph traversal as a sequential decision process, where each hop decides \emph{what to ask} (subquery), \emph{how to retrieve} (tool/strategy), and \emph{where to move} (edge type/granularity) given the current evidence.
% To operationalize this formulation, we design an agentic framework in which each agent is explicitly equipped to invoke LLM reasoning when needed.
% Specifically, \textsc{\Ours} supports specialized tools (e.g., a planner and a multi-strategy traverser) coordinated by an orchestrator, enabling on-demand LLM reasoning and flexible switching across retrieval modes.
% \textit{(2) Adaptive accuracy--efficiency trade-offs with history-aware backtracking.}
% \textsc{\Ours} maintains a portfolio of strategy variants spanning an accuracy--efficiency spectrum, from low-cost vector matching for easy hops to high-cost LLM-intensive reasoning for ambiguous hops.
% Crucially, we incorporate a history-aware backtracking mechanism that turns failures into actionable feedback, making traversal both efficient and accurate.
% \emph{Efficiency:} the orchestrator begins with cheap strategies and escalates to stronger (but more expensive) reasoning only when a hop is difficult or an attempt fails.
% Upon reaching a dead end, the retriever backtracks while retaining failure traces to avoid redundant exploration and to revise the next hop (subquery/objective and strategy choice).
% \emph{Effectiveness:} these failure signals guide subsequent hops toward more plausible paths, improving multihop reliability without repeatedly paying for expensive reasoning.


% % Paragraph 6) Contributions
% In summary, we make three primary contributions:
% \squishlist
%     \item [1.] Formulation as a Sequential Decision Process with Economic Rationality: We redefine open-domain multimodal retrieval as an information-state MDP, operationalizing it through an LLM-enabled agentic workflow that treats retrieval strategy as a dynamic choice.
%     % We formulate open-domain multimodal retrieval as a \textbf{sequential decision process} and operationalize it through an \textbf{LLM-enabled agentic workflow}. 
%     %; this allows an orchestrator and specialized tools (e.g., planner, multi-strategy traverser) to adaptively decide what to ask, how to retrieve, and where to move at each hop.
%     \item [2.] Dynamic Cost-Aware Strategy Escalation: We introduce a novel mechanism that maintains a portfolio of strategies across an accuracy-efficiency spectrum. 
%     Our orchestrator avoids over-reasoning by starting with low-cost vector matching and only escalating to high-cost LLM reasoning when a hop is identified as ambiguous or follows a recorded failure.
%     % We introduce an \textbf{adaptive accuracy--efficiency mechanism} that combines cost-aware strategy escalation with \textbf{history-aware backtracking}. %, leveraging traces of failed traversals as constructive feedback to minimize redundant computation while significantly improving multihop reliability.
%     \item [3.] History-Aware Backtracking for Resilient Navigation: We convert failed traversals into constructive feedback. 
%     By piggybacking on failure traces, the agent re-anchors its search to prior contexts while revising its subqueries and escalating its strategy, ensuring both robustness and efficiency in complex graph environments.
%     % Extensive experiments show that \textsc{\Ours} achieves state-of-the-art retrieval performance on the \textsc{MultimodalQA}, \textsc{MMCoQA}, and \textsc{WebQA} benchmarks, significantly outperforming existing graph-based and single-index competitors. % by effectively navigating complex, interlinked multimodal documents.
% \squishend

















%%%%%%% Version 3


% Paragraph 4


% To overcome these limitations, we argue that a retriever must evolve from a static path-follower into an \textbf{adaptive decision-maker} that can navigate the graph through a sequential reasoning process.
% Achieving this requires addressing three key challenges.
% \textbf{(1) Reasoning-intensive retrieval.}
% In layered graphs, selecting what to retrieve (and which edge to follow) can require reasoning beyond direct similarity---e.g., inferring hop semantics from accumulated evidence, or interpreting visual/structural cues such as captions, table headers, and layout.
% \textbf{(2) Adaptive reasoning.}
% The optimal hop strategy changes as evidence accumulates; 
% thus, the retriever must dynamically refine its hypotheses and subqueries, and recover from dead ends by reflecting on failures rather than adhering to a fixed, pre-defined plan.
% \textbf{(3) Balancing accuracy and efficiency.}
% While LLM-based multimodal reasoning can be highly accurate for ambiguous hops, it is expensive; lightweight retrieval is efficient but often brittle.
% A practical OMDR system must orchestrate a spectrum of strategies and allocate computation adaptively per hop to achieve both reliability and low latency.


% To overcome these limitations, we argue that a retriever must evolve from a static path-follower into an \textbf{adaptive decision-maker} that can navigate the graph through a sequential reasoning process.
% Achieving this requires addressing three key challenges.
% \textbf{(1) Adapting reasoning to evolving context.}
% As evidence is accumulated, the information need shifts; thus, the retriever must dynamically adjust its strategy at each step rather than adhering to a fixed, pre-defined plan.
% \textbf{(2) Resilience to dead ends.}
% In open-domain graphs, encountering irrelevant paths or noisy subgraphs is inevitable.
% The challenge is not merely to backtrack to a previous state, but to \emph{recover intelligently} by analyzing why the previous path failed, turning the failure into a signal to prune future search paths.
% \textbf{(3) Orchestrating diverse strategies.}
% Different nodes and edges in a layered graph require different reasoning capabilities---some necessitate simple text matching, while others demand complex multimodal reasoning over visual and structural layouts.
% A robust system must possess a diverse toolkit and an orchestration mechanism to assign the most effective reasoning tool to each specific hop based on the evolving context.



% Paragraph 5

% To address these needs, we propose \textsc{\textbf{\OurFullName}} (\textsc{\Ours}), which introduces two key innovations:
% % Unlike prior methods that treat failure as a stopping condition, \textsc{\Ours} treats failure as a constructive signal through two key innovations.
% \textbf{(1) Agentic Workflow.}
% To execute this adaptive navigation, we introduce an agentic framework comprising a \textit{tool list} that covers diverse reasoning strategies (e.g., local/global hop, LLM reasoning, vector search granularity) and an \textit{orchestrator}.
% At each step, the orchestrator analyzes the current search context---including both successful history and previous failures---to assign the most effective tool and subquery for the next hop.
% This allows the system to seamlessly switch strategies, recovering from dead ends to find the correct evidence path.
% \textbf{(2) History-aware Backtracking.}
% Standard backtracking simply reverts the state, discarding the effort spent on the failed path.
% In contrast, our approach \textit{piggybacks} on the context of failed traversals.
% By leveraging insights from the "rejected" paths, the model optimizes the selection of alternative paths, ensuring it does not repeat the same mistake.


% To address these needs, we propose \textsc{\textbf{\OurFullName}} (\textsc{\Ours}), which introduces two key innovations:
% \textbf{(1) Modeling Traversal as a Sequential Decision Process}
% % Sequential decision process로 modeling한 다음에, 이 process의 각 부분을 agent로 설계한다.
% We introduce an agentic framework comprising a \textit{tool list} that covers diverse reasoning strategies and an \textit{orchestrator}.
% At each step, the orchestrator analyzes the current search context to assign the most effective tool and subquery for the next hop.
% This allows the system to seamlessly switch between different granularities and reasoning modes, ensuring that the most appropriate capability is applied to each specific part of the graph.
% \textbf{(2) Balancing Effectiveness and Efficiency via Adaptive Strategies and Backtracking.}
% To solve the retrieval problem efficiently, \textsc{\Ours} utilizes a range of strategy variants—from high-cost, LLM-intensive paths for complex reasoning to low-cost, vector-based paths for simple matching.
% Furthermore, we incorporate a \textbf{history-aware backtracking mechanism} that treats failures as constructive signals.
% Instead of simply discarding failed paths, our approach analyzes the context of "dead ends" to refine the selection of alternative routes, thereby preventing redundant computations and intelligently recovering from errors to find the correct evidence path.


% To address these needs, we propose \textsc{\textbf{\OurFullName}} (\textsc{\Ours}), which introduces two key innovations:
% \textbf{(1) Agentic Workflow.}
% We introduce an agentic framework comprising a \textit{tool list} that supports diverse hop reasoning operations (e.g., local/global hop, retrieval at multiple granularities, optional multimodal/LLM reasoning for disambiguation and verification) and an \textit{orchestrator}.
% At each step, the orchestrator analyzes the current search context---including the query, gathered evidence, and traversal history---to choose the next tool, subquery, and edge to execute.
% \textbf{(2) Efficient \& effective self-correcting traversal.}
% We explicitly construct multiple traversal variants that trade off cost and accuracy, spanning from efficient but coarse embedding-based strategies to more effective but slower LLM-heavy strategies, and invoke them selectively based on hop difficulty.
% Crucially, we incorporate \textit{history-aware backtracking}: when a path leads to an unproductive region, \textsc{\Ours} backtracks while retaining and exploiting the failure context as negative feedback, pruning future search paths to avoid repeating the same mistakes and improving multihop reliability.

% \textsc{\Ours} maintains a portfolio of strategy variants that span an accuracy--efficiency spectrum, ranging from low-cost vector-based matching for easy hops to high-cost LLM-intensive reasoning for ambiguous hops.
% Rather than committing to a single fixed routine, the orchestrator starts from efficient strategies and \emph{escalates} to stronger (but more expensive) reasoning only when the current hop is difficult or when earlier attempts fail.
% Crucially, we incorporate a \textbf{history-aware backtracking mechanism} that treats failures as constructive feedback.
% When the traversal hits a dead end, \textsc{\Ours} does not merely revert the state; it records the failed evidence, traversed edges, and attempted strategies, and uses them to (i) avoid redundant exploration of similar paths, (ii) revise the subquery or hop objective, and (iii) select a more appropriate strategy variant for the next attempt.
% This backtracking-driven adaptation improves \emph{efficiency} by minimizing unnecessary expensive calls and repeated search, while simultaneously improving \emph{effectiveness} by using failure signals to steer traversal toward more plausible evidence paths.

% \textsc{\Ours} maintains strategy variants spanning cheap vector matching to expensive LLM-intensive reasoning, and escalates cost only when needed.
% When a path fails, our history-aware backtracking records failure traces (evidence/edges/strategies) to avoid redundant exploration, revise the next hop, and choose better strategies, improving both efficiency and multihop reliability.



% To address these needs, we propose \textsc{\textbf{\OurFullName}} (\textsc{\Ours}), which introduces two key innovations:
% \textbf{(1) Modeling Traversal as a Sequential Decision Process with LLM-Enabled Agents.}
% We cast graph traversal as a sequential decision process, where each hop is an action that decides \emph{what to ask} (subquery), \emph{how to retrieve} (tool/strategy), and \emph{where to move} (edge type and granularity) based on the current evidence state.
% To operationalize this formulation, we design an agentic framework in which each agent is explicitly equipped to invoke LLM reasoning when needed.
% Concretely, \textsc{\Ours} provides a tool suite that assigns different roles---e.g., a \emph{planner} that proposes and refines hop-level objectives, and a \emph{multi-strategy traverser} that executes retrieval and edge-following across layers---together with an \emph{orchestrator} that supervises these agents.
% At every step, the orchestrator selects the most suitable agent/tool and generates the next subquery, enabling flexible switching across reasoning modes and retrieval granularities throughout the traversal.
% \textbf{(2) Balancing Effectiveness and Efficiency via Adaptive Strategies and History-Aware Backtracking.}
% \textsc{\Ours} maintains a portfolio of strategy variants that span an accuracy--efficiency spectrum, ranging from low-cost vector-based matching for easy hops to high-cost LLM-intensive reasoning for ambiguous hops.
% Rather than committing to a single fixed routine, the orchestrator starts from efficient strategies and \emph{escalates} to stronger (but more expensive) reasoning only when the current hop is difficult or when earlier attempts fail.
% Crucially, we incorporate a \textbf{history-aware backtracking mechanism} that treats failures as constructive feedback.
% When the traversal hits a dead end, \textsc{\Ours} does not merely revert the state; it records the failed evidence, traversed edges, and attempted strategies, and uses them to (i) avoid redundant exploration of similar paths, (ii) revise the subquery or hop objective, and (iii) select a more appropriate strategy variant for the next attempt.
% This backtracking-driven adaptation improves \emph{efficiency} by minimizing unnecessary expensive calls and repeated search, while simultaneously improving \emph{effectiveness} by using failure signals to steer traversal toward more plausible evidence paths.









%%%%%%%%%%%%%% Version 2


% Paragraph 2) Existing approaches & their problems
% Prior work on OMDR has advanced along two complementary directions.
% The \textit{first direction} reduces multimodal retrieval to a single-space nearest-neighbor search by converting every component into one modality.
% One common pipeline converts non-text components into text through OCR, captioning, or layout-to-text linearization, and then applies dense retrieval in the text embedding space~\cite{unimmqa, solar}.
% Another pipeline rasterizes pages or regions into images and retrieves them in a vision-language embedding space, treating retrieval as image search~\cite{3_visrag, 4_colpali, 5_m3docvqa}.
% These single-index methods are simple and scalable, but the conversion step can be lossy and can blur fine-grained signals into a large retrieval unit~\cite{6_densexretrieval, 7_mixofgran}.
% A flat index also treats each unit independently, so it does not directly exploit explicit connections such as within-page references or hyperlinks that support multihop retrieval.




% Given these intricate characteristics of OMDR, the underlying data for solving the task is inherently \textit{multi-granular} and \textit{highly interconnected}.
% Each document is composed of heterogeneous multimodal \textit{components}---including paragraphs, tables, and images---whose meanings are often shaped by local context such as captions, surrounding text, and layout.
% Moreover, components are linked not only within a document via explicit and implicit signals (e.g., cross-references, hyperlinks, and section-level adjacency), but also across documents through citations and hyperlinks, forming a large, noisy graph that users implicitly traverse while browsing.
% These properties make a graph abstraction particularly natural: by modeling components (and optionally documents/sections) as nodes and their contextual and navigational relations as edges, graph-based retrieval can explicitly leverage the connectivity signals required for multihop evidence discovery, and has thus emerged and evolved as a principled paradigm for OMDR~\cite{1_lilac, 9_ircot}.
% In particular, recent work shows that a \textit{layered component graph} better exploits these characteristics by representing document--component relations at multiple granularities (e.g., coarse document/section-level navigation followed by fine-grained component selection), enabling more effective and scalable multihop retrieval over multimodal corpora~\cite{1_lilac}.



% Gemini
% Paragraph 2) Transition to Graph-based Methods and their Limitations
% Given these intricate characteristics of OMDR, representing a document collection as a \textit{layered component graph} has emerged as a powerful paradigm to capture the multi-granularity and interconnectedness of multimodal evidence~\cite{1_lilac}.
% By modeling each paragraph, table, and image as distinct nodes and encoding their structural relationships---such as intra-document layouts and inter-document hyperlinks---as edges, this approach allows retrievers to navigate the complex information landscape more effectively than flat, single-index methods.

% However, while this graph-based structure provides a rich representation of the data, existing retrieval algorithms fail to fully exploit its potential due to their operational rigidity.
% First, they predominantly rely on a uniform, vector-based similarity metric for traversal, which overlooks the \textit{hop-specific semantics}---failing to distinguish when a hop requires visual pattern matching versus logical text-based deduction.
% Second, these methods typically operate under rigid, pre-defined plans, such as executing a fixed sequence of subqueries generated a priori.
% Consequently, they lack the flexibility for dynamic error correction; once the retriever follows a misleading link or encounters a noise-heavy subgraph, it cannot recover, leading to inevitable error propagation throughout the chain~\cite{9_ircot, 10_selfrag}.



% ChatGPT
% Paragraph 2) Prior Works
% The characteristics of OMDR data naturally motivate a graph-based view.
% Because a corpus consists of heterogeneous \textit{components} (paragraphs, tables, images), whose meanings depend on local context (captions, surrounding text, layout), and because components and documents are densely connected via explicit and implicit links (hyperlinks, cross-references, adjacency), representing the corpus as a component graph allows a retriever to directly exploit the same navigational signals that users follow while browsing.
% Accordingly, recent work has advanced structure-aware retrieval by constructing a \textit{layered component graph} and performing multihop subgraph retrieval on top of it~\cite{1_lilac}.
% In particular, LILaC models multimodal information at dual granularities (coarse vs.\ fine), enabling efficient candidate discovery at a coarse layer while preserving precise evidence selection through fine-grained components~\cite{1_lilac}.
% Overall, this line of work demonstrates that explicitly modeling document--component relations and connectivity is a powerful inductive bias for OMDR.

% However, despite the representational gains of layered graphs, existing graph-based retrievers still share rigidity in their \textit{retrieval algorithms}.
% First, traversal is typically driven by a single, embedding-based scoring rule, which is not sufficient to express \textit{hop-specific needs} that may go beyond similarity---e.g., deciding when to rely on direct multimodal matching versus reasoning over previously accumulated evidence.
% Second, traversal is often governed by a largely pre-specified procedure (e.g., a fixed sequence of subqueries produced upfront and executed with a fixed traversal routine), which limits dynamic error correction.
% As a result, once the retriever follows a spurious edge or gets trapped in a dead end, mistakes tend to propagate across hops and degrade final retrieval quality~\cite{9_ircot, 10_selfrag}.

% ChatGPT
% Paragraph 3) Challenges for reasoning-aware graph traversal
% To overcome this algorithmic rigidity, we view OMDR graph traversal as a \textit{sequential decision process} in which the retriever repeatedly (i) interprets the current state---the user query, retrieved evidence, and traversal history---and (ii) chooses the next retrieval action.
% This shift from static planning to decision-making introduces three key challenges.
% \textbf{(1) Adapting reasoning to evolving context.}
% Since the information need evolves as evidence is accumulated, the retriever must adjust its strategy at each step, rather than committing to a fixed plan decided at the beginning.
% \textbf{(2) Resilience to dead ends.}
% In open-domain graphs, failures are inevitable due to noise, missing links, or misleading associations.
% The goal is not only to backtrack, but to \emph{recover intelligently} by diagnosing why a path failed and using that failure as feedback for selecting a better alternative.
% \textbf{(3) Orchestrating diverse strategies.}
% Different hops require different capabilities---some benefit from coarse-grained navigation, others from fine-grained evidence selection, and still others from multimodal or logical reasoning.
% An effective system therefore needs a diverse toolkit and an orchestration mechanism that can assign the most suitable strategy to each hop based on the current context.










% % Paragraph 2) Prior Works
% The \textit{second direction} makes structure explicit by representing a document collection as a \textit{layered component graph} and performing graph traversal for retrieval~\cite{1_lilac}.
% It models each paragraph, table, and image as a node, utilizing edges to capture co-occurrence within documents and navigational links across them.
% While this structure-aware view supports multihop retrieval, existing approaches exhibit two critical limitations rooted in their rigidity.
% First, they rely on a uniform, vector-based similarity metric for traversal, which overlooks \textit{hop-specific semantics}---failing to distinguish when a hop requires visual matching versus logical deduction.
% Second, these methods typically operate under rigid, pre-defined plans (e.g., executing a fixed sequence of subqueries generated a priori).
% Consequently, they lack the flexibility to perform dynamic error correction; once the retriever follows a misleading link or encounters a dead end, it cannot recover, leading to error propagation throughout the chain~\cite{9_ircot, 10_selfrag}.



% % Paragraph 3) Challenges for reasoning-aware graph traversal
% These limitations suggest that a robust retriever must not only traverse a graph but also operate as an adaptive decision-maker.
% To achieve this, the retrieval process must evolve from static planning to a sequential decision process, addressing three key challenges.
% \textbf{(1) Adapting reasoning to evolving context.}
% Since the information need changes as evidence is accumulated, the retriever must dynamically adjust its strategy at each step, rather than following a static plan.
% \textbf{(2) Resilience to dead ends.}
% In open-domain graphs, failures are inevitable due to noise or missing links.
% The challenge is not merely to backtrack (i.e., revert to a previous state) but to \emph{recover intelligently} by analyzing why the previous path failed, turning the failure into a signal for the next move.
% \textbf{(3) Orchestrating diverse strategies.}
% Different hops require different reasoning capabilities---some need simple text matching, while others require complex multimodal reasoning.
% An effective system must possess a diverse toolkit and an orchestration mechanism to assign the most effective reasoning tool to each specific hop.














%%%%%%%%%%%%%%% Version 1






% % Paragraph 1) Motivation of Multimodal Document RAG
% Searching the web has become part of everyday life.
% This routine increasingly underpins multimodal retrieval-augmented generation (RAG), where a model answers a user query by grounding its output in retrieved evidence~\cite{4_colpali}.
% In practice, much of this evidence lives in webpages or PDFs, which are multimodal documents that show the following characteristics
% (i) A document is composed of multimodal \textit{components}, mixing paragraphs, tables, and images.
% (ii) The meaning of each component is often shaped by local context within the document such as captions, surrounding text, or layout.
% (iii) Components are connected through explicit signals, including hyperlinks and cross-references, and through implicit signals, including same-section adjacency.
% Furthermore, documents are further linked to other documents through hyperlinks and citations, forming a large graph that users implicitly navigate while browsing.
% In such a situation, the \textbf{open-domain multimodal document retrieval (OMDR)} task asks a system to return a small ranked set of relevant components from this large, noisy, and interlinked graph, often requiring multihop exploration~\cite{1_lilac, 9_ircot}.
% This capability is becoming a core primitive for multimodal RAG, as multimodal embedders and multimodal LLMs continue to improve in representing diverse modalities and reasoning over retrieved evidence~\cite{4_colpali}.



% % Paragraph 2) Existing approaches & their problems
% Prior work on OMDR has advanced along two complementary directions.
% The \textit{first direction} reduces multimodal retrieval to a single-space nearest-neighbor search by converting every component into one modality.
% One common pipeline converts non-text components into text through OCR, captioning, or layout-to-text linearization, and then applies dense retrieval in the text embedding space~\cite{unimmqa, solar}.
% Another pipeline rasterizes pages or regions into images and retrieves them in a vision-language embedding space, treating retrieval as image search~\cite{3_visrag, 4_colpali, 5_m3docvqa}.
% These single-index methods are simple and scalable, but the conversion step can be lossy and can blur fine-grained signals into a large retrieval unit~\cite{6_densexretrieval, 7_mixofgran}.
% A flat index also treats each unit independently, so it does not directly exploit explicit connections such as within-page references or hyperlinks that support multihop retrieval.


% The \textit{second direction} makes structure explicit by representing a document collection as a \textit{component graph} and performing graph traversal for retrieval~\cite{1_lilac}.
% It models each paragraph, table, and figure as a node, and it adds edges that capture co-occurrence within the same document and navigational links across documents.
% With each node embedded, it traverses through the graph by expanding through the edges.
% This structure-aware view supports multihop retrieval, but current traversal pipelines still resolve relationships in a shallow manner.
% Edge-based traversal is driven primarily by embedding similarity, which may not interpret what a specific link implies for the query.
% It also traverses using a single fixed plan and fixed scoring rules, driven by decomposed subqueries generated once before the traversal. 
% As a result, the retriever has limited ability to revise subgoals, retry with stronger reasoning, or switch traversal strategies when early steps are ambiguous or wrong~\cite{9_ircot, 10_selfrag}.




% % Paragraph 4) Challenges for reasoning-aware graph traversal
% These limitations point to a missing capability in current structure-aware retrieval.
% A retriever should not only traverse a document graph but also \emph{interpret} what an edge implies for the query, and \emph{revise} its plan as intermediate evidence changes which hop is most promising~\cite{9_ircot}.
% A natural direction is to use an LLM as a controller during traversal~\cite{10_selfrag, 11_react}.
% In this view, retrieval becomes a sequential decision process: at each step, the retriever determines the current subgoal, selects candidates to explore next, and chooses how much computation to spend before committing to a hop.

% A practical solution, however, faces three challenges.
% \textbf{(1) Exploration beyond local neighbors.}
% Query-relevant evidence is not always reachable via short-range neighbor expansion, since hyperlinks and cross-references can be sparse, noisy, or missing in open corpora.
% The retriever therefore must interleave local traversal with occasional \emph{global jumps} (e.g., corpus-wide search), while still exploiting the graph when it provides reliable navigational cues.
% \textbf{(2) Adaptive retrieval strategies.}
% Successful multihop retrieval requires handling diverse cases that cannot be covered by a single fixed procedure.
% First, the appropriate retrieval \emph{granularity} varies across queries and across hops, ranging from coarse components to fine-grained regions~\cite{6_densexretrieval, 7_mixofgran}.
% Second, the need for \emph{explicit reasoning} is also hop-dependent: some steps can be resolved by cheap similarity matching, while others require reasoning over an edge-induced dependency (e.g., interpreting what a link or reference implies under the current subgoal).
% \textbf{(3) Accuracy--efficiency trade-off.}
% Invoking an LLM at every hop for planning, critique, and relation interpretation can be expensive and unstable~\cite{10_selfrag}.
% The retriever should escalate reasoning only when necessary, reuse intermediate results to avoid redundant search, and carefully control the \emph{context budget} passed to the LLM so that reasoning does not rely on ever-growing prompt contexts.

% % Document-level도 지금 생략된 상태
% % Replanning은 지금 생략된 상태






% % Paragraph 5) Our approach
% To address these challenges, we propose \textsc{\Ours}, an accurate component retriever based on LLM-guided graph traversal over a layered component graph.
% We build on a representation that treats a multimodal corpus as connected \textit{components}, following prior work~\cite{1_lilac}.
% \textsc{\Ours} adds adaptive control on top of this structure through two ideas.
% \textbf{(1) Multi-strategy traversal.}
% \textsc{\Ours} maintains a strategy bank that varies traversal scope, retrieval granularity, and reasoning effort.
% Depending on the subgoal and the reliability of local links, the agent can expand only to a node's neighbors, re-seed with a corpus-wide vector search, or combine both when local edges are uninformative.
% Candidate scoring can likewise adapt: it may operate at the coarse component level, descend to fine-grained subcomponents, or use hybrid scoring that focuses on small query-relevant regions inside a component.
% When similarity signals are ambiguous, the agent selectively invokes an LLM to interpret relations induced by links or adjacency and rerank candidates accordingly.
% \textbf{(2) Progressive Traversal Orchestration.}
% To balance accuracy and efficiency, \textsc{\Ours} schedules strategies from cheap to expensive.
% After each retrieval attempt, the orchestrator inspects the retrieved components and judges whether the current subgoal is satisfied; if not, it either retries with a stronger strategy or replans the subgoal using partial evidence already obtained.
% If the subgoal is satisfied, the orchestrator summarizes the newly acquired evidence into an updated state, derives the next subgoal (or subquery), and selects an appropriate strategy to pursue it.
% This failure-aware loop corrects early mistakes without restarting traversal from scratch, while limiting LLM usage by escalating only on demand and keeping prompts compact via evidence selection and reuse.




% % Paragraph 6) Contributions
% In summary, we make four contributions.
% \squishlist
%     \item [1.] We motivate a reasoning-aware and failure-aware view of open-domain multimodal component retrieval, and cast it as sequential, agentic traversal over a linked multimodal component graph.
%     \item [2.] We introduce a multi-strategy traversal agent with a strategy bank that adapts traversal scope (local expansion vs.\ global jumps), retrieval granularity (coarse vs.\ fine), and optional LLM reasoning for relation interpretation and reranking.
%     \item [3.] We design a progressive traversal orchestrator that detects failures, retries, and replans subgoals using intermediate evidence, achieving improved retrieval accuracy and accuracy.
%     \item [4.] Extensive experiments show that our implementation of \textsc{\Ours} achieves the state-of-the art results on the benchmarks of \textsc{MultimodalQA}, \textsc{MMCoQA}, and \textsc{WebQA}, both on retrieval accuracy and end-to-end QA accuracy.
% \squishend





























% \begin{figure*}[t]
%   \centering
%   \includegraphics[width=\linewidth]{figures/Figure3.pdf}
%   \caption{
%     \textbf{Overview of Evaluation Benchmarks.}
%     We evaluate our method across simulation and real-world tasks.
%     \textbf{Top (Simulation):} We construct \textit{Personalized-SIMPLER} (left/middle) and \textit{Personalized-VLABench} (right) by repopulating existing environments with user-specific assets.
%     \textbf{Bottom (Real-world):} We conduct physical experiments on a SO-101 arm using 8 diverse object categories, covering both selection and pick-and-place tasks.
%     In all scenarios, the agent must identify a specific target instance among visually similar distractors, requiring precise instance-level grounding beyond generic category recognition.
%     }
%   \label{fig:benchmarks}
% \end{figure*}




% These limitations point to a missing capability in current structure-aware retrieval. 
% Traversal should be able to interpret what an edge or a relation implies for the query, and it should revise the plan when intermediate evidence changes the next hop~\cite{9_ircot}.
% A natural direction is to use an LLM as a controller during traversal~\cite{10_selfrag, 11_react}.
% The retriever repeatedly decides the current subgoal, the next candidates to explore, and the amount of computation to spend before committing to a hop.
% A practical solution, however, faces three challenges. 
% \textbf{(1) Exploration beyond local neighbors.} 
% Relevant evidence is not always reachable by short neighbor expansion, because hyperlinks can be sparse, noisy, or missing in open corpora.
% The retriever therefore needs to mix local traversal with occasional global jumps, while still exploiting the graph when it is informative.
% \textbf{(2) Adaptive retrieval strategies.}
% 올바른 retrieval을 위해서는 여러 경우의 수를 고려해야 하며, 이런 것들은 다양한 strategy로 표현되어야 한다.
% 첫째는 vector search의 granularity로, the right retrieval unit changes across queries and across hop~\cite{6_densexretrieval, 7_mixofgran}.
% 둘째는 traversal에 LLM의 reasoning이 필요한지의 여부로, Some steps succeed with coarse components and cheap similarity, while others require fine-grained matching or explicit reasoning about a dependency expressed by an edge.
% \textbf{(3) Accuracy--efficiency trade-off.} 
% Agentic control can be expensive and unstable if an LLM is invoked at every hop for planning and critique.
% The retriever should escalate reasoning only when necessary, and it must reuse intermediate results to avoid redundant search.
% 또한, 각 reasoning을 할 때 LLM에게 주어지는 context의 길이를 크게 하지 않도록 조절하는 것 또한 중요하다. 




% To address these challenges, we propose \textsc{\Ours}, an accurate component retriever based on llm-reasoning-powered graph traversal over a layered component graph.
% We build on a graph representation that treats a multimodal corpus as connected \textit{components}, following prior work~\cite{1_lilac}. 
% % Each page contributes coarse nodes for whole paragraphs, tables, and figures, and it also contributes fine nodes for their constituent units such as sentences, table rows, or detected visual objects.
% % Edges preserve navigational signals, including within-page adjacency and cross-document hyperlinks, and they also connect each coarse node to its fine-grained subcomponents.
% \textsc{\Ours} adds adaptive control on top of this structure through two ideas. 
% \textbf{(1) Multi-strategy traversal.} 
% \textsc{\Ours} maintains a strategy bank that varies traversal scope, similarity granularity, and reasoning effort. 
% A strategy may expand only graph neighbors, perform a corpus-wide vector search, or combine both when local edges are uninformative.
% A strategy may score candidates at the coarse level, at the fine level, or with hybrid scoring that focuses on small query-relevant regions inside a component.
% A strategy may also invoke an LLM to interpret relations and rerank candidates when similarity alone is ambiguous. 
% \textbf{(2) Progressive Traversal Orchestration.} 
% \textsc{\Ours} schedules strategies from cheap to expensive to balance accuracy and efficiency.
% After each retrieval attempt, the orchestrator inspects the retrieved components and judges whether the current subgoal is satisfied.
% If it is not, the orchestrator either retries with a stronger strategy or replans the subgoal using the partial evidence already retrieved.
% This failure-aware loop corrects early mistakes without restarting traversal from scratch, and it limits LLM usage by escalating only on demand.
% Finally, we aggregate candidates across attempts and output a compact ranked evidence set for downstream multimodal RAG.



% In summary, we make four contributions. 
% \squishlist
%     \item [1.] We motivate a failure-aware and reasoning-aware view of open-domain multimodal component retrieval, and we cast it as agentic traversal over a multimodal component graph with explicit links.
%     \item [2.] We present \textsc{\Ours}, which augments layered component graphs with adaptive planning while preserving multimodal component boundaries and navigational structure.
%     \item [3.] We introduce a traversal agent with a strategy bank that controls traversal scope, retrieval granularity, and optional LLM reasoning for relation interpretation and reranking.
%     \item [4.] We design a progressive traversal orchestrator that retries, switches strategies, and replans subtasks based on intermediate evidence, and we demonstrate improved retrieval accuracy and accuracy--cost trade-offs on open-domain multimodal multihop benchmarks.
% \squishend

\section{Preliminary}
\vspace{-2mm}

In this paper, we address multimodal document retrieval, defined as the task of retrieving a ranked list of multimodal components relevant to a given natural language query.
Formally, a retrieval corpus $\mathcal{D}$ comprises a collection of multimodal documents $\{D_1, D_2, \dots, D_{k_{doc}}\}$.
Each multimodal document $D = [C_1, \dots, C_{k_{comp}}]$ is a sequence of multimodal components.
A multimodal component $C$ may belong to one of three distinct modalities
\squishlist
    \item \textit{Paragraph} $P$: a sequence of tokens, forming an unstructured text segment.
    \item \textit{Table} $T$: a structured matrix with rows $T_{i}$ indexed by row number $i$.
    \item \textit{Image} $I$: a tensor $I \in \mathbb{R}^{w \times h \times a}$, with $w$, $h$, and $a$ denote the width, height and the number of channels, respectively.
\squishend
Given a natural language query $Q$, a retrieval corpus $\mathcal{D}$ and a link mapping $\mathcal{L}$, the retrieval task aims to produce a ranked list of components $\mathcal{R} = [C_1, \dots, C_{n_{ret}}]$.
% The goal for $\mathcal{R}$ is to include the ground truth set of relevant components $\{C_{gt_1}, \dots,  C_{gt_r}\}$.
The goal is for the ranked list $\mathcal{R}$ to contain the ground truth set of relevant components ${C_{gt_1}, \dots, C_{gt_r}}$.

The link mapping $\mathcal{L} = \mathcal{C} \to \mathcal{D}$ represents the association or hyperlink relationships between individual components $C$ and their respective multimodal documents $D$, similar to hyperlinks commonly used in webpages and PDF files.

\section{Related Work}



\subsection{Multimodal Document Retrieval}


Early multimodal retrieval methods primarily used a \emph{text-centric} strategy, converting all components—paragraphs, tables, and figures—into plain text, thus losing essential visual cues~\cite{skurg, solar, unimmqa, helios}. 
Later approaches maintained separate embedding spaces for text and images, encoding each modality independently and merging their scores heuristically~\cite{mmragsurvey, beyondtext}. 
However, these methods struggle with reasoning across modalities due to disjoint embeddings.


Recent work pushes modality unification a step further through \textbf{VisRAG} pipelines: documents are rasterized into page- or region-level screenshots, so that paragraphs, tables, and images alike are embedded in a single \emph{visual} space.
\texttt{VisRAG} demonstrates end-to-end vision-based retrieval–augmented generation, while \texttt{ColPali} introduces a late-interaction vision–language model that produces multi-vector page embeddings.
Despite their strengths, VisRAG approaches inherit some limitations.
(i) \textbf{Fixed granularity:} retrieval granularity is fixed as full-page screenshots, which may contain query-irrelevant context.
(ii) \textbf{Limited multihop reasoning:} current pipelines treat each screenshot independently, ignoring the dependencies between components.



\subsection{Granularity of Retrieval}

\begin{figure*}[ht]
  \centering
  \includegraphics[width=\linewidth]{figures/system_overview.pdf}
  \caption{Overview of \texttt{LILaC}.
  (a) A layered component graph is constructed by organizing multimodal documents into coarse- and fine-grained layers. 
  (b) The query is decomposed, followed by modality classification for each subquery.
  (c) \texttt{LILaC} dynamically retrieves a query-relevant subgraph through iterative beam-search traversal. 
  }
  \label{fig:idea_overview}
  \vspace{-0.5cm}
\end{figure*}

Previous studies have explored retrieval granularity across various modalities. 
In text retrieval, \texttt{DenseXRetrieval} demonstrates improved retrieval accuracy using finer sentence- and proposition-level units~\cite{densexretrieval}. 
\texttt{Mix-of-granularity} dynamically selects the optimal granularity tailored to each query~\cite{mixofgranularity}, while \texttt{RAPTOR} starts from sentences and recursively clusters and summarizes them into coarser units~\cite{raptor}. 
For table modality, \textsc{OTT-QA} segments tables into header-plus-row units for targeted row-level retrieval~\cite{ottqa}. 
However, granularity in multimodal document retrieval remains largely unexplored.









\subsection{Multimodal Embedder Models}

Recently, multimodal embedders and their corresponding benchmarks~\cite{vlm2vec, uniir} have emerged as active research areas due to the limitations of traditional uni- or cross-modal embedders in dynamic retrieval scenarios.
Unlike conventional unimodal embedders~\cite{dpr, sentencebert}, multimodal approaches specifically address dynamic settings characterized by retrieval tasks guided by explicit \textit{modality instructions}.
Advanced models such as \texttt{MMEmbed}, \texttt{UniME}, and \texttt{mmE5} leverage sophisticated multimodal language models along with modality-specific fine-tuning, significantly improving retrieval performance under clear modality instructions~\cite{mmembed, unime, mme5}.
However, existing multimodal embedders predominantly focus on training at the component level, leaving the effective use of these models for multimodal document retrieval largely unexplored.
Furthermore, scenarios involving retrieval tasks without explicit instructions or with ambiguous contexts have yet to be thoroughly investigated.








\section{Proposed Method}


We propose \texttt{LILaC}, a novel retrieval algorithm utilizing a layered component graph and traversal method to retrieve a query-relevant subgraph. 
As shown in Figure~\ref{fig:idea_overview}, it consists of two stages: 
(i) \textbf{Layered Graph Construction} organizes multimodal documents into a layered component graph with explicit intra- and inter-document edges. 
(ii) \textbf{Late-interaction-based Subgraph Retrieval} iteratively traverses the layered graph in an edge-wise manner.
To score an edge using node-level embeddings, it uses late interaction between the decomposed subqueries and low-layer subcomponents of an edge.
















\subsection{Layered Component Graph Construction}
% 0.75페이지


In the offline phase, \texttt{LILaC} constructs a layered graph structure $\mathcal{G}$, called the \textit{layered component graph}, from the multimodal document set $\mathcal{D}$ and the associated link mapping $\mathcal{L}$.
\updated{
The graph is designed to represent \emph{relationships among components} while also allowing each component to be expressed via \emph{fine-grained constituent elements}.
It comprises two distinct layers explicitly designed to represent semantic relationships among multimodal components, offering two primary advantages.
First, the top layer supports multihop retrieval by explicitly modeling relationships between components and documents, enabling identification of relevant contexts.
Second, the lower layer facilitates precise, fine-grained reasoning by further decomposing components into finer \textit{subcomponents} (defined in Definition~\ref{def:subcomponent}), thus providing detailed context for accurate retrieval.
In addition, the edges explicitly encode two relations among these nodes: 
(i) \emph{hierarchical containment}, which links coarse components to fine-grained subcomponents; 
and (ii) \emph{navigational relations}, which preserve potential cross-component affinity (both intra- and cross-document) without prematurely committing to a specific semantic.}

% This graph comprises two distinct layers explicitly designed to represent semantic relationships among multimodal components, offering two primary advantages.
% First, the top layer supports multihop retrieval by explicitly modeling relationships between components and documents, enabling identification of relevant contexts.
% Second, the lower layer facilitates precise, fine-grained reasoning by further decomposing components into finer \textit{subcomponents}, thus providing detailed context for accurate retrieval.

\begin{definition}[\textbf{Layered Component Graph}]
\label{def:layered_component_graph}
We define a \textit{layered component graph} as $\mathcal{G} = (V, E, \lambda, \tau)$, where $V$ is a set of vertices.
A vertex $v$ belongs to one of the two layers, determined by the layer map \(\lambda : V \to \{0,1\}\), where $0$ and $1$ corresponds to the coarse-grained and fine-grained nodes, respectively.
\vspace{-3mm}
\begin{align*}
    V_0 &= V_{\text{para}}\;\cup\;V_{\text{tbl}}\;\cup\;V_{\text{img}} \\ 
    V_1 &= V_{\text{sent}}\;\cup\;V_{\text{row}}\;\cup\;V_{\text{obj}} 
    \vspace{-3mm}
\end{align*}
We denote each vertex set - \(V_{\text{para}}\): paragraphs, \(V_{\text{tbl}}\): tables, \(V_{\text{img}}\): images, \(V_{\text{sent}}\): sentences, \(V_{\text{row}}\): table rows, \(V_{\text{obj}}\): visual objects detected in images.
The type map
\(\tau : V \to
  \{\texttt{para},\texttt{tbl},\texttt{img},\texttt{sent},\texttt{row},\texttt{obj}\}
\)
refines the vertex set $V$ into the six disjoint categories.
The edge set \(E \subseteq V \times V\) is the union \(E = E_{0} \cup E_\downarrow\) where
\vspace{-3mm}
\begin{align*}
  E_0 &\;=\; \bigl\{(u,v)\in V_0^2\}              \\
  E_\downarrow &\;=\; \bigl\{(u, v)\mid u \in V_0, v \in V_1\}
  \vspace{-3mm}  
\end{align*}
\(E_0\) captures \emph{relationships} between the macro components, while \(E_\downarrow\) captures the containment of a macro component of its subcomponent.
\end{definition}

\begin{definition}[\textbf{Subcomponent}]
\label{def:subcomponent}
Let \(C\) be a multimodal component.  
A \emph{subcomponent} \(c \in \mathcal{S}(C)\) is defined in a modality-specific manner:
\squishlist
    \item \textbf{Paragraph.}  
          For a paragraph \(P = [p_1,\dots,p_{k_{sent}}]\) consisting of sentences, each sentence $p_j$ is a subcomponent.
    \item \textbf{Table.}  
          Let \(T = [T_0;T_1;\dots;T_{k_{row}}]\) where \(T_0\) is the header row.  
          For every data row \(T_i \;(1 \le i \le k_{row})\), the two-row segment $t_i = [\,T_0;\,T_i\,]$ is a subcomponent.
    \item \textbf{Image.}  
          Given an image tensor \(I \in \mathbb{R}^{w \times h \times a}\) and an object detector that
          returns a bounding box \((x_1,y_1,x_2,y_2)\), the corresponding patch
          \vspace{-3mm}
          \[
            i = I[x_1:x_2,\;y_1:y_2,\,:]
            \vspace{-3mm}
          \]
          is a subcomponent.
\squishend
\end{definition}

% Paragraph 2) Tree를 먼저 만들고, 이들끼리 link를 만들어 최종 graph를 생성한다.
Layered component graph $\mathcal{G}$ is constructed in two steps.
First, \texttt{LILaC} builds a \textit{component tree} for each component $C$ within $\mathcal{D}$.
A component tree is a two-level tree structure with the root representing the component itself and its children representing the subcomponents, which are extracted differently depending on the modality of the component.
\updated{The roots and leaves of these trees form the nodes of $V_0$ and the nodes of $V_1$, respectively, while the parent–child links correspond to the edges in $E_{\downarrow}$.}
For a paragraph $P$, \texttt{LILaC} utilizes a Sentence-aware Transformer (\texttt{SaT}) model to split it into a set of sentences.
A table $T$ is parsed to generate a set of table segments.
Lastly, a multimodal LLM is used to detect objects within $I$.
\texttt{LILaC} then generates an edge $(C, c) \in E_\downarrow$ for $c \in \mathcal{S}(C)$.

    % Paragraph 3) Component끼리의 edge를 만드는 방법
In the next step, \texttt{LILaC} generates the inter-component edges $E_0$ using both inherent structural relationships and hyperlink-based connections. 
For every document $D \in \mathcal{D}$, a clique is formed among its components:
\vspace{-3mm}
\begin{equation}
    E_{intra} = \{(C_i, C_j) | C_i \neq C_j, C_i, C_j \in D\}
    \vspace{-3mm}
\end{equation}
To enable cross-document multihop reasoning, \texttt{LILaC} then follows the link mapping $\mathcal{L}$.
For each pair $(C, D) \in \mathcal{L}$, it connects $C$ to every component in the linked document $\mathcal{D}$.
\vspace{-3mm}
\begin{equation}
    E_{inter} = \{(C, C') | (C, D) \in \mathcal{L}, C' \in D\}
    \vspace{-3mm}
\end{equation}
The inter-component edge set for the top layer is therefore $E_0 = E_{intra} \cup E_{inter}$.
Finally, every node $v \in V$ receives an embedding $\textbf{v} = f(v)$ from a pre-trained multimodal encoder $f$.











\subsection{Late-interaction-based Subgraph Retrieval}


During the online phase, \texttt{LILaC} retrieves a query-relevant subgraph $\mathcal{G}'$ from the layered component graph $\mathcal{G}$ given a query $Q$.  
This retrieval faces two key challenges: 
(1) Direct identification of an optimal subgraph from all possible candidates is computationally infeasible due to a combinatorial explosion~\cite{grag}. 
In particular, the layered component graph contains numerous edges, making explicit embedding of all edges prohibitively expensive in terms of space and computation.
(2) Queries often lack explicit modality instructions, causing ambiguity for multimodal embedders, particularly in complex multihop scenarios~\cite{uniir}.
To address these, we introduce a two-step retrieval strategy: 
(i) \textit{LLM-driven query decomposition}, which explicitly generates modality-specific subqueries, and 
(ii) \textit{Late-interaction-guided graph traversal}, a beam-search traversal method dynamically scoring edges based on fine-grained interactions within the low-level nodes.






\subsubsection{LLM-driven Query Decomposition}

Given a potentially complex query $Q$, \texttt{LILaC} first leverages an LLM to explicitly decompose $Q$ into simpler modality-specific subqueries.
Specifically, we utilize a zero-shot prompting strategy to generate a small set of subqueries:
\vspace{-2mm}
\begin{equation}
\{q_1, \dots, q_{k_{sub}}\} = \text{LLM}(Q;\, prompt_{\mathrm{dec}})
    \vspace{-2mm}
\end{equation}
Each subquery is then classified into a modality label  
$m_j\!\in\!\{\texttt{text},\texttt{table},\texttt{image}\}$ with a second prompt:
\vspace{-2mm}
\begin{equation}
m_j \;=\; \text{LLM}(q_j;\, \textit{prompt}_{\mathrm{mod}})
    \vspace{-2mm}
\end{equation}
Using these labels, we obtain modality-specific embeddings $\mathbf{q}_j = f(q_j;\, m_j)$ for every subquery, while the original query is embedded coarsely as $\mathbf{Q} = f(Q;\, \varepsilon)$ to seed the initial candidate search.  
We denote the set of embedded subqueries as $\textbf{Q}_{\text{sub}} = \{\mathbf q_1,\dots,\mathbf q_{k_{\text{sub}}}\}$.
Full prompt templates appear in \S\ref{sec:prompt_templates}.





















\subsubsection{Late-interaction-guided Graph Traversal}
\label{sec:late_interaction}

At inference time, \texttt{LILaC} searches for a subgraph $\mathcal{G}'\!\subseteq\!\mathcal{G}$ that best matches the query.  
\texttt{LILaC} maintains a beam of size $b$ and iteratively identify a candidate subgraph $\mathcal{G}_t=({V}_t,{E}_t, \lambda, \tau)$ consisting of $b$ edges.
Initially, to efficiently narrow the search space from numerous candidate nodes, \texttt{LILaC} identifies a set of top-$b$ top-level nodes $V_0$ most relevant to the query.
\begin{equation}
{V}_{0}
  = \operatorname*{arg\,max}_{C\in V_0}^{b}
    \operatorname{sim}\!\bigl(\mathbf{Q},\mathbf{C}\bigr),
\quad
{E}_{0}= \{\} 
\end{equation}


\texttt{LILaC} then initiates iterative traversal of the graph starting from these candidate nodes. 
In each iteration, \texttt{LILaC} first expands the candidate nodes via one-hop traversal to consider adjacent nodes, dynamically computing query-relevance scores for all edges formed by these expansions. 
Subsequently, only the top-$b$ scored edges are retained for the next iteration forming subgraph, and their constituent nodes become the new set of candidate nodes, forming $\mathcal{G}_i = (V_i, E_i, \lambda, \tau)$. 
After the final iteration $n_i$, \texttt{LILaC} returns the top-$n_{ret}$ nodes from the final subgraph $\mathcal{G}_{n_i}$.

\begin{figure}[ht]
  \centering
  \includegraphics[width=\linewidth]{figures/late_interaction.pdf}
  \caption{An example case of edge-level late interaction.}
  \label{fig:late_interaction}
\end{figure}

\textbf{Late Interaction Edge Scoring.}
As previously discussed, naively calculating edge scores negatively impacts both effectiveness and efficiency. 
Specifically, this is because 
(1) subqueries, each potentially targeting distinct modalities, must accurately align with the relevant nodes, and 
(2) embedding all edges within the layered graph is inefficient due to their vast number.

To efficiently address these issues, \texttt{LILaC} employs a \emph{late interaction} strategy, scoring each edge on-the-fly with \emph{fine-grained} evidence.
\updated{\texttt{LILaC} extends the standard token-level late interaction to operate at the node-subquery level, by matching decomposed subqueries against the subcomponents contained within an edge.}
Let an edge be $e = (C_\alpha, C_\beta)$ and $\mathcal{S}_e = \mathcal{S}(C_\alpha) \cup \mathcal{S}(C_\beta)$.
\texttt{LILaC} gathers every subcomponent that could provide evidence on either side of the edge in the set $\mathcal{S}_e$.
\vspace{-3mm}
\begin{equation}
s(e;\textbf{Q}_{sub})
    \;=\;
    \sum_{\mathbf q\in\mathbf{Q}_{sub}}
        \max_{c\in\mathcal S_e}
        \operatorname{sim}\bigl(f(c),\mathbf q\bigr).
    \vspace{-3mm}
\label{eq:edge_score}
\end{equation}
The inner \(\max\) selects, for each sub-query \(\mathbf q\), the single most relevant sub-component \(\mathbf c\) incident to the edge, while the outer sum ensures every sub-query contributes exactly once. 
Figure~\ref{fig:late_interaction} shows two example cases of late interaction scoring.
This scoring approach is designed to reflect practical scenarios where each subquery specifically targets fine-grained details located within particular subcomponents.
By aggregating the maximum similarity scores across these detailed elements, rather than relying solely on coarse component embeddings, \texttt{LILaC} effectively prioritizes precise, subcomponent-level matches. 
This strategy enhances retrieval accuracy by focusing directly on relevant information, reducing the noise introduced by broader, less relevant contexts.



We introduce two special cases of edge scoring: 
\textit{(i) Isolated nodes.}  
If a component $C$ has no explicit neighbor, we introduce a dummy edge $(C,\varepsilon)$ so that $C$ can still be considered.
\textit{(ii) One-sided matches.}  
If an edge score $s(e;Q)$ equals the best single-node score of one endpoint, we return only that node to avoid including irrelevant neighbors.
Refer to Figure~\ref{fig:late_interaction} (b) for a specific example.

\section{Experiments} 
\label{sec:exp}
\vspace{-2mm}
\subsection{Experimental Setups}


\begin{table*}[t]
  \centering
  \setlength{\tabcolsep}{4pt}
  \renewcommand{\arraystretch}{1.2}
  \scriptsize

  % ──────────────────────────────────────────────────────────────────
  \begin{tabular}{@{}l|c|cc|cc|cc|cc|cc@{}}
    \toprule
    \multirow{3}{*}{\textbf{Algorithm}}
        & \multirow{3}{*}{\textbf{Embedder Type}}
        & \multicolumn{2}{c|}{\textbf{\texttt{MP-DocVQA}}}
        & \multicolumn{2}{c|}{\textbf{\texttt{SlideVQA}}}
        & \multicolumn{2}{c|}{\textbf{\texttt{InfoVQA}}}
        & \multicolumn{2}{c|}{\textbf{\texttt{MultimodalQA}}}
        & \multicolumn{2}{c@{}}{\textbf{\texttt{MMCoQA}}} \\[-1.5pt]
    \cmidrule(lr){3-12}
        &        & R@3 & MRR@10
                 & R@3 & MRR@10
                 & R@3 & MRR@10
                 & R@3 & MRR@10
                 & R@3 & MRR@10 \\
    \midrule
    % ===================== Baselines ===============================
    \texttt{NV-Embed-v2}  & Text 
                 & 67.85 & 61.91 
                 & 88.49 & 79.55 
                 & 86.21 & 80.86
                 & 60.19 & 67.86 
                 & 46.16 & 41.45 \\
    \midrule
    % ---------------- VisRAG & ColPali (merged Image) --------------
    \texttt{VisRAG-Ret} & \multirow{2}{*}{Image} 
                 & \cellcolor{orange!20}{83.25} & \cellcolor{orange!20}{75.55}
                 & \cellcolor{orange!20}{91.55} & \cellcolor{orange!20}{84.30}
                 & \cellcolor{orange!20}{92.76} & \cellcolor{orange!20}{86.22}
                 & 50.08 & 55.08
                 & 27.63 & 23.75 \\[-1pt]
    \texttt{ColPali}      &  
                 & 80.71 & 74.86 
                 & 89.39 & 81.55 
                 & 88.30 & 82.76
                 & 58.73 & 65.05 
                 & 36.24 & 32.33 \\
    \midrule
    % ---------------------- DraGO variants -------------------------
    \textbf{\texttt{LILaC} (w/ \texttt{mmE5})} & \multirow{3}{*}{Multimodal}
                 & 61.25 & 55.30 
                 & 77.52 & 68.80 
                 & 75.09 & 69.86
                 & 54.79 & 59.02 
                 & 48.88 & 40.30 \\[-1pt]
    \textbf{\texttt{LILaC} (w/ \texttt{UniME})} &  
                 & 77.83 & 71.42 
                 & 84.35 & 77.93 
                 & 82.28 & 75.27
                 & 58.52 & 61.44 
                 & 49.63 & 42.97 \\[-1pt]
    \textbf{\texttt{LILaC} (w/ \texttt{MM-Embed})} &  
                 & \textbf{83.59} & \textbf{78.75} 
                 & \textbf{92.81} & \textbf{84.43} 
                 & \textbf{93.17} & \textbf{86.83}
                 & \textbf{69.07} & \textbf{75.28} 
                 & \textbf{55.80} & \textbf{50.77} \\
    \bottomrule
  \end{tabular}
  \vspace{-1mm}
    \caption{
    \updated{Retrieval accuracy (Recall@3 (\textit{R@3}) and \textit{MRR@10}) of \texttt{LILaC} and its competitors on five benchmarks.  
  The best score in each column is in \textbf{bold}.  
  The in-domain fine-tuned settings are colored in \colorbox{orange!20}{orange}.}
  }
  \vspace{-3mm}
  \label{tab:retrieval_accuracy}
\end{table*}


\begin{table*}[t]
  \centering
  \setlength{\tabcolsep}{4pt}
  \renewcommand{\arraystretch}{1.2}
  \scriptsize

  % ──────────────────────────────────────────────────────────────────
  \begin{tabular}{@{}l|c|cc|cc|cc|cc|cc@{}}
    \toprule
    \multirow{3}{*}{\textbf{Algorithm}}
        & \multirow{3}{*}{\textbf{MLLM}}
        & \multicolumn{2}{c|}{\textbf{\texttt{MP-DocVQA}}}
        & \multicolumn{2}{c|}{\textbf{\texttt{SlideVQA}}}
        & \multicolumn{2}{c|}{\textbf{\texttt{InfoVQA}}}
        & \multicolumn{2}{c|}{\textbf{\texttt{MultimodalQA}}}
        & \multicolumn{2}{c}{\textbf{\texttt{MMCoQA}}} \\[-1.5pt]
        
    \cmidrule(lr){3-12}
            &    & EM & F1
                 & EM & F1
                 & EM & F1
                 & EM & F1
                 & EM & F1 \\
    \midrule

    \texttt{NV-Embed-v2}  
                 & \texttt{Qwen2.5-VL 7B} 
                 & 56.51 & 63.16 
                 & 53.77 & 64.41 
                 & 60.72 & 63.40 
                 & 37.23 & 43.85 
                 & 28.05 & 34.67 \\[-1pt]
    \midrule
    \texttt{VisRAG-Ret}
                 & \texttt{MiniCPM V2.6}
                 & \cellcolor{orange!20}54.31 & \cellcolor{orange!20}68.86 
                 & \cellcolor{orange!20}43.88 & \cellcolor{orange!20}62.37 
                 & \cellcolor{orange!20}50.83 & \cellcolor{orange!20}57.55
                 & 28.18 & 34.01 
                 & 21.51 & 27.87 \\[-1pt]
    \texttt{VisRAG-Ret}  & \texttt{Qwen2.5-VL 7B} 
                 & \cellcolor{orange!20}65.34 & \cellcolor{orange!20}72.24 
                 & \cellcolor{orange!20}55.03 & \cellcolor{orange!20}66.13 
                 & \cellcolor{orange!20}60.16 & \cellcolor{orange!20}61.93
                 & 22.24 & 25.55 
                 & 16.69 & 20.90 \\[-1pt]
    \texttt{ColPali}      
                 & \texttt{Qwen2.5-VL 7B} 
                 & 64.46 & 71.16 
                 & 53.77 & 64.54 
                 & 58.07 & 60.38 
                 & 23.59 & 27.37 
                 & 18.07 & 22.30 \\[-1pt]
    \midrule
    
    % \textbf{Ours (w/ QQMM)}  & Multimodal 
    %              & $\ldots$ & $\ldots$ & $\ldots$ & $\ldots$ & $\ldots$ & $\ldots$
    %              & 40.96 & 47.34 & 34.42 & 40.95 \\[-1pt]

    \textbf{\texttt{LILaC} (w/ \texttt{mmE5})}  
                 & \texttt{Qwen2.5-VL 7B} 
                 & 52.96 & 59.53 
                 & 50.89 & 59.07 
                 & 50.12 & 53.12
                 & 40.72 & 47.46 
                 & 33.90 & 40.38 \\[-1pt]

    \textbf{\texttt{LILaC} (w/ \texttt{UniME})} 
                 & \texttt{Qwen2.5-VL 7B} 
                 & 62.43 & 69.40 
                 & 53.05 & 62.89 
                 & 53.39 & 56.86
                 & 43.42 & 49.72 
                 & 33.39 & 40.12 \\[-1pt]
                 
    \textbf{\texttt{LILaC} (w/ \texttt{MM-Embed})} 
                 & \texttt{Qwen2.5-VL 7B} 
                 & \textbf{65.48} & \textbf{72.42}
                 & \textbf{55.57} & \textbf{66.32} 
                 & \textbf{60.91} & \textbf{62.87}
                 & \textbf{44.57} & \textbf{51.97} 
                 & \textbf{36.31} & \textbf{43.22} \\[-1pt]
    \bottomrule
  \end{tabular}
  \vspace{-1mm}
    \caption{
    \updated{End-to-end accuracy (EM and F1) of \texttt{LILaC} and its competitors for the 5 benchmarks.  
  The best score in each column is in \textbf{bold}.
  Generation results corresponding to in-domain fine-tuned settings are colored in \colorbox{orange!20}{orange}.}
  }
  \label{tab:end2end_accuracy}
  \vspace{-3mm}
\end{table*}





















\quad 
\textbf{Datasets \& Evaluation Metrics. }
We evaluate on total five benchmarks.
Three are \texttt{VisRAG}-extended open-domain VQA datasets—\texttt{MP-DocVQA}~\cite{mpdocvqa} (industrial documents), \texttt{SlideVQA}~\cite{slidevqa}(presentation slides with multi-hop queries), and \texttt{InfoVQA}~\cite{infovqa} (infographics).
For a realistic webpage retrieval setting, we extend multimodal QA benchmarks (\texttt{MultimodalQA}~\cite{multimodalqa}, \texttt{MMCoQA}~\cite{mmcoqa}) using \texttt{M3DocRAG}'s methodology~\cite{m3docrag}. 
Specifically, we reconstruct webpages from URLs annotated in each component label. 
\texttt{MultimodalQA} comprises 3,235 webpages, each averaging approximately 37 components, corresponding to about 12 PDF pages.
\texttt{MMCoQA} comprises 453 webpages, each averaging approximately 32 components, 11 PDF pages.


Following \texttt{VisRAG}, we evaluate retrieval using Mean Reciprocal Rank at 10 (MRR@10). 
Additionally, we include Recall@3 to assess whether the retrieval component successfully captures relevant information within the top three components, aligning with \texttt{VisRAG}'s experimental design that inputs three components to the generation model.
Further details are explained in \textsection~\ref{sec:implementation_details}.


\textbf{Compared Methods.} 
We employ two SOTA methods of VisRAG approaches - \texttt{VisRAG}, which directly encodes document images via VLMs~\cite{visrag}, and \texttt{ColPali}, which employs late-interaction multi-vector embeddings from document images~\cite{colpali}.
We additionally compare with \texttt{NV-Embed-v2}, a SOTA TextRAG method reported by \texttt{VisRAG}. It utilizes a 7.85B model for embedding textualized components.



\textbf{Applied Multimodal Embedding Models. }
We use three multimodal embedders: \texttt{MM-Embed}~\cite{mmembed}, \texttt{UniME}~\cite{unime} and \texttt{mmE5}~\cite{mme5}.
Details about the embedding models can be further found in \textsection~\ref{sec:appendix_model_details}.














\subsection{Retrieval Accuracy Comparison}
\label{sec:retrieval_accuracy_comparison}

\begin{table*}[t]
  \centering
  \setlength{\tabcolsep}{4pt}
  \renewcommand{\arraystretch}{1.2}
  \scriptsize

  % ──────────────────────────────────────────────────────────────────
  \begin{tabular}{@{}l|l|cc|cc|cc|cc|cc@{}}
    \toprule
    \multirow{3}{*}{\textbf{Embedder Model}}
        & \multirow{3}{*}{\textbf{Variant}}
        & \multicolumn{2}{c|}{\textbf{\texttt{MP-DocVQA}}}
        & \multicolumn{2}{c|}{\textbf{\texttt{SlideVQA}}}
        & \multicolumn{2}{c|}{\textbf{\texttt{InfoVQA}}}
        & \multicolumn{2}{c|}{\textbf{\texttt{MultimodalQA}}}
        & \multicolumn{2}{c }{\textbf{\texttt{MMCoQA}}} \\[-1.5pt]
    \cmidrule(lr){3-12}
        &   & R@3 & MRR@10
            & R@3 & MRR@10
            & R@3 & MRR@10
            & R@3 & MRR@10
            & R@3 & MRR@10 \\
    \midrule

    % % =====================  (b)  ===================================
                 
    % Screenshot kNN & Multimodal 
    %              & 20.30 & 18.09 
    %              & 63.31 & 54.02 
    %              & 56.55 & 48.02
    %              & 19.71 & 19.96 
    %              & 16.78 & 14.46 \\[-1pt]
                 
    
    \multirow{3}{*}{\texttt{mmE5}} 
                & \texttt{LILaC (w/o LCG \& QD)}
                 & 48.90 & 43.97 
                 & 75.91 & 68.13 
                 & 65.60 & 58.55
                 & 42.99 & 46.92 
                 & 41.22 & 34.51 \\[-1pt]
                 
                 & \texttt{LILaC (w/o QD)}
                 & 60.81 & 55.02 
                 & 74.14 & 67.58 
                 & 67.70 & 60.01
                 & 45.15 & 51.12
                 & 44.18 & 36.62
                 \\[-1pt]
                 
                & \textbf{\texttt{LILaC}} 
                 & \textbf{61.25} & \textbf{55.35}
                 & \textbf{76.80} & \textbf{68.99}
                 & \textbf{68.91} & \textbf{61.18}
                 & \textbf{54.78} & \textbf{59.32}
                 & \textbf{48.54} & \textbf{40.22} \\[-1pt]

    % \midrule

    % \rowcolor{blue!10}
    % \multicolumn{12}{c}{\textbf{\texttt{UniME}}} \\

    \midrule
                 
    % Screenshot kNN & Multimodal 
    %              & 19.46 & 16.91 
    %              & 68.71 & 59.23 
    %              & 69.64 & 62.20
    %              & 28.63 & 30.00 
    %              & 25.83 & 23.03 \\[-1pt]
                 
    \multirow{3}{*}{\texttt{UniME}} 
                 & \texttt{LILaC (w/o LCG \& QD)}
                 & 52.12 & 45.31 
                 & 81.47 & 71.22 
                 & 83.57 & 77.07
                 & 47.68 & 49.06 
                 & 45.78 & 38.41 \\[-1pt]
                 
                 & \texttt{LILaC (w/o QD)}
                 & 77.83 & 71.27 
                 & 83.45 & 75.70 
                 & 85.11 & 78.01
                 & 52.18 & 54.01
                 & 47.11 & 39.85
                 \\[-1pt]
                 
                & \textbf{\texttt{LILaC}} 
                 & \textbf{77.83} & \textbf{71.39}
                 & \textbf{84.35} & \textbf{77.93}
                 & \textbf{85.53} & \textbf{78.81}
                 & \textbf{58.43} & \textbf{61.32}
                 & \textbf{49.45} & \textbf{42.91} \\[-1pt]

    \midrule

    % \rowcolor{blue!10}
    % \multicolumn{12}{c}{\textbf{\texttt{MM-Embed}}} \\

    % \midrule
                 
    % Screenshot kNN & Multimodal 
    %              & 47.21 & 42.01 
    %              & 78.78 & 70.29 
    %              & 57.38 & 51.80
    %              & 45.78 & 49.14 
    %              & 28.03 & 24.15 \\[-1pt]
                 
    \multirow{3}{*}{\texttt{MM-Embed}} 
                 & \texttt{LILaC (w/o LCG \& QD)}
                 & 75.80 & 69.09 
                 & 92.80 & 82.19 
                 & 90.39 & 83.71
                 & 61.10 & 67.35 
                 & 47.94 & 43.75 \\
                 
                 & \texttt{LILaC (w/o QD)}
                 & 82.23 & 77.75 
                 & 92.27 & 83.20 
                 & 92.17 & 85.53
                 & 63.19 & 69.91
                 & 50.18 & 45.59
                 \\[-1pt]
                 
                 & \textbf{\texttt{LILaC}} 
                 & \textbf{83.59} & \textbf{78.75}
                 & \textbf{92.81} & \textbf{84.43}
                 & \textbf{93.17} & \textbf{86.83}
                 & \textbf{69.07} & \textbf{75.28}
                 & \textbf{55.80} & \textbf{50.77} \\

    \bottomrule
  \end{tabular}
  \vspace{-1mm}
  \caption{
    \updated{Ablation study analyzing retrieval accuracy (Recall@3 and MRR@10) of different \texttt{LILaC} variants.
    Best scores per embedder and dataset are highlighted in bold (\texttt{LCG} = Layered Component Graph, \texttt{QD} = Query Decomposition).
    }
  }
  \label{tab:ablation_study}
  \vspace{-3mm}
\end{table*}



We evaluated retrieval accuracies using Recall@3 (R@3) and MRR@10 across five benchmarks. 
Table~\ref{tab:retrieval_accuracy} summarizes the retrieval performance of \texttt{LILaC} and competing methods. 
\updated{Our results indicate that \texttt{LILaC} achieves state-of-the-art (SOTA) performance on all five benchmarks.
Notably, \texttt{LILaC} outperforms the previous VisRAG SOTA models, \texttt{VisRAG-Ret} and \texttt{ColPali}, by substantial margins of 14.24\% and 11.62\% in R@3, and 15.75\% and 11.74\% in MRR@10, on average, respectively. 
These performance gains are especially prominent on datasets that inherently require fine-grained and multihop reasoning (\texttt{MultimodalQA} and \texttt{MMCoQA}), where the relative improvements in average Recall@3 reached 60.68\% and 31.49\%, and MRR@10 improved by 59.90\% and 45.92\%, respectively.}


Our analysis highlights two key findings:
(i) TextRAG of \texttt{NV-Embed-v2}, consistently shows the lowest retrieval accuracy on visually-dependent VQA datasets that include plots and charts, highlighting inherent limitations in handling visual modalities.
(ii) VisRAG methods notably struggle in webpage retrieval settings (\texttt{MultimodalQA}, \texttt{MMCoQA}), underperforming even when compared to the text-based \texttt{NV-Embed-v2}.
Specifically, the stronger VisRAG model, \texttt{ColPali}, showed accuracy drops against \texttt{NV-Embed-v2}, with reductions of 10.70\% in Recall@3 and 20.96\% in MRR@10. 
% (iii) Finally, \texttt{LILaC} underperformed VisRAG methods on \texttt{InfoVQA}, achieving R@3 and MRR@10 scores lower by 6.3\% and 4.16\% than \texttt{VisRAG-Ret}, respectively. 
% Our subsequent analysis attributes this specific gap primarily to suboptimal subcomponent detection within image components in \texttt{InfoVQA}, leading to ineffective late interaction. 















\subsection{End-to-end Accuracy Comparison}

% Paragraph 1: Experiment setting
We conducted end-to-end question answering (QA) experiments to analyze the impact of retrieval accuracy on downstream QA performance. 
The retrieved results were directly input into a multimodal LLM generator for answer generation, primarily using the \texttt{Qwen2.5-VL 7B} model~\cite{qwen2_5}. 
We limited the number of retrieved units fed into the generator to 3, consistent with the experimental setup of \texttt{VisRAG}.
We additionally provide the results from \texttt{MiniCPM V2.6} for comprehensive comparison, following the original \texttt{VisRAG} pipeline. 
Applied prompts are detailed in \textsection~\ref{sec:prompt_templates}.


% Paragraph 2: Experiment results
\updated{Table~\ref{tab:end2end_accuracy} shows that \texttt{LILaC} achieves SOTA end-to-end accuracy on every benchmark, with average EM and F1 scores of 52.56\% and 59.36\%, respectively. 
This represents substantial improvements of 18.67\% and 19.62\% compared to the previously best-performing VisRAG setup, \texttt{VisRAG} with \texttt{Qwen2.5-VL}, which scored 44.29\% in EM and 49.62\% in F1.}
Overall, the end-to-end QA accuracy trends closely align with retrieval accuracy. 
However, a notable exception arises.
Interestingly, despite \texttt{LILaC (w/ mmE5)} having approximately 8.97\% lower retrieval accuracy (R@3) compared to \texttt{NV-Embed-v2}, its EM score surpasses \texttt{NV-Embed-v2} by 19.71\%. 
This divergence highlights the significant information loss inherent to TextRAG methods, which convert visual content entirely into text, underscoring the importance of preserving visual modalities for effective QA. 
% Second, \texttt{LILaC (w/ MM-Embed)} fails to attain SOTA on \texttt{MP-DocVQA}, although it scores the highest retrieval accuracies. 










\subsection{Ablation Study}


% Paragraph 1: 어떠한 variant들이 있는지
We performed an ablation study to assess the individual contributions of each key component in our framework to retrieval accuracy.
\updated{
Specifically, we evaluated two variants of \texttt{LILaC} across all three multimodal embedding models.
\texttt{LILaC (w/o QD)} is a variant of \texttt{LILaC} without its query decomposition module, adn thus not incorporating the late interaction score mechanism.
It instead incorporates a two-stage retrieval approach on the layered graph: it first selects the top $b$-nearest neighbor components at the coarse level, and then reranks these components by considering subcomponent-level relevance scores.
\texttt{LILaC (w/o LCG \& QD)} further discards the layered component graph (\texttt{LCG}) structure.
It directly applies a k-nearest neighbor search on individual top-layer components without leveraging finer-grained subcomponents.
}

\updated{
Table~\ref{tab:ablation_study} indicates that incorporating the layered component graph to the simple baseline (\texttt{LILaC (w/o LCG \& QD)}) shows notable average improvements—7.33\% in R@3 and 10.13\% in MRR@10.
Further integrating query decomposition with the late interaction mechanism to \texttt{LILaC (w/o QD)} completes the \texttt{LILaC} algorithm, yielding incremental gains of 3.19\% in R@3 and 4.7\% in MRR@10.
}
While these improvements seem modest, closer inspection reveals significant benefits in datasets requiring complex multihop reasoning, particularly \texttt{MultimodalQA} and \texttt{MMCoQA}. 
Specifically, incorporating \texttt{QD} improves R@3 by an average of 7.40\% and MRR@10 by 10.70\% for these two datasets.
Overall, \texttt{LILaC} is demonstrated to be a generalizable method, evidenced by its consistent performance improvements across all multimodal datasets and embedding models.
This robust trend underscores \texttt{LILaC}'s ability to universally enhance retrieval performance across a variety of multimodal embedding scenarios.






\vspace{-1mm}
\subsection{Algorithm Execution Time}
\vspace{-1mm}





Figure~\ref{fig:algorithm_execution_time} (a) shows the average retrieval and generation times for each algorithm.
\texttt{LILaC} is approximately 20.76\% slower than \texttt{VisRAG}, yet 18.24\% faster than \texttt{ColPali}. 
Despite employing a unigranular retrieval approach, \texttt{ColPali}'s runtime remained slower due to its inherent complexity from multi-vector embedding methods.
Notably, both VisRAG methods had longer generation times compared to ours.
\texttt{VisRAG} required 1.70$\times$, and \texttt{ColPali} 1.15$\times$ times our average generation runtime, primarily because their pixel-heavy image inputs increased MLLM inference times.

Figure~\ref{fig:algorithm_execution_time} (b) presents the detailed runtime breakdown for \texttt{LILaC}, showing a total average runtime of 3,047 ms. 
Remarkably, the late-interaction-based subgraph retrieval step accounts for only about 48 ms (approximately 1.5\% of the total runtime). 
The major performance bottleneck lies in the query decomposition phase, averaging 1,423 ms. 
Since this step relies on advanced reasoning with the computationally heavy \texttt{Qwen2.5 72B} model, future improvements in runtime efficiency could be realized by utilizing lighter models, thus balancing speed and retrieval accuracy more effectively.


\begin{figure}[t]
\centering
\small
% \vspace{-4mm}
\begin{tabular}{@{}c@{}c@{}}
% \noalign{\vskip -0.1mm}
    { \includegraphics[width=.5\columnwidth]{figures/plot_runtime/Average.pdf}} &
    { \includegraphics[width=.5\columnwidth]{figures/plot_breakdown/Average.pdf}} \\
    \noalign{\vskip -1.3mm}
    (a) Average  runtime &
    (b) Breakdown of \texttt{LILaC} \\
    \end{tabular}
    \vspace{-1mm}
\caption{(a) Comparison of average algorithm execution times across different methods, and (b) detailed runtime breakdown of \texttt{LILaC}.}
\label{fig:algorithm_execution_time}
\end{figure}


\vspace{-1mm}
\subsection{Query Decomposition Analysis}
\vspace{-1mm}

\label{sec:exp_query_decomp}
\updated{We evaluate the query decomposition module in isolation and its impact on retrieval accuracy. 
Because benchmarks do not provide gold subqueries, we approximate decomposition quality via the Jaccard similarity between the \emph{predicted} modality set $\widehat{\mathcal{M}}(q)$ (union of modalities assigned to generated subqueries) and the \emph{gold} modality set $\mathcal{M}^\star(q)$ derived from ground-truth components (where $q$ is a query).
Specifically, the score for a query $q$ is $J(q)=\frac{|\widehat{\mathcal{M}}(q)\cap\mathcal{M}^\star(q)|}{|\widehat{\mathcal{M}}(q)\cup\mathcal{M}^\star(q)|}$, and the final accuracy is obtained by averaging over all queries.}

\updated{We compare Qwen2.5 (8B, 72B) and Llama3.1 (7B, 70B) using the prompts of \S\ref{sec:prompt_templates}, keeping all other components fixed. 
In table~\ref{tab:query_decomp_llms}, we report (i) decomposition accuracy, (ii) Recall@3 (R@3), and (iii) decomposition runtime (time (ms)) on \texttt{MultimodalQA} and \texttt{MMCoQA}, as they are the only datasets with $\mathcal{M}^\star(q)$ labeled.
}


\begin{table}[t]
    \centering
    \setlength{\tabcolsep}{4pt}
    \renewcommand{\arraystretch}{1.2}
    \footnotesize
    \begin{tabular}{@{}c|c|c|c|c@{}}
        \toprule
        
        \textbf{LLM} & \textbf{Params} & \textbf{DAcc (\%)} & \textbf{R@3 (\%)} & \textbf{Time (ms)} \\
        
        \midrule
        
        \multirow{2}{*}{Qwen2.5}    & 8B & 63.29 & 59.01 & 258 \\
        \cmidrule(lr){2-5}
        
                                    & 72B & 72.23 & 62.43 & 1849 \\
        \cmidrule(lr){1-5}
        
        \multirow{2}{*}{Llama3.1}   & 7B & 58.11 & 57.44 & 336 \\
        \cmidrule(lr){2-5}
        
                                    & 70B & 66.34 & 61.24 & 1731 \\
                                    
        \bottomrule
    \end{tabular}
    \vspace{-1mm}
    \caption{Query decomposition analysis result on \texttt{MultimodalQA} and \texttt{MMCoQA} datasets.}
    \label{tab:query_decomp_llms}
\end{table}

\updated{
We notice that the LLM-driven decomposition attains reasonable accuracy of 72.23\%, and also that larger models improve both decomposition and retrieval at higher latency. 
Notably, decomposition accuracy and Recall@3 are strongly correlated across model variants (Pearson $\rho{=}0.954$), underscoring that better query decomposition directly benefits retrieval.
}

\vspace{-1mm}
\subsection{Additional Experiments}
\vspace{-1mm}

\updated{
Additional experiments were conducted, but are detailed in the appendix due to space limitations.
These include (i) parameter sensitivity (\textsection~\ref{sec:parameter_sensitivity}), 
(ii) analysis of offline layered component graph construction runtime (\textsection~\ref{sec:lcg_construction_overhead}), and 
(iii) a comparison of retrieval accuracy across different embedder models (\textsection~\ref{sec:embedder_comparison}).
}

\vspace{-1mm}
\section{Conclusion}
\vspace{-1mm}


We presented \texttt{LILaC}, a multimodal retrieval framework designed to address the limitations of existing methods by incorporating layered component graph and late-interaction-based subgraph retrieval.
Our layered graph construction explicitly captures semantic relationships among multimodal components, facilitating effective multihop reasoning. 
The late-interaction retrieval method dynamically evaluates fine-grained component relevance, significantly enhancing retrieval accuracy, yet efficient.
\updated{\texttt{LILaC}'s usage of pretrained multimodal encoders allows it to inherit the improvements from newer off-the-shelf embeddings.
Extensive experiments confirm that \texttt{LILaC} consistently outperforms state-of-the-art approaches across all five benchmarks, also demonstrating its broad applicability and effectiveness in open-domain multimodal retrieval.}

\section{Limitations}


Our current approach focuses on effectively harmonizing pre-trained multimodal models to achieve enhanced retrieval performance without additional fine-tuning. 
Consequently, the accuracy of our retrieval method significantly depends on the quality of subcomponent extraction. %, especially within image and table modalities.
% As demonstrated in our empirical analysis (e.g., with the \texttt{InfoVQA} dataset), inaccuracies during subcomponent extraction can negatively affect retrieval quality. 
Also, although our retrieval accuracy surpasses existing methods, there remains substantial room for improvement in end-to-end generation tasks. 
% This highlights the necessity of developing more sophisticated integration strategies between retrieval and generation components.



\section*{Acknowledgements}

\updated{
This work was partly supported by the National Research Foundation of Korea(NRF) grant funded by the Korea government(MSIT) (RS-2025-00517736, 50\%), Institute of Information \& communications Technology Planning \& Evaluation (IITP) grant funded by the Korea government(MSIT) (No. RS-2024-00509258, Global AI Frontier Lab, 30\%) (No. RS-2018-II181398, Development of a Conversational, Self-tuning DBMS, 10\%) (No.  RS-2024-00454666, Developing a Vector DB for Long-Term Memory Storage of Hyperscale AI Models, 5\%), and Basic Science Research Program through the National Research Foundation of Korea Ministry of Education(No. RS-2024-00415602, 5\%).
}

% 우리는 현재 존재하는 pre-trained multimodal model들을 effective하게 사용하는 방법에 대해 approach한다.
% Future work에는,  domain-specific하게 late interaction을 학습시키는 방식에 대해 연구하고자 한다.
% 또한, 우리의 방식은 subcomponent extraction method의 성능에 영향을 받는다. 
% Retrieval accuracy는 높지만, end-to-end generation을 위한 room이 존재한다.

% Our current approach focuses on effectively leveraging existing pre-trained multimodal embedding models without fine-tuning. 
% Consequently, the performance of \texttt{LILaC} inherently relies on the quality and generalizability of these embedding models. 
% An important direction for future research involves domain-specific fine-tuning of the late interaction component, which could further enhance retrieval precision by better capturing domain nuances. 
% Our current approach focuses on harmonizing multimodal models in their pre-trained nature to 달성하다 더 좋은 effectivity를.
% Thus, the accuracy of our retrieval method significantly depends on the subcomponent extraction process, particularly for image and table modalities. 
% Imperfections or inaccuracies during subcomponent extraction can directly impact retrieval quality, as seen in our empirical analysis with certain datasets (e.g., InfoVQA).
% Lastly, while our retrieval accuracy outperforms existing approaches, there remains substantial room for improvement in end-to-end generation tasks, emphasizing the need for more sophisticated integration between retrieval and generation components.

% Bibliography entries for the entire Anthology, followed by custom entries
%\bibliography{anthology,custom}
% Custom bibliography entries only
\bibliography{custom}

\newpage

\appendix

\section*{Appendix}
\setcounter{section}{0}
\renewcommand\thesection{\Alph{section}}



\section{Software and Data Licenses}

The licenses for the software and datasets used in this paper as follows:

\squishlist
    \item \texttt{VisRAG-Ret}: Apache-2.0
    \item \texttt{ColPali}: PaliGemma License, MIT License
    \item \texttt{MiniCPM-v2.6}: Apache-2.0
    \item \texttt{Qwen2.5-VL 7B}: Apache-2.0
    \item \texttt{Qwen2.5 72B}: Qwen
    \item \texttt{MM-Embed}: CC-BY-NC-4.0
    \item \texttt{NV-Embed-v2}: CC-BY-NC-4.0
    \item \texttt{UniME}: MIT License
    \item \texttt{mmE5}: MIT License
\squishend


All software and datasets were used strictly for research purposes and were not utilized in any non-research contexts, particularly for 
commercial applications. 



\section{AI Assistants}


We implemented our code efficiently using ChatGPT-o3~\cite{jaech2024openai}, enabling rapid debugging and effective error resolution. 
Additionally, we revised our paper using ChatGPT-4.5, which helped us enhance sentence clarity and readability through iterative rephrasing.






\section{Reproducibility Statement}

\texttt{VisRAG-Ret} was reproduced using the official code available at \href{https://github.com/OpenBMB/VisRAG}{\textcolor{linkpink}{\texttt{VisRAG official github}}}.

\texttt{ColPali} and \texttt{NV-Embed-v2} were implemented applying their official model cards introduced in  \href{https://huggingface.co/vidore/colpali}{\textcolor{linkpink}{\texttt{ColPali huggingface}}} and \href{https://huggingface.co/nvidia/NV-Embed-v2}{\textcolor{linkpink}{\texttt{NV-Embed-v2 huggingface}}}, respectively.
The source code, data, and other artifacts for \texttt{LILaC} have been made available at \href{https://github.com/joohyung00/lilac}{\textcolor{linkpink}{\texttt{our github repository}}}.




\section{Model Details}
\label{sec:appendix_model_details}


\textbf{(Multimodal) Large language models:}
\squishlist
    \item \texttt{Qwen2.5 72B}: 72B parameters
    \item \texttt{Qwen2.5-VL 7B}: 7B parameters
    \item \texttt{MiniCPM-v2.6}: 8.1B parameters
\squishend

\noindent \textbf{Text embedders}
\squishlist
    \item \texttt{NV-Embed-v2}: 7.85B parameters
\squishend

\noindent \textbf{Cross-modal embedders:}
\squishlist
    \item \texttt{ColPali}: 3B parameters
    \item \texttt{VisRAG-Ret}: 3.43B parameters
\squishend


\noindent \textbf{Multimodal embedders:}
\squishlist
    \item \texttt{MM-Embed}: 8.18B parameters
    \item \texttt{UniME}: 7.57B parameters
    \item \texttt{mmE5}: 10.6B parameters 
\squishend
\texttt{MM-Embed} is fine-tuned via modality-aware hard negative mining~\cite{mmembed}.
\texttt{UniME} is enhanced with textual discriminative knowledge distillation and instruction-tuned hard negatives~\cite{unime}.
\texttt{mmE5} leverages  synthetic multilingual data for robust cross-modal alignment~\cite{mme5}.










\section{Experiment Supplementaries}

\subsection{Hardware and Software Settings}
All our experiments were conducted on a system with an Intel Xeon Gold 6230 GPU @ 2.10GHz, 1.5TB of RAM, and four NVIDIA RTX A6000 GPUs.



\subsection{Implementation Details}
\label{sec:implementation_details}

We set the default hyperparameters for all experiments as beam width $b$ = 30 and number of iterations $n_i$ = 1.
Additionally, for the ablation study that exclusively uses the layered graph structure without late interaction, we also maintained an identical beam width ($b$ = 30) to ensure a fair comparison.

All experiments were conducted with `temperature = 0' and `do\_sample = False'. 
To further ensure fair comparison, we aligned the ratio of components between the VisRAG methods and our approach to approximately 1:3, as justified by the empirical observation that a typical screenshot in our datasets encompasses roughly three distinct multimodal components. 
Specifically, the \texttt{MultimodalQA} dataset contains 39,093 screenshots and 122,521 components, and the \texttt{MMCoQA} dataset comprises 5,175 screenshots and 14,493 components, both yielding a component-to-screenshot ratio close to 3:1.





\subsection{Benchmark Details}

\quad \texttt{\textbf{MP-DocVQA}}: It is a multimodal visual question answering benchmark designed for industrial documents. 
It includes challenging questions that require extracting and reasoning over textual and visual information such as tables, figures, and charts found in documents. 
The development set contains 591 questions sourced from a corpus of 741 multimodal document pages.

\texttt{\textbf{SlideVQA}}: It focuses on extracting information from presentation slides and often requires multihop reasoning across multiple slides. 
It emphasizes the capability to handle diverse layouts and structured textual information commonly found in presentations. 
The \texttt{SlideVQA} development set comprises 556 questions, with the corpus containing 1,284 slide pages.

\texttt{\textbf{InfoVQA}}: It targets visual question answering on infographics, which blend images, charts, and textual descriptions. 
This dataset presents complex multimodal reasoning tasks where models must interpret visual elements combined with succinct textual explanations. 
Its development set includes 718 questions drawn from a corpus of 459 infographic pages.


\texttt{\textbf{MultimodalQA}}: It refers to the extended version of \texttt{MultimodalQA}, with its extension methodology introduced in M3DocRAG~\cite{m3docrag}.
The dataset covers a wide variety of document types, including texts, images, and tables, requiring complex multihop reasoning across multiple documents. 
Its evaluation set comprises 2,441 questions from over 3,368 PDF documents totaling approximately 41,005 pages.


\texttt{\textbf{MMCoQA}}: It is a conversational multimodal question-answering dataset aimed at testing a system’s ability to handle multimodal information across multiple turns in a conversational context. 
This dataset is also an extension of the \texttt{MMCoQA} dataset, which originally operates in a distractor setting.
It involves coherent, multi-turn question sequences requiring integration of information from text, images, and tables. 
The dataset includes 5,753 questions organized into 1,179 conversational dialogues. 
Its corpus consists of 218,285 textual passages, 10,042 tables, and 57,058 images.


\begin{table*}[t]
    \centering
    \small
    \begin{tabular}{l|c|c|c|c|c}
        \toprule
        \textbf{Step} & \textbf{20\%} & \textbf{40\%} & \textbf{60\%} & \textbf{80\%} & \textbf{100\%} \\
        \midrule
        Node Generation       & 2m 8s   & 3m 53s  & 6m 20s  & 8m 29s  & 10m 20s \\
        Edge Generation       & 38s     & 1m 6s   & 1m 46s  & 2m 18s  & 2m 54s  \\
        Embedding Generation  & 24m 27s & 47m 44s & 1h 15m 31s & 1h 40m 42s & 2h 2m 43s \\
        \midrule
        \textbf{Total}        & \textbf{27m 13s} & \textbf{52m 43s} & \textbf{1h 23m 37s} & \textbf{1h 51m 29s} & \textbf{2h 15m 57s} \\
        \bottomrule
    \end{tabular}
    \vspace{-1mm}
    \caption{
    \updated{Average offline construction time by corpus fraction. }
    }
    \label{tab:lcg_offline_runtime}
    \vspace{-2mm}
\end{table*}



\begin{table*}[t]
    \centering
    \small
    \begin{tabular}{l|c|c|c}
        \toprule 
        \textbf{Model} & \textbf{Params} & \textbf{Recall@3 (\%)} & \textbf{MRR@10 (\%)} \\
        \midrule
        \texttt{MM-Embed}
            & 8B  & 78.89 & 75.18 \\   \cmidrule(lr){1-4}        
        \texttt{UniME (LLaVA-OneVision)}
            & 7B  & 69.83 & 65.30 \\ \cmidrule(lr){1-4} 
        \texttt{mmE5-mllama (instruct)}
            & 11B & 62.54 & 57.83 \\ \cmidrule(lr){1-4} 
        \texttt{QQMM-embed}
            & 8B  & 66.09 & 62.00 \\ \cmidrule(lr){1-4} 
        \multirow{3}{*}{\texttt{LLaVE}}
            & 0.5B & 56.59 & 51.76 \\ \cmidrule(lr){2-4} 
            & 2B   & 62.01 & 57.09 \\ \cmidrule(lr){2-4} 
            & 7B   & 67.14 & 62.13 \\ \cmidrule(lr){1-4} 
        \multirow{2}{*}{\texttt{VLM2Vec (Qwen2-VL)}}
            & 2B   & 47.57 & 42.58 \\ \cmidrule(lr){2-4} 
            & 7B   & 53.24 & 47.68 \\
        \bottomrule
    \end{tabular}
    \vspace{-1mm}
    \caption{
    \updated{Retrieval accuracy compared with different pretrained embedders.}
    }
    \label{tab:embedder_bias}
    \vspace{-2mm}
\end{table*}










%%%%%%%%%%%%%%%%%%%%%%%%%%%%%%%%%%%%%%%%%%%%%%%%%%%%%%%%%%%%%%%%%%%%%%%%%%%%%%%%%%%%%%%%%%%%%%%%%%%%%%%%%%%%%%%%%%%
%%%%%%%%%%%%%%%%%%%%%%%%%%%%%%%%%%%%%%%%%%%%%%%%%%%%%%%%%%%%%%%%%%%%%%%%%%%%%%%%%%%%%%%%%%%%%%%%%%%%%%%%%%%%%%%%%%%
%%%%%%%%%%%%%%%%%%%%%%%%%%%%%%%%%%%%%%%%%%%%%%%%%%%%%%%%%%%%%%%%%%%%%%%%%%%%%%%%%%%%%%%%%%%%%%%%%%%%%%%%%%%%%%%%%%%
%%%%%%%%%%%%%%%%%%%%%%%%%%%%%%%%%%%%%%%%%%%%%%%%%%%%%%%%%%%%%%%%%%%%%%%%%%%%%%%%%%%%%%%%%%%%%%%%%%%%%%%%%%%%%%%%%%%



\subsection{Parameter Sensitivity}
\label{sec:parameter_sensitivity}



\begin{figure}[t]
\centering
\small
\begin{tabular}
{@{}c@{}c@{}}
    { \includegraphics[width=.545\columnwidth]{figures/plot_beamwidth/AVERAGE.pdf}} &
    { \includegraphics[width=.415\columnwidth]{figures/plot_iterations/AVERAGE_num_iters.pdf}   } \\
    (a) Beam width &
    (b) Number of \\     
                    & 
        iterations \\
\end{tabular}
    \vspace{-3mm}
    \caption{Change in retrieval accuracy with varying parameter values.}
    \label{fig:parameter_sensitivity}
    \vspace{-4mm}
\end{figure}

%------------------ Appendix: parameter-sensitivity plots ------------------
\begin{figure*}[t]
  \centering
  \small
  %––––– helper settings ––––––––––––––––––––––––––––––––––––––––––––––
  \newcommand{\imgwa}{0.45\linewidth}   % common width for each cell
  \newcommand{\imgwb}{0.33\linewidth}   % common width for each cell
  % adjust these paths if your folders are named differently
  \newcommand{\beam}[1]{\includegraphics[width=\imgwa]{%
      figures/plot_beamwidth/#1.pdf}}
  \newcommand{\iter}[1]{\includegraphics[width=\imgwb]{%
      figures/plot_iterations/#1.pdf}}
  \setlength{\tabcolsep}{3pt}
  %–––––––––––––––––––––––––––––––––––––––––––––––––––––––––––––––––––––
  \begin{tabular}{@{}c@{}c@{}}

        %––– MP-DocVQA –
    \beam{MP-DocVQA} & \iter{MP-DocVQA} \\
    (g) \texttt{MP-DocVQA}: $b$ & (h) \texttt{MP-DocVQA}: $n_i$ \\

        %––– SlideVQA ––
    \beam{SlideVQA} & \iter{SlideVQA} \\
    (g) \texttt{SlideVQA}: $b$ & (h) \texttt{SlideVQA}: $n_i$ \\
  
    %––– InfoVQA –––
    \beam{InfoVQA} & \iter{InfoVQA} \\
    (g) \texttt{InfoVQA}: $b$ & (h) \texttt{InfoVQA}: $n_i$ \\

    %––– MMWebQA ––
    \beam{MMWebQA} & \iter{MMWebQA} \\
    (g) \texttt{MultimodalQA}: $b$ & (h) \texttt{MultimodalQA}: $n_i$ \\

    %––– MMCoQA –––
    \beam{MMCoQA} & \iter{MMCoQA} \\
    (g) \texttt{MMCoQA}: $b$ & (h) \texttt{MMCoQA}: $n_i$ \\

   
  \end{tabular}
  \vspace{-3mm}
  \caption{Parameter-sensitivity analysis for each dataset: effect of
           beam width $b$ (left) and number of iterations $n_i$ (right).}
  \label{fig:appendix_param_sensitivity}
\end{figure*}
%------------------------------------------------------------------------


%------------------------- Appendix figure -------------------------
\begin{figure*}[t]
  \centering
  \small
  %––– a couple of helpers to keep the code short ––––––––––––––––––
  \newcommand{\imgw}{0.37\linewidth}     % common width for each cell
  \newcommand{\runtime}[1]{\includegraphics[width=\imgw]{figures/plot_runtime/#1.pdf}}
  \newcommand{\breakdown}[1]{\includegraphics[width=\imgw]{figures/plot_breakdown/#1.pdf}}
  \setlength{\tabcolsep}{3pt}            % tighten column padding
  %––––––––––––––––––––––––––––––––––––––––––––––––––––––––––––––––––
  \begin{tabular}{@{}c@{}c@{}}
  %–––– MP-DocVQA ––
    \runtime{MP-DocVQA} & \breakdown{MP-DocVQA} \\
    (a) \texttt{MP-DocVQA} runtime comparison & 
    (b) \texttt{LILaC}'s runtime breakdown on \texttt{MP-DocVQA} \\

    %–––– SlideVQA –––
    \runtime{SlideVQA} & \breakdown{SlideVQA} \\
    (c) \texttt{SlideVQA} runtime comparison &
    (d) \texttt{LILaC}'s runtime breakdown on \texttt{SlideVQA} \\
    
    %–––– InfoVQA ––––
    \runtime{InfoVQA}  & \breakdown{InfoVQA}  \\
    (e) \texttt{InfoVQA} runtime comparison &
    (f) \texttt{LILaC}'s runtime breakdown on \texttt{InfoVQA} \\
    
    %–––– MMWebQA –––
    \runtime{MMWebQA}  & \breakdown{MMWebQA}  \\
    (g) \texttt{MultimodalQA} runtime comparison &
    (h) \texttt{LILaC}'s runtime breakdown on \texttt{MultimodalQA} \\
    
    %–––– MMCoQA ––––
    \runtime{MMCoQA}   & \breakdown{MMCoQA}   \\
    (i) \texttt{MMCoQA} runtime comparison  & 
    (j) \texttt{LILaC}'s runtime breakdown on \texttt{MMCoQA} \\
    
  \end{tabular}
  \vspace{-3mm}
  \caption{Comparison of algorithm execution time (i.e., runtime) for each algorithm per dataset (left) and \texttt{LILaC}'s runtime breakdown per dataset (right).}
  \label{fig:appendix_runtime_breakdown}
\end{figure*}
%-------------------------------------------------------------------

We explored the impact of varying the beam width $b\ (\in \{1, 2, 3, 4, 5, 10, 20, 30\})$ on the retrieval accuracies.
As depicted in Figure~\ref{fig:parameter_sensitivity} (a), retrieval accuracy increased monotonically with larger beam widths, showing a significant improvement of 34.6\% in R@3 when expanding from the minimum of 1 to 30.
This trend highlights the benefit of wider beam searches, enabling more comprehensive and accurate graph traversal. 
Interestingly, despite these substantial accuracy gains, the overall execution time increased only marginally (2.8\%), indicating that graph traversal itself does not constitute the main computational bottleneck. 

Figure~\ref{fig:parameter_sensitivity} (b) presents retrieval accuracy as a function of iteration count $n_i$, varied from 0 to 2. 
We observed a modest yet meaningful 2.93\% improvement in R@3 when transitioning from zero to one iteration. 
This accuracy gain primarily results from enabling multihop reasoning, which is inherently unavailable at $n_i = 0$. 
While the overall increase might appear limited, it is particularly relevant to datasets explicitly requiring complex multihop reasoning, such as \texttt{MultimodalQA} and \texttt{MMCoQA}.


We further analyzed how varying key hyperparameters—beam width $b$ and the number of iterations $n_i$—affect the accuracy of \texttt{LILaC} across the five different datasets. 
We provide comprehensive plots illustrating the sensitivity and robustness of our method concerning these parameters in Figure~\ref{fig:appendix_param_sensitivity}.





\subsection{Layered Component Graph Construction Overhead}
\label{sec:lcg_construction_overhead}

\updated{
\textit{Theoretical complexity.}
We analyze the offline cost of building $\mathcal{G}$ (cf.\ Definition~\ref{def:layered_component_graph}). 
Let $n$ be the number of documents, $c$ the average number of \emph{components} per document, and $s$ the average number of \emph{subcomponents} per component.
The total cost decomposes as
\vspace{-1mm}
\begin{equation*}
\label{eq:tbuild-decomp}
    T_{\text{build}} = T_{\text{nodes}} + T_{\text{edges}} + T_{\text{embed}}
    \vspace{-3mm}
\end{equation*}
}

\updated{
\textit{Node generation.}
We enumerate components in each document and extract subcomponents for every component; we also add the containment links $(C,c)$ to $E_{\downarrow}$.
Enumerating all components across the corpus costs $O(nc)$, and extracting \& linking subcomponents costs $O(ncs)$:
\vspace{-1mm}
\begin{equation*}
\label{eq:tbuild-nodes}
    T_{\text{nodes}} = O(nc) + O(ncs)
    \vspace{-3mm}
\end{equation*}
}

\updated{
\textit{Edge generation.}
Within a document, we form the intra-document clique over $c$ components, yielding $\Theta(c^2)$ edges per document and $O(nc^2)$ overall.
Across documents, we follow the link mapping $\mathcal{L}$; with a hash map for document lookup, retrieving targets is $O(1)$ per link and contributes the same order.
Hence
\vspace{-1mm}
\begin{equation*}
\label{eq:tbuild-edges}
    T_{\text{edges}} = O(nc^2)
\end{equation*}
}

\updated{
\textit{Embedding generation.}
We embed all component and subcomponent nodes using $f$, which scales with their counts:
\begin{equation*}
\label{eq:tbuild-embed}
    T_{\text{embed}} = O(nc) + O(ncs)
    \vspace{-3mm}
\end{equation*}
}

\updated{
Summing the terms gives
% \vspace{-3mm}
\begin{equation*}
\label{eq:tbuild-final}
    T_{\text{build}} = O(ncs + nc^2)
\end{equation*}
In typical regimes where $c,s \ll n$, the offline construction is approximately linear in $n$.
The embedding term is usually dominant; importantly, it is fully offline, cacheable, and parallelizable across documents.
}


\updated{
\textbf{Empirical runtime.}
To validate the scalability, we measured average wall-clock time for each construction stage on increasing corpus fractions (20\%\,/\,40\%\,/\,60\%\,/\,80\%\,/\,100\%), holding the encoder $f$ and batching fixed.
Results are shown in Table~\ref{tab:lcg_offline_runtime}. 
They closely follow the above analysis, exhibiting near-linear growth in $n$ and revealing embedding as the primary bottleneck.
At 100\% of the data, total build time is 2h\,15m\,57s; embedding accounts for ${\sim}90.23\%$ of the cost, with node and edge generation contributing ${\sim}7.60\%$ and ${\sim}2.13\%$, respectively.
We emphasize that the offline cost can be further reduced via batching, sharding, and incremental updates when documents are added or modified.
}


\subsection{Comparison of Diverse Embedders}
\label{sec:embedder_comparison}

\updated{
To probe how biases in pretrained embeddings manifest in retrieval, we hold the \texttt{LILaC} pipeline fixed and vary only the multimodal embedder.
We evaluate seven families—\texttt{MM-Embed}, \texttt{UniME} (LLaVA-OneVision-7B-LoRA-Res336), \texttt{mmE5-mllama} (11B, instruct), \texttt{QQMM-embed}, \texttt{LLaVE} (0.5B/2B/7B), and \texttt{VLM2Vec} (Qwen2-VL; 2B/7B)—and report Recall@3 and MRR@10.}

\updated{
In Table~\ref{tab:embedder_bias}, we observe a consistent scaling trend: within \texttt{LLaVE}, Recall@3 improves by $+9.5\%$ relative from 0.5B to 2B ($62.01{-}56.59$ over $56.59$) and a further $+8.2\%$ from 2B to 7B; within \texttt{VLM2Vec}, 7B exceeds 2B by $+11.9\%$.
Overall, the top performers are \texttt{MM-Embed}, \texttt{UniME}, and \texttt{LLaVE}-7B.
These results indicate that LILaC's retrieval quality is sensitive to the inductive biases of the underlying encoder, yet benefits directly from stronger, larger models.
}









\subsection{Algorithm Execution Runtime: Further Analysis}

We conducted an in-depth examination of runtime efficiency. 
Specifically, we compared the overall execution time of our proposed method, \texttt{LILaC}, against other baseline algorithms across all datasets. 
We further broke down \texttt{LILaC}'s runtime into individual components (such as retrieval, reranking, and LLM refinement) to clearly identify performance bottlenecks and highlight the efficiency of different pipeline stages.
Detailed results are shown in Figure~\ref{fig:appendix_runtime_breakdown}.
























\section{Prompt Templates}
\label{sec:prompt_templates}

We present detailed examples of the specific prompt templates used in our experiments. 
These prompts correspond to three key tasks: \textsc{Object Detection}, \textsc{Query Decomposition}, \textsc{Modality Selection} and \textsc{Answer Generation}. 
For each task, we provide clear instructions, expected input-output formats, and task-specific heuristics.


\begin{figure*}[htb]

\begin{tcolorbox}
[title = \textsc{Object Detection}, colback = gray!10, colframe = black, sharp corners, boxrule=0.5mm]

\textbf{Instruction}: \\
Detect all objects in the image and return ONLY a JSON list of {\texttt{\{class, bbox\_2d: [x1, y1, x2, y2]\}}}. Do NOT include markdown or extra text. \\
\\
\textbf{Image:} {\texttt{\{image\}}} \\
\textbf{Output:}
\end{tcolorbox}

\label{fig:object_detection}
\end{figure*}


\begin{figure*}[htb]

\begin{tcolorbox}
[title = \textsc{Query Decomposition}, colback = gray!10, colframe = black, sharp corners, boxrule=0.5mm]

\textbf{Instruction:} \\
You are a retrieval-oriented query decomposer. \\
\\
Goal – Produce the smallest set (1 – 5) of component-targeting sub-queries.  \\
Each sub-query must describe one retrievable component (sentence, paragraph, table row, figure, etc.) whose embedding should be matched.  \\
Together, the sub-queries must supply all the information needed to answer the original question. \\
\\
\textbf{Guidelines:} \\
1. Entity \& noun-phrase coverage: Every noun phrase and named entity that appears in the original question must appear at least once across the entire set of sub-queries (you may distribute them). 
Keep each phrase exactly as written. \\ 
2. One-component rule: A sub-query should reference only the facts expected to co-occur within the same component. If two facts will likely be in different components, put them in different sub-queries. \\ 
3. No unnecessary splitting: If the whole answer can be found in a single component, return only one sub-query. \\ 
4. De-contextualize: Rewrite pronouns and implicit references so every sub-query is understandable on its own. \\ 
5. Keyword distribution: Spread constraints logically (e.g., one sub-query for “light rail completion date”, another for “city with a large arched bridge from the 1997 Australia rugby-union test match”). \\ 
6. Remove redundancy: Merge duplicate or paraphrased sub-queries before you output. \\ 
7. Ordering for dependencies: If the answer to one sub-query is needed for another, place the prerequisite first. \\ 
8. Output format: Return only a JSON array of strings — no keys, explanations, or extra text. \\
\\
\textbf{Question:} {\texttt{\{question\}}} \\
\textbf{Output:}
\end{tcolorbox}
\label{fig:query_decomposition}
\end{figure*}



\begin{figure*}[htb]
\begin{tcolorbox}
[title = \textsc{Modality Selection}, colback = gray!10, colframe = black, sharp corners, boxrule=0.5mm]

\textbf{Instruction:} \\
You are a modality selector for multimodal QA. \\
\\
\textbf{Task:} \\
Given the single sub-question below, choose the one modality that is most appropriate for obtaining its answer. \\
\\
\textbf{Allowed modalities:}  \\
• text: unstructured prose (paragraphs, sentences, propositions) \\ 
• table: structured rows/columns (spreadsheets, stats tables, infoboxes) \\ 
• image: visual information (photos, posters, logos, charts) \\
\\
\textbf{Heuristics:}  \\
1. Numeric totals, percentages, year-by-year figures $\rightarrow$ table \\ 
2. Visual appearance, colours, logos, “what does … look like” $\rightarrow$ image \\ 
3. Definitions, roles, biographies, causal explanations, quotes $\rightarrow$ text \\ 
4. If two modalities could work, pick the one that will yield the answer fastest. \\
\\
\textbf{Output format:} \\
Return only the modality label on a single line – exactly \texttt{text}, \texttt{table}, or \texttt{image}. \\ 
No JSON, no additional text. \\
\\
\textbf{Subquery:} {\texttt{\{subquery\}}} \\
\textbf{Output:}
\end{tcolorbox}
\label{fig:modality_selection}
\end{figure*}








% \begin{figure*}[htb]
% \begin{tcolorbox}[
%     title = Answer Generation Prompt,
%     colback = gray!10,
%     colframe = black,
%     sharp corners,
%     boxrule = 0.5mm
% ]
% Below is an instruction that describes a task, paired with an input that provides further context.  
% Write a response that appropriately completes the request.\\
% \\
% \textbf{Instruction:} \\
% Using the \texttt{f\_answers()} API, return a list of answers to the question based on \emph{retrieved webpage components}.
% A retrieved component can be a passage, a table, or an image.
% Strictly follow the format of the example below and keep the answer short.
% For \emph{yes/no} questions, respond only with \texttt{f\_answers(["yes"])} or \texttt{f\_answers(["no"])}. \\
% \\
% \textbf{Example:} \\

% \texttt{Passage} \\
% Title: South Asia \\
% The current territories of Afghanistan, Bangladesh, Bhutan, Maldives, Nepal, India, Pakistan, and Sri Lanka form South Asia. The South Asian Association for Regional Cooperation (SAARC) is an economic cooperation organisation established in 1985 that includes all eight nations comprising South Asia. \\
% \\
% \texttt{Passage} \\
% Title: UK Joint Expeditionary Force \\
% The UK Joint Expeditionary Force (JEF) is a United Kingdom-led expeditionary force which may include Denmark, Finland, Estonia, Latvia, Lithuania, the Netherlands, Sweden, and Norway. It is distinct from the Franco-British Combined Joint Expeditionary Force. \\
% \\
% \texttt{Table} \\
% Title: Lithuanian Armed Forces — Current operations \\

% Deployment | Organization | Operation | Personnel \\
% Somalia | EU | Operation Atalanta | 15 \\
% Mali | EU | EUTM Mali | 2 \\
% Afghanistan | NATO | Operation Resolute Support | 29 \\
% Libya | EU | EU Navfor Med | 3 \\
% Mali | UN | MINUSMA | 39 \\
% Iraq | CJTF | Operation Inherent Resolve | 6 \\
% Central African Republic | EU | EUFOR RCA | 1 \\
% Kosovo | NATO | KFOR | 1 \\
% Ukraine | — | Training mission | 40 \\
% \\
% Question: Among the Lithuanian Armed Forces' current operations, which deployment involves fewer personnel: \emph{Kosovo}, or the deployment in the nation that, along with six others, constitutes the sub-continent of South Asia? \\
% Answer: The South Asia passage shows Afghanistan is part of that region. The table lists 29 personnel in Afghanistan and only 1 in Kosovo, so \texttt{f\_answers(["Kosovo"])}. \\
% \\
% \textbf{Input:} \\
% Using the images and texts given, answer the question below in a single word or phrase. \\
% \\
% \textbf{Question:} \texttt{\{question\}} \\
% \textbf{Answer:}
% \end{tcolorbox}
% \label{fig:answer_generation_prompt}
% \end{figure*}



\begin{figure*}[htb]
\begin{tcolorbox}[
  enhanced,
  title = \textsc{Answer Generation},
  colback = gray!10,
  colframe = black,
  sharp corners,
  boxrule = 0.5mm
]

\textbf{Instruction:}\\
Using the \texttt{f\_answers()} API, return a list of answers to the question based on \emph{retrieved webpage components}.
A retrieved component can be a passage, a table, or an image.
Strictly follow the format of the example below and keep the answer short.
For \emph{yes/no} questions, respond only with \texttt{f\_answers(["yes"])} or \texttt{f\_answers(["no"])}.\\

\textbf{Example:}\\[-2pt]
\noindent\rule{\linewidth}{0.5pt} % ---------- top wrapping line ----------

\texttt{[Passage]}\\
Document title: South Asia\\
The current territories of Afghanistan, Bangladesh, Bhutan, Maldives, Nepal, India, Pakistan, and Sri Lanka form South Asia. The South Asian Association for Regional Cooperation (SAARC) is an economic cooperation organisation established in 1985 that includes all eight nations comprising South Asia. \\

\texttt{[Passage]}\\
Document title: UK Joint Expeditionary Force\\
The UK Joint Expeditionary Force (JEF) is a United Kingdom-led expeditionary force which may include Denmark, Finland, Estonia, Latvia, Lithuania, the Netherlands, Sweden, and Norway. It is distinct from the Franco-British Combined Joint Expeditionary Force. \\

\texttt{[Table]}\\
Document title: Lithuanian Armed Forces — Current operations\\[2pt]
\TextHeader
\TextRow{Somalia}{EU}{Operation Atalanta}{15}
\TextRow{Mali}{EU}{EUTM Mali}{2}
\TextRow{Afghanistan}{NATO}{Operation Resolute Support}{29}
\TextRow{Libya}{EU}{EU Navfor Med}{3}
\TextRow{Mali}{UN}{MINUSMA}{39}
\TextRow{Iraq}{CJTF}{Operation Inherent Resolve}{6}
\TextRow{Central African Republic}{EU}{EUFOR RCA}{1}
\TextRow{Kosovo}{NATO}{KFOR}{1}
\TextRow{Ukraine}{—}{Training mission}{40}

\vspace{4mm}
\textbf{Question:} Among the Lithuanian Armed Forces' current operations, which deployment involves fewer personnel: Kosovo, or the deployment in the nation that, along with six others, constitutes the sub-continent of South Asia? \\
\textbf{Answer:} The South Asia passage shows Afghanistan is part of that region. The table lists 29 personnel in Afghanistan and only 1 in Kosovo, so \texttt{f\_answers(["Kosovo"])}.
\noindent\rule{\linewidth}{0.5pt} 
% ---------- bottom wrapping line ----------
\\ \\
Using the images and texts given, answer the question below in a single word or phrase.\\ \\
\texttt{\{retrieved components\}}\\ \\
\textbf{Question:} \texttt{\{question\}}\\
\textbf{Answer:}
\end{tcolorbox}
\label{fig:answer_generation_prompt}
\end{figure*}


\end{document}
